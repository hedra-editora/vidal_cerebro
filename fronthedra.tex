\AtBeginDocument{%
\begingroup\pagestyle{empty}\raggedright\parindent0pt
{\Formular{\Huge{\titulagemfront{}}}}
\vspace{11.25mm}

\Large{\autor}
\vfill
\clearpage

%% Créditos ------------------------------------------------------
\raggedright
\linha{título original}{\titulooriginal}
\linha{título em português}{\tituloportugues}
\linha{copyright©}{\copyrightlivro}
\linhalayout{edição brasileira©}{n-1 edições / Hedra \ifdef{\ano}{\ano}{\the\year}}
\linha{tradução©}{\copyrighttraducao}
\linha{organização©}{\copyrightorganizacao}
\linha{prefácio©}{\copyrightintroducao}
\linha{ilustração©}{\copyrightilustracao}\smallskip
\linha{edição consultada}{\edicaoconsultada}
\linha{primeira edição}{\primeiraedicao}
\linha{agradecimentos}{\agradecimentos}
\linha{indicação}{\indicacao}\smallskip
\linha{edição}{\edicao}
\linha{direção de arte}{\direcaodearte}
\linha{coedição}{\coedicao}
\linha{assistência editorial}{\assistencia}
\linha{revisão}{\revisao}
\linha{preparação}{\preparacao}
\linha{iconografia}{\iconografia}
\linha{capa}{\capa}
\linha{imagem da capa}{\imagemcapa}\smallskip
\linha{ISBN}{\ISBN}\smallskip
\begingroup\tiny
%\ifdef{\conselho}{\conselho}{\relax}
\par\endgroup\bigskip

\begingroup \tiny

\textit{Grafia atualizada segundo o Acordo Ortográfico da Língua\\
Portuguesa de 1990, em vigor no Brasil desde 2009.}\\

\vspace{35pt}

\begin{flushleft}
\hspace{10pt}Dados Internacionais de Catalogação na Publicação (CIP) de acordo com ISBD
\_\_\_\_\_\_\_\_\_\_\_\_\_\_\_\_\_\_\_\_\_\_\_\_\_\_\_\_\_\_\_\_\_\_\_\_\_\_\_\_\_\_\_\_\_\_\_\_\_\_\_\_\_\_\_\_\_\_\_\_\_\_\_\_\_\_\_\_\_\_\_\_\_\_\_\_\_\_\_\_\_\_\_\_\_\_\_\_\_\_\\
V648s \hspace{10.3pt}Vidal, Fernando\\[4pt]
\hspace{30pt}\parbox{230pt}{Somos nosso cérebro: neurociências, subjetividade e cultura / Fernando Vidal,\linebreak Francisco Ortega ; Traduzido por Alexandre Martins. -- n-1 Edições, 2019}\\[4pt]

\hspace{30pt}p. 346; 23cm x 16cm.\\[6pt]

\hspace{30pt}\parbox{230pt}{Tradução de: Being brains - make the cerebral subjects\\Inclui bibliografia e índice.\\ISBN 978-65-9582-035-7}\\[6pt]

\hspace{30pt}\parbox{230pt}{1. Neurociências. 2. Antropologia médica. 3. Fenomenologia. I. Ortega, Francisco. II. Martins, Alexandre. III. Título.}

\hspace{30pt}2019-2116 \hspace{160pt}\parbox{33pt}{CDD 612.8\\CDU 612.8}
\_\_\_\_\_\_\_\_\_\_\_\_\_\_\_\_\_\_\_\_\_\_\_\_\_\_\_\_\_\_\_\_\_\_\_\_\_\_\_\_\_\_\_\_\_\_\_\_\_\_\_\_\_\_\_\_\_\_\_\_\_\_\_\_\_\_\_\_\_\_\_\_\_\_\_\_\_\_\_\_\_\_\_\_\_\_\_\_\_\_\\
\hspace{10pt}Elaborado por Vagner Rodolfo da Silva - CRB-8/9410 
\end{flushleft} 

\begin{centering}
\textbf{Índice para catálogo sistemático:}\\
1. Neurociências 612.8\\
2. Neurociências 612.8

\end{centering}
\endgroup

\begingroup \tiny

\vfill\textit{Direitos reservados em l\'ingua\\ portuguesa somente para o Brasil}\\\medskip

%
\textsc{n-1 edições ltda.}\\
R.~Frei Caneca, 322 | cj. 52\\
01307--000 São Paulo \textsc{sp} Brasil\\\smallskip
oi@n-1edicoes.org\\
www.n-1publications.org\\
\bigskip
Foi feito o depósito legal.\\\endgroup
\pagebreak\raggedright
%% Front ---------------------------------------------------------
% Titulo
{\Formular{\Huge{\titulagem}}}
\vspace{11.25mm}

{\Large{\autor} \par}%\vspace{1.5ex}}
\vspace{6cm} %9.3cm
\ifdef{\organizador}{{\small {\organizador} (\textit{organização} e \textit{tradução})} \par}{}
\ifdef{\introdutor}{{\small {\introdutor} (\textit{prefácio})} \par}{}\vspace{8.5mm}
\ifdef{\tradutor}{{\normalsize {\tradutor} (\textit{tradução})}\par}{}

{{\footnotesize{} \ifdef{\numeroedicao}{\numeroedicao}{1}ª edição} \par}
%logos
\vfill
\normalsize
%\ifdef{\logo}{\IfFileExists{\logo}{\hfill\includegraphics[width=3cm]{\logo}\hfill\logoum{}\\ São Paulo\_\the\year}}{\logoum\break{} São Paulo\_\the\year}
%\includegraphics[width=.4\textwidth,trim=0 0 25 0]{logo.jpg}\\\smallskip
\par\clearpage\endgroup
% Resumo -------------------------------------------------------
\begingroup \footnotesize \parindent0pt \parskip 5pt \thispagestyle{empty} \vspace*{-0.5\textheight}\mbox{} \vfill
\baselineskip=.98\baselineskip
\IfFileExists{PRETAS.tex}{\textbf{Charles Baudelaire} (Paris, 1821---\textit{id}. 1867), escritor francês, é
 hoje reverenciado como um dos paradigmas máximos da modernidade.
 Dono de uma imagética pujante e original, Baudelaire foi também um
 influente crítico de arte e um tradutor de grande envergadura. Alma
 inquieta e conturbada, via com desconfiança a era do progresso,
 entrevendo na modernidade uma morbidez oculta que sua sensibilidade
 extremada não tolerava. Em 1857, a publicação de \textit{As flores do mal}, sua
 obra-prima, ofende a moral burguesa e lhe vale um processo no qual é
 obrigado a pagar uma multa considerável, além de ter de retirar sete poemas do livro.
 Alguns dos sonetos ali encerrados já prefiguravam o simbolismo e o decadentismo, correntes que começavam a
 tomar corpo. Em \textit{Os paraísos artificiais} (1860), explora o potencial
 criador sob o efeito do ópio e do haxixe. Como tradutor, verte
 muitos dos contos e ensaios de  Edgar Allan Poe para o francês, tendo influído assim decisivamente
 para o futuro reconhecimento desse autor, que 
 exerceu influência em sua obra também. Solitário, doente e sem recursos,
 morre em 1867.

\textbf{Escritos sobre arte} reúne quatro textos da produção crítica de Baudelaire: 
``Da essência do riso'' (\textit{Le Portefeuille}, 1855), ``Alguns caricaturistas estrangeiros'' 
(\textit{Le Présent, 1857}), ``A arte filosófica'' (original encontrado entre papéis 
de Baudelaire e publicado postumamente em \textit{Arte romântica}) e 
``A obra e a vida de Eugène Delacroix'' (\textit{L'Opinion Nationale}, 1863).
Produzidos para periódicos, sem o objetivo de serem posteriormente coligidos em livro,
estes ensaios apresentam um conjunto de reflexões estéticas incomuns para o período, 
como o riso na caricatura, a definição da arte filosófica e a recuperação de autores pouco valorizados. 

\textbf{Plínio Augusto Coêlho} fundou em 1984 a Novos Tempos Editora, em
Brasília, dedicada à publicação de obras libertárias. A partir de
1989, transfere-se para São Paulo, onde cria a Editora Imaginário,
mantendo a mesma linha de publicações. É idealizador
e co-fundador do \textsc{iel} (Instituto de Estudos Libertários).

\textbf{Dirceu Villa} é poeta, tradutor e mestre em letras pela Universidade
de São Paulo. Autor do livro de poemas \textit{Descort} (Hedra, 2003), traduziu
\textit{Lustra}, de Ezra Pound (inédito) e colabora em diversos veículos de imprensa.



}{% 
\ifdef{\resumo}{\resumo\par}{}
\ifdef{\sobreobra}{\sobreobra}{}
\ifdef{\sobreautor}{\mbox{}\vspace{4pt}\newline\sobreautor}{}
\ifdef{\sobretradutor}{\newline\sobretradutor}{\relax}
\ifdef{\sobreorganizador}{\vspace{4pt}\newline\sobreorganizador}{\relax}\par}
\thispagestyle{empty} \endgroup
\ifdefvoid{\sobreautor}{}{\pagebreak\ifodd\thepage\paginabranca\fi}
% Sumário -------------------------------------------------------

\sumario{}
%\IfFileExists{INTRO.tex}{\include{INTRO}}

%\IfFileExists{TEXTO.tex}{\mbox{}\include{TEXTO}}
%\part[{{\def\break{}\titulo}}]{\titulo}
} % fim do AtBeginDocument

% Finais -------------------------------------------------------
\AtEndDocument{%


\pagebreak\ifodd\thepage\paginabranca\fi

\ifdef{\imagemficha}{\IfFileExists{\imagemficha}{\includegraphics[width=.7\textwidth]{\imagemficha}\par}}{}

\mbox{}\vfill\small\thispagestyle{empty}
\begin{center}
\begin{minipage}{.8\textwidth}
\centering\tiny\noindent{}Adverte-se aos curiosos que se imprimiu este livro
em \today \ifdef{\papelmiolo}{em papel \papelmiolo}, em tipologia Formular e \tipopadrao{}, com diversos sofwares livres, 
entre eles, Lua\LaTeX, git \& ruby. \ifdef{\RevisionInfo{}}{\par(v.\,\RevisionInfo)}{}\par \begin{center}\normalsize\adforn{64}\end{center}
\end{minipage}
\end{center}
}
