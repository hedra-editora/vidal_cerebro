\AtBeginDocument{%
\begingroup\pagestyle{empty}\raggedright\parindent0pt
{\Formular{\Huge{\titulagemfront{}}}}
\vspace{11.25mm}

\Large{\autor}
\vfill
\clearpage

%% Créditos ------------------------------------------------------
\raggedright
\linha{título original}{\titulooriginal}
\linha{título em português}{\tituloportugues}
\linha{copyright©}{\copyrightlivro}
\linhalayout{edição brasileira©}{n-1 edições / Hedra \ifdef{\ano}{\ano}{\the\year}}
\linha{tradução©}{\copyrighttraducao}
\linha{organização©}{\copyrightorganizacao}
\linha{prefácio©}{\copyrightintroducao}
\linha{ilustração©}{\copyrightilustracao}\smallskip
\linha{edição consultada}{\edicaoconsultada}
\linha{primeira edição}{\primeiraedicao}
\linha{agradecimentos}{\agradecimentos}
\linha{indicação}{\indicacao}\smallskip
\linha{edição}{\edicao}
\linha{direção de arte}{\direcaodearte}
\linha{coedição}{\coedicao}
\linha{assistência editorial}{\assistencia}
\linha{revisão}{\revisao}
\linha{preparação}{\preparacao}
\linha{iconografia}{\iconografia}
\linha{capa}{\capa}
\linha{imagem da capa}{\imagemcapa}\smallskip
\linha{ISBN}{\ISBN}\smallskip
\begingroup\tiny
%\ifdef{\conselho}{\conselho}{\relax}
\par\endgroup\bigskip

\begingroup \tiny

\textit{Grafia atualizada segundo o Acordo Ortográfico da Língua\\
Portuguesa de 1990, em vigor no Brasil desde 2009.}\\

\vspace{35pt}

\begin{flushleft}
\hspace{10pt}Dados Internacionais de Catalogação na Publicação (CIP) de acordo com ISBD
\_\_\_\_\_\_\_\_\_\_\_\_\_\_\_\_\_\_\_\_\_\_\_\_\_\_\_\_\_\_\_\_\_\_\_\_\_\_\_\_\_\_\_\_\_\_\_\_\_\_\_\_\_\_\_\_\_\_\_\_\_\_\_\_\_\_\_\_\_\_\_\_\_\_\_\_\_\_\_\_\_\_\_\_\_\_\_\_\_\_\\
V648s \hspace{10.3pt}Vidal, Fernando\\[4pt]
\hspace{30pt}\parbox{230pt}{Somos nosso cérebro: neurociências, subjetividade e cultura / Fernando Vidal,\linebreak Francisco Ortega ; Traduzido por Alexandre Martins. -- n-1 Edições, 2019}\\[4pt]

\hspace{30pt}p. 346; 23cm x 16cm.\\[6pt]

\hspace{30pt}\parbox{230pt}{Tradução de: Being brains - make the cerebral subjects\\Inclui bibliografia e índice.\\ISBN 978-65-9582-035-7}\\[6pt]

\hspace{30pt}\parbox{230pt}{1. Neurociências. 2. Antropologia médica. 3. Fenomenologia. I. Ortega, Francisco. II. Martins, Alexandre. III. Título.}

\hspace{30pt}2019-2116 \hspace{160pt}\parbox{33pt}{CDD 612.8\\CDU 612.8}
\_\_\_\_\_\_\_\_\_\_\_\_\_\_\_\_\_\_\_\_\_\_\_\_\_\_\_\_\_\_\_\_\_\_\_\_\_\_\_\_\_\_\_\_\_\_\_\_\_\_\_\_\_\_\_\_\_\_\_\_\_\_\_\_\_\_\_\_\_\_\_\_\_\_\_\_\_\_\_\_\_\_\_\_\_\_\_\_\_\_\\
\hspace{10pt}Elaborado por Vagner Rodolfo da Silva - CRB-8/9410 
\end{flushleft} 

\begin{centering}
\textbf{Índice para catálogo sistemático:}\\
1. Neurociências 612.8\\
2. Neurociências 612.8

\end{centering}
\endgroup

\begingroup \tiny

\vfill\textit{Direitos reservados em l\'ingua\\ portuguesa somente para o Brasil}\\\medskip

%
\textsc{n-1 edições ltda.}\\
R.~Frei Caneca, 322 | cj. 52\\
01307--000 São Paulo \textsc{sp} Brasil\\\smallskip
oi@n-1edicoes.org\\
www.n-1publications.org\\
\bigskip
Foi feito o depósito legal.\\\endgroup
\pagebreak\raggedright
%% Front ---------------------------------------------------------
% Titulo
{\Formular{\Huge{\titulagem}}}
\vspace{11.25mm}

{\Large{\autor} \par}%\vspace{1.5ex}}
\vspace{6cm} %9.3cm
\ifdef{\organizador}{{\small {\organizador} (\textit{organização} e \textit{tradução})} \par}{}
\ifdef{\introdutor}{{\small {\introdutor} (\textit{prefácio})} \par}{}\vspace{8.5mm}
\ifdef{\tradutor}{{\normalsize {\tradutor} (\textit{tradução})}\par}{}

{{\footnotesize{} \ifdef{\numeroedicao}{\numeroedicao}{1}ª edição} \par}
%logos
\vfill
\normalsize
%\ifdef{\logo}{\IfFileExists{\logo}{\hfill\includegraphics[width=3cm]{\logo}\hfill\logoum{}\\ São Paulo\_\the\year}}{\logoum\break{} São Paulo\_\the\year}
%\includegraphics[width=.4\textwidth,trim=0 0 25 0]{logo.jpg}\\\smallskip
\par\clearpage\endgroup
% Resumo -------------------------------------------------------
\begingroup \footnotesize \parindent0pt \parskip 5pt \thispagestyle{empty} \vspace*{-0.5\textheight}\mbox{} \vfill
\baselineskip=.98\baselineskip
\IfFileExists{PRETAS.tex}{\textbf{Fernando Vidal} é professor de investigação do \versal{ICREA} (Instituto
Catalão de Pesquisa e Estudos Avançados) e do Centro de Investigação em
Antropologia Médica (\versal{MARC}) da Universidad Rovira i Virgili de Tarragona,
Espanha. Formado pela Universidade de Harvard e pós"-graduado pelas
Universidades de Genebra e Paris e pela École des Hautes Études en
Sciences Sociales (Paris), foi pesquisador visitante na Academia
Americana de Roma, na Universidade de Harvard e na Fundação Brocher e
professor visitante em universidades em Buenos Aires, Paris, Rio de
Janeiro, México, Taipei e Kyoto. É membro associado do Centro Alexandre
Koyré (Paris) e foi eleito na Academia Europeia. Tem trabalhado
amplamente sobre temas da história intelectual e cultural das ciências
da mente e do cérebro desde o início da época moderna até o presente.
Seu principal projeto atual combina ética biomédica, antropologia
médica, estudos da deficiência, estudos sociais da ciência e
fenomenologia para explorar como os transtornos da consciência se
articulam com noções e práticas da pessoa e a criação de subjetividades.
Foi organizador de \emph{Jean Starobinski -- Las razones del cuerpo}
(1999), \emph{The Moral Authority of Nature} (com Lorraine Daston -
2004), \emph{Neurocultures: Glimpses into an Expanding Universe }(com
Francisco Ortega, 2011), \emph{Endangerment, Biodiversity and
Culture} (com Nélia Dias, 2015) e autor, entre outros livros, de \emph{The
Sciences of the Soul: The Early Modern Origins of Psychology} (2011).

\textbf{Francisco Ortega} é professor titular do Instituto de Medicina
Social da Universidade do Estado do Rio de Janeiro, diretor de pesquisa
do Centro Rio de Saúde Global, professor visitante do Departamento de
Saúde Global e Medicina Social do King's College de Londres e
pesquisador do \versal{CNP}q. Formado em Filosofia pela Universidade
Complutense de Madri, fez doutorado na Universidade de Bielefeld,
Alemanha. Foi professor visitante em universidades em Londres, Berlim,
Madri, Buenos Aires, Oldenburg e Bielefeld. É membro do Advisory
Board do Movement for Global Mental Health e do Steering Committee do
Global Social Medicine Network. Seus diversos interesses combinam
história e filosofia da ciência, fenomenologia, antropologia médica,
psiquiatria transcultural, estudos da deficiência, saúde coletiva e
saúde global em um enfoque interdisciplinar que examina as formas pelas
quais as ciências biomédicas e as práticas de saúde contribuem para
moldar a identidade pessoal com base nas características corporais, a
formação de identidades sociais e pessoais informadas pelo conhecimento
biomédico, a redefinição de fronteiras entre as ciências da vida e as
ciências sociais e humanas, e a interseção entre a biopsiquiatria global
e as epistemologias psiquiátricas locais. É autor, entre outros livros, de
\emph{Corporeality, Medical Technologies and Contemporary Culture} (2014,
traduzido para o português, espanhol e italiano) e organizou, com
Fernando Vidal, \emph{Neurocultures: Glimpses into an Expanding Universe}
(2011).


}{% 
\ifdef{\resumo}{\resumo\par}{}
\ifdef{\sobreobra}{\sobreobra}{}
\ifdef{\sobreautor}{\mbox{}\vspace{4pt}\newline\sobreautor}{}
\ifdef{\sobretradutor}{\newline\sobretradutor}{\relax}
\ifdef{\sobreorganizador}{\vspace{4pt}\newline\sobreorganizador}{\relax}\par}
\thispagestyle{empty} \endgroup
\ifdefvoid{\sobreautor}{}{\pagebreak\ifodd\thepage\paginabranca\fi}
% Sumário -------------------------------------------------------

\sumario{}
%\IfFileExists{INTRO.tex}{\include{INTRO}}

%\IfFileExists{TEXTO.tex}{\mbox{}\include{TEXTO}}
%\part[{{\def\break{}\titulo}}]{\titulo}
} % fim do AtBeginDocument

% Finais -------------------------------------------------------
\AtEndDocument{%


\pagebreak\ifodd\thepage\paginabranca\fi

\ifdef{\imagemficha}{\IfFileExists{\imagemficha}{\includegraphics[width=.7\textwidth]{\imagemficha}\par}}{}

\mbox{}\vfill\small\thispagestyle{empty}
\begin{center}
\begin{minipage}{.8\textwidth}
\centering\tiny\noindent{}Adverte-se aos curiosos que se imprimiu este livro
em \today \ifdef{\papelmiolo}{em papel \papelmiolo}, em tipologia Formular e \tipopadrao{}, com diversos sofwares livres, 
entre eles, Lua\LaTeX, git \& ruby. \ifdef{\RevisionInfo{}}{\par(v.\,\RevisionInfo)}{}\par \begin{center}\normalsize\adforn{64}\end{center}
\end{minipage}
\end{center}
}
