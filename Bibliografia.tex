\chapter{Bibliografia}

\begin{Parskip}
Abbott, Alison. 2011. ``Novartis to Shut Brain Research Facility.''
\emph{Nature} 480 (8 Dez.): 161--162.

Abi"-Rached, Joelle M. 2008. ``The New Brain Sciences: Field or Fields?''
Brain Self and Society working paper no. 2. London: \versal{BIOS}/London School
of Economics and Political Science.

Abi"-Rached, Joelle M., e Nikolas Rose. 2010. ``The Birth of the
Neuromolecular Gaze.'' \emph{History of the Human Sciences} 23:11--36.

Abi"-Rached, Joelle M., Nikolas Rose, e Andrei Mogoutov. 2010. ``Mapping
the Rise of the New Brain Sciences.'' Brain Self and Society working
paper no. 4. \versal{BIOS}/London School of Economics and Political Science.

Abou"-Saleh, Mohammed T. 2006. ``Neuroimaging in Psychiatry: An Update.''
\emph{Journal of Psychosomatic Research} 61:289--293.

Adams, Jon. 2008. ``The Sufficiency of Code, \emph{Galatea 2.2}, and the
Necessity of Embodiment.'' In Burn e Dempsey 2008, 137--150.

Adler, Hans, e Sabine Gross. 2002. ``Adjusting the Frame, Comments on
Cognitivism and Literature.'' \emph{Poetics Today} 23 (2): 195--220.

Albano, Caterina, Ken Arnold, e Marina Wallace, Orgs. 2002. \emph{Head
On: Art with the Brain in Mind}. London: Artakt.

Alcaro, Antonio, et al. 2010. ``Is Subcortical"-Cortical Midline Activity
in Depression Mediated by Glutamate and \versal{GABA}? A Cross"-Species
Translational Approach.'' \emph{Neuroscience and Biobehavioral Reviews}
34:592--605.

Aldworth, Susan. 2011. ``The Physical Brain and the Sense of Self: An
Artist's Exploration.'' In Ortega e Vidal 2011, 273--292.

Aldworth, Susan, Paul Broks, Robert Mason, e Gill Saunders. 2008.
\emph{Scribing the Soul}. London: Susan Aldworth.

Álvarez"-Jiménez, Mario, et al. 2008. ``Non"-pharmacological Management of
Antipsychotic"-Induced Weight Gain: Systematic Review and Meta"-analysis
of Randomised Controlled Trials.'' \emph{British Journal of Psychiatry}
193:101--107.

Ambady, Nalini, e Jamshed Bharucha. 2009. ``Culture and the Brain.''
\emph{Current Directions in Psychological Science} 18 (6): 342--345.

Ames, Daniel L. e Susan T. Fiske. 2010. ``Cultural Neuroscience.''
\emph{Asian Journal of Social Psychology}13:72--82.

Aminoff, J. Michael. 1993. \emph{Brown"-Sequard: A Visionary of Science}.
New York: Raven.

Ananthaswamy, Anil. 2015. \emph{The Man Who Wasn't There: Investigations
Into the Strange New Science of the Self}. New York: Dutton.

Anderson, Ian, e John Camm, Orgs. 2014. \emph{Handbook of Depression}. 2
ed. New York: Springer.

Anderson, Rodney J., et al. 2012. ``Deep Brain Stimulation for
Treatment"-Resistant Depression: Efficacy, Safety, and Mechanisms of
Action.'' \emph{Neuroscience and Biobehavioral Reviews} 36:1920--1933.

Andreasen, Nancy Coover. 2004. \emph{Brave New Brain: Conquering Mental
Illness in the Era of the Genome}. New York: Oxford University Press.
(trad. port.: \emph{Admirável Cérebro Novo: Vencendo a Doença Mental na
Era do Genoma}. Porto Alegre: Artmed, 2005.)

\_\_\_\_\_. 2006. \emph{The Creating Brain: The Neuroscience of Genius}.
New York: Plume.

Angell, Marcia. 2004. \emph{The Truth About the Drug Companies: How They
Deceive Us and What to Do About It}. New York: Random House.

\_\_\_\_\_.2011. ``The Epidemic of Mental Illness: Why?'' \emph{New York
Review of Books} (23 Jun.).
\textless{}\emph{https://bit.ly/1iOlJT2}\textgreater{}.

Angermeyer, Matthias C., e Herbert Matschinger. 2005. ``Causal Beliefs
and Attitudes to People with Schizophrenia: Trend Analysis Based on Data
from Two Population Surveys in Germany.'' \emph{British Journal of
Psychiatry} 186 (3): 331--334.

Anker, Suzanne, e Giovanni Frazzetto. 2006. \emph{Neuroculture: Visual
Art and the Brain}. Westport, Conn.: Westport Arts Center.

Anonymous. 2006. ``Retraining the Brain. Doctors Test Drug"-Free Methods
to Restore Lost Mental Capabilities.'' \emph{\versal{CBS} News} (15 Jan.).
\textless{}\emph{https://cbsn.ws/2LY6ezA}\textgreater{}.

Antonetta, Susanne. 2005. \emph{A Mind Apart: Travels in a Neurodiverse
World}. Tarcher: Penguin.

\versal{APA} 2007. ``Functional Magnetic Resonance Imaging: A New Research
Tool.'' Washington, D.C.: American Psychological Association.
\textless{}\emph{http://www.apa.org/research/tools/fmri"-booklets.aspx}\textgreater{}.

Appelbaum, Kalman. 2006. ``Educating for Global Mental Health: The
Adoption of \versal{SSRI}s in Japan.'' In Petryna, Lakoff, e Kleinman 2006.

Arango, Ángel. 1964. \emph{¿A dónde van los cefalomos?} Havana:
Ediciones R.

Arbabshirani, Mohammad R., et al. 2013. ``Classification of
Schizophrenia Patients Based on Resting"-State Functional Network
Connectivity.'' \emph{Frontiers in Neuroscience} 7, art. 133.
doi:10.3389/fnins.2013.00133.

Ariel, Cindy N., e Robert A. Naseef, Orgs. 2006. \emph{Voices from the
Spectrum: Parents, Grandparents, Siblings, People with Autism, and
Professionals Share Their Wisdom}. London: Jessica Kingsley.

Arikha, Noga. 2007. \emph{Passions and Tempers: A History of the
Humours}. New York: Ecco/HarperCollins.

Arminjon, Mathieu, François Ansermet, e Pierre Magistretti. 2011.
``Emergence du moi cérébral de Théodore Meynert à Antonio Damasio.''
\emph{Psychiatrie Sciences Humaines Neurosciences} 9:153--161.

Armstrong, Thomas. 2010. \emph{Neurodiversity: Discovering the
Extraordinary Gifts of Autism, \versal{ADHD}, Dyslexia, and Other Brain
Differences}. Cambridge, Mass.: Da Capo.

Arpaly, Nomy. 2005. ``How It Is Not `Just Like Diabetes': Mental
Disorder sand the Moral Psychologist.'' \emph{Philosophical Issues}
15:282--298.

Ash, Imogen. 2012. ``How Is the Selective Nature of Memory Explored by
Ian McEwan and in Biology?'' \emph{PsyArt. An Online Journal or the
Psychological Study of the Arts}.
\textless{}\emph{http://www.psyartjournal.com/article/show/ash-howis\_theselectivenatureofmemoryex}\textgreater{}.

Bagatell, Nancy. 2007. ``Orchestrating Voices: Autism, Identity, and the
Power of Discourse.'' \emph{Disability and Society} 22 (4): 413--426.

\_\_\_\_\_.2010. ``From Cure to Community: Transforming Notions of
Autism.'' \emph{Ethos} 38:33--55.

Baker, Dana L. 2011. \emph{The Politics of Neurodiversity: Why Public
Policy Matters?} Boulder, Colo.: Lynne Rienner.

Balt, Steve. 2014. ``Assessing and Enhancing the Effectiveness of
Antidepressants.'' \emph{Psychiatric Times}.
\textless{}\emph{https://bit.ly/327HJ92}\textgreater{}.

Balter, Michael. 2014. ``Talking Back to Madness.'' \emph{Science} 343
(6176): 1190--1193.

Bareither, Isabelle, Felix Hasler, e Anna Strasser. 2015. ``9 Ideen fur
eine bessere Neurowissenschaft.'' \emph{Spektrum} (9 Jan.).
\textless{}\emph{https://bit.ly/1xsWKzb}\textgreater{}.

Barilan, Yechiel M. 2002. ``Head"-Counting vs. Heart"-Counting: An
Examination of the Recent Case of the Conjoined Twins from Malta.''
\emph{Perspectives in Biology and Medicine} 45 (4): 593--603.

\_\_\_\_\_.2003. ``One or Two: An Examination of the Recent Case of the
Conjoined Twins from Malta.'' \emph{Journal of Medicine and Philosophy}
28 (1): 27--44.

Baron"-Cohen, Simon. 2002. ``The Extreme Male Brain Theory of Autism.''
\emph{\versal{TRENDS} in Cognitive Sciences} 6 (6): 248--254.

Bartlett, Frederick. 1932. \emph{Remembering}. Cambridge: Cambridge
University Press.

Bartlett, Tom. 2015. ``Can the Human Brain Project Be Saved? Should It
Be?'' \emph{Chronicle of Higher Education} 61 (23): A6.

Bass, Alison. 2010. ``Emory Neurologist Has History of Failing to
Disclose Conflicts of Interest.'' 15 Novembro.
\textless{}\emph{https://bit.ly/2M0tNrx}\textgreater{}.

Battaglia, Fortunato, Sarah H. Lisanby, e David Freedberg. 2011.
``Corticomotor Excitability During Observation and Imagination of a Work
of Art.'' \emph{Frontiers in Human Neuroscience}.
doi:10.3389/fnhum.2011.00079.

Baxendale, Sallie. 2004. ``Memories Aren't Made of This: Amnesia in the
Movies.'' \emph{British Medical Journal} 329:1480.

Beaulieu, Anne. 2012. ``Fast"-Moving Objects and Their Consequences. A
Response to the \versal{NT} in Practice.'' In Littlefield e Johnson 2012.

Becker, Nicole. 2006. \emph{Die Neurowissenschaftliche Herausforderung
der Pädagogik}. Bad Heilbrun: Klinkhardt.

Beckham, Eduard E., e William R. Leber, Orgs. 1985. \emph{Handbook of
Depression: Treatment, Assessment, and Research}. Homewood, Ill.:
Dorsey.

\_\_\_\_\_.1995. \emph{Handbook of Depression: Treatment, Assessment, and
Research}. 2ed. New York: Guilford.

Beecher, Henry K., et al. 1968. ``A Definition of Irreversible Coma.
Report of the Ad Hoc Committee of the Harvard Medical School to Examine
the Definition of Brain Death.'' \emph{\versal{JAMA}} 205 (6): 337--340.

Begley, Sharon. 2010. ``West Brain, East Brain. What a Difference
Culture Makes.''
\textless{}\emph{http://web.archive.org/web/20120311141118/http://www.thedailybeast.com/newsweek/2010/02/17/ west-brain-ast-brain.html}\textgreater{}.

Béhague, Dominique. 2009. ``Psychiatry and the Politicization of Youth
in Pelotas, Brazil: The Equivocal Uses of `Conduct Disorder' and Related
Diagnoses.'' \emph{Medical Anthropology Quarterly} 23 (4): 455--482.

Beliaev, Alexandre. 1980 {[}1925{]} \emph{Professor Dowell's Head}. New
York: Macmillan. (trad. port.: \emph{A Cabeça do Doutor Dowell}.
Tradução de Robert Tsipen. Moscou: Editora Mir, 1989.)

Bell, Matthew. 2014. \emph{Melancholia: The Western Malady}. New York:
Cambridge University Press.

Belluck, Pam. 2006. ``As Minds Age, What's Next? Brain Calisthenics.''
\emph{New York Times} (27 Dez.).
\textless{}\emph{https://nyti.ms/2OznzAw}\textgreater{}.

Bennett, Laura, Kathryn Thirlaway, e Alexandra J. Murray. 2008. ``The
Stigmatising Implications of Presenting Schizophreniaas a Genetic
Disease.'' \emph{Journal of Genetic Counseling} 17 (6): 550--559.

Bennett, M. R., e P. M. S. Hacker. 2003. \emph{Philosophical Foundations
of Neuroscience}. Malden, Mass.: Blackwell. (trad. port.:
\emph{Fundamentos Filosóficos da Neurociência.} Lisboa: Instituto
Piaget, 2006.
\textless{}\emph{https://bit.ly/2M30xjO}\textgreater{}.)

Bentall, Richard P. 2009. \emph{Doctoring the Mind: Why Psychiatric
Treatments Fail}. New York: New York University Press.

Bernal, John Desmond. 1969 {[}1929{]}. \emph{The World, the Flesh, and
the Devil: An Enquiry Into the Future of the Three Enemies of the
Rational Soul}. Bloomington: Indiana University Press.

Bernat, James L. 2005. ``The Concept and Practice of Brain Death.''
\emph{Progress in Brain Research} 150:369--379.

\_\_\_\_\_.2009. ``Contemporary Controversies in the Definition of
Death.'' \emph{Progress in Brain Research} 77:21--31.

\_\_\_\_\_.2013. ``Controversies in Defining and Determining Death in
Critical Care.'' \emph{Nature Reviews Neurology} 9:164--173.

Bernucci, Leopoldo. 2008. ``Cientificismo e aporias em Os Sertões''. In
\emph{Discurso, Ciência e controvérsia em Euclides da Cunha}, Org.
Leopoldo Bernucci. São Paulo: Edusp.

Bertilsdotter Rosqvist, Hanna, Charlotte Brownlow, e Lindsay
O'Dell.2013. ``Mapping the Social Geographies of Autism---On- and
Offline Narratives of Neuro"-shared and Neuro"-separate Spaces.''
\emph{Disability and Society} 28 (3): 367--379.

Besser, Stephan. 2013. ``From the Neuron to the World and Back: The
Poetics of the Neuromolecular Gaze in Bart Koubaa's \emph{Het gebied van
Nevski} and James Cameron's \emph{Avatar}.'' \emph{Journal of Dutch
Literature} 4 (2): 43--67.

\_\_\_\_\_. 2015. ``Mixing Repertoires: Cerebral Subjects in Contemporary
Dutch Neurological Fiction.'' In \emph{Illness and Literature in the Low
Countries from the Middle Ages Until the Twenty"-First Century}, Org.
Jaap Grave, Rick Honings, e Bettina Noak, 253--272. Gottingen: V\&R
Unipress.

Bhar, Sunil, e Aaron Beck. 2009. ``Treatment Integrity of Studies That
Compare Short"-Term Psychodynamic Psychotherapy with Cognitive"-Behaviour
Therapy.'' \emph{Clinical Psychology: Science and Practice} 16:370--378.

Bickle, John. 2013. ``Multiple Realizability.'' In \emph{The Stanford
Encyclopedia of Philosophy}.
\textless{}\emph{https://stanford.io/2osACcv}\textgreater{}.

Biehl, João. 2005. \emph{Vita: Life in a Zone of Social Abandonment}.
Berkeley: University of California Press.

\_\_\_\_\_. 2006. ``Pharmaceutical Governance.'' In Petryna, Lakoff, e
Kleinman 2006.

Biever, Celeste. 2007. ``Let's Meet Tomorrow in Second Life.'' \emph{New
Scientist} 2610:26--27.

Bioy Casares, Adolfo. 2004 {[}1973{]}. \emph{Asleep in the Sun}.
Tradução de Suzanne Jill Levine. New York: \versal{NYRB}.

Blais, Mark A., et al. 2013. ``Treatment as Usual (\versal{TAU}) for Depression:
A Comparison of Psychotherapy, Pharmacotherapy, and Combined Treatment
at a Large Academic Medical Center.'' \emph{Psychotherapy} 50 (1):
110--118.

Blakemore, Sarah"-Jayne. 2008. ``The Social Brain in Adolescence.''
\emph{Nature Reviews Neuroscience} 9:267--277.

Blakeslee, Sandra. 2007. ``A Small Part of the Brain, and Its Profound
Effects.'' \emph{New York Times} (6 Fevereiro).
\textless{}\emph{https://nyti.ms/2ICxr9b}\textgreater{}.

Blank, Robert H. 1999. \emph{Brain Policy: How the New Neuroscience Will
Change Our Lives and Our Politics}. Washington, D.C.: Georgetown
University Press.

\_\_\_\_\_. 2013. \emph{Intervention in the Brain: Politics, Policy, and
Ethics}. Cambridge, Mass.: \versal{MIT} Press.

Block, Pamela, e Fatima Cavalcante. 2014. ``Historical Perceptions of
Autism in Brazil: Professional Treatment, Family Advocacy, and Autistic
Pride, 1943--2010.'' In \emph{Disability Histories}, Org. Susan Burch e
Michael Rembis. Champaign: University of Illinois Press.

Blume, Harvey. 1997a. ``Autism and the Internet, or It's the Wiring,
Stupid.'' \textless{}\emph{http://web.mit.edu/ m-i-t/articles/indexblume.html}\textgreater{}.

\_\_\_\_\_. 1997b. ``Autistics, freed from face"-to"-face encounters, are
communicating in cyberspace.'' \emph{New York Times} (30 Julho).
\textless{}\emph{http://www.nytimes.com/1997/06/30/business/autistics-freed-from-face-to-face-encounters-are-communicating-in-cyberspace.html}\textgreater{}.

Boekel, Wouter, Eric"-Jan Wagenmakers, Luam Belay, Josine Verhagen, Scott
Brown, e Birte U. Forstmann. 2015. ``A Purely Confirmatory Replication
Study of Structural Brain"-Behavior Correlations.'' \emph{Cortex} 66:
115--133.

Boekel, Wouter, Birte U. Forstmann, e Eric"-Jan Wagenmakers. 2016.
``Challenges in Replicating Brain"-Behavior Correlations: Rejoinder to
Kanai (2015) and Muhlert and Ridgway (2015).'' \emph{Cortex}
74:348--352.

Bogen, Joseph E. 1969. ``The Other Side of the Brain: An Appositional
Mind.'' \emph{Bulletin of the Los Angeles Neurological Society}
34:135--162.

\_\_\_\_\_. 1971. ``Neowiganism.'' In \emph{Drugs, Development, and
Cerebral Function}, Org. W. Lynn Smith. Springfield, Ill.: C. C. Thomas.

\_\_\_\_\_. 1985. ``Foreword.'' In \emph{A New View of Insanity: The
Duality of the Mind Proved by the Structure, Functions, and Diseases of
the Brain}, Org. Arthur L. Wigan {[}1844{]}. Malibu, Calif.: Joseph
Simon.

Boileau"-Narcejac, Pierre L. 1965. \emph{. . . Et mon tout est un homme}.
Paris: Denoel.

Bonnet, Charles. 1760. \emph{Essai analytique sur les facultés de
l'âme}. In \emph{OEuvres d'histoire naturelle et de} philosophie.
Neuchâtel: Samuel Fauche, 1779--1783, tomo 6 (= vol. 8).

Borck, Cornelius. 2005. \emph{Hirnströme. Eine Kulturgeschichte der
Elektroenzephalographie}. Göttingen: Wallstein.

Borg, Emma. 2007. ``If Mirror Neurons Are the Answer, What Was the
Question?'' \emph{Journal of Consciousness Studies} 14 (8): 5--19.

Borgelt, Emily L., Daniel Z. Buchman, e Judy Illes. 2012. ``Neuroimaging
in Mental Health Care: Voices in Translation.'' \emph{Frontiers in Human
Neuroscience} 6, art. 293: 1--5.

Boshears, Rhonda, e Harry Whitaker. 2013. ``Phrenology and Physiognomy
in Victorian Literature.'' In \emph{Literature, Neurology, and
Neuroscience: Historical and Literary Connections}, Org. Anne Stiles,
Stanley Finger, e François Boller, 87--112. Amsterdam: Elsevier.

Bottoni, Patrizia. 2012. ``Il romanzo gotico di Francesco Mastriani.''
PhD diss., University of Toronto.

Bould, Mark, e Sherryl Vint. 2007. ``Of Neural Nets and Brains in Vats:
Model Subjects in \emph{Galatea 2.2} and \emph{Plus}.'' \emph{Biography}
30 (1): 84--105.

Bound Alberti, Fay. 2010. \emph{Matters of the Heart: History, Medicine,
and Emotion}. New York: Oxford University Press.

Boundy, Kathryn. 2008. ```Are You Sure, Sweetheart, That You Want to Be
Well?': An Exploration of the Neurodiversity Movement.'' \emph{Radical
Psychology} 7: 1--20.

Bowers, Jeffrey S. 2016. ``The Practical and Principled Problems with
Educational Neuroscience.'' \emph{Psychological Review}.
\textless{}\emph{https://bit.ly/2MqHT4C}\textgreater{}.

Bowman, James. 2004. ``Memory and the Movies.'' \emph{New Atlantis: A
Journal of Technology and Society} 5:85--90.

Boyce, Alison C. 2009. ``Neuroimaging in Psychiatry: Evaluating the
Ethical Consequences for Patient Care.'' \emph{Bioethics} 23:349--359.

Boyd, Robynne. 2008. ``Do People Use Only 10 Percent of Their Brains?''
\emph{Scientific American} (7 Fevereiro).
\textless{}\emph{https://bit.ly/1mwKKr5}\textgreater{}.

Braden, Charles S. 1963. \emph{Spirits in Rebellion: The Rise and
Development of New Thought.} Dallas: Southern Methodist University
Press.

Broca, Paul. 1861. ``Remarques sur le siège de la faculté du langage
articulé, suivies d'une observation d'aphémie (perte de la parole).''
\emph{Bulletin de la Société Anatomique} 6:330--357.

Broer, Christian, e Marjolijn Heerings. 2013. ``Neurobiology in Public
and Private Discourse: The Case of Adults with \versal{ADHD}.'' \emph{Sociology
of Health and Illness} 35 (1): 49--65.

Broks, Paul. 2003. \emph{Into the Silent Land: Travels in
Neuropsychology}. New York: Atlantic Monthly Press.

Bronte, Charlotte. 1985 {[}1849{]}. \emph{Shirley}. London: Penguin.
(trad. port.: \emph{Shirley}. Espírito Santo: Pedrazul Editora, 2014.)

Brooks, David. 2013. ``Beyond the Brain.'' \emph{New York Times} (17
Junho).
\textless{}\emph{http://www.nytimes.com/2013/06/18/opinion/brooks-beyond-thebrain.html?r=0}\textgreater{}.

Brosnan, Caragh, e Mike Michael. 2014. ``Enacting the `Neuro' in
Practice: Translational Research, Adhesion, and the Promise of
Porosity.'' \emph{Social Studies of Science} 44 (5): 680--700.

Brown"-Séquard, Charles"-Édouard. 1874a. ``The Brain Power of Man: Has He
Two Brains or Has He One?'' \emph{Cincinnati Lancet and Observer}
17:330--333.

\_\_\_\_\_. 1874b. ``Dual Character of the Brain.'' \emph{Smithsonian
Miscellaneous Collections} 15:1--21.

\_\_\_\_\_. 1890. ``Have We Two Brains or One.'' \emph{The Forum} 9:
627--643.

Brown, Roger, e James Kulik. 1977. ``Flashbulb Memories.''
\emph{Cognition} 5: 73--99.

Brownlee, Christen. 2006a. ``Eat Smart. Foods May Affect the Brain as
Well as the Body.'' \emph{Science News} 169 (9): 136--137.

\_\_\_\_\_. 2006b. ``Buff and Brainy. Exercising the Body Can Benefit the
Mind.'' \emph{Science News} 169 (8): 122--124.

Brownlow, Charlotte. 2007. ``The Construction of the Autistic
Individual: Investigations in Online Discussion Groups.'' PhD diss.,
University of Brighton.

Brownlow, Charlotte, e Lindsay O'Dell. 2006. ``Constructing an Autistic
Identity: \versal{AS} Voices Online.'' \emph{Mental Retardation} 44 (5):
315--321.

Broyd, Samantha J., Charmaine Demanuele, Stefan Debener, Suzannah K.
Helps, Christopher J. James, e Edmund J. S. Sonuga"-Barke. 2009.
``Default"-Mode Brain Dysfunction in Mental Disorders: A Systematic
Review.'' \emph{Neuroscience and Biobehavioral Reviews} 33:279--296.

Bryson, Norman. 2003. ``Introduction: The Neural Interface.'' In
\emph{Blow"-Up: Photography, Cinema, and the Brain}, Org. Warren Neidich.
New York: Distributed Art Publishers.

Bucchi, Masimiano, e Federico Neresini. 2007. ``Science and Public
Participation.'' In \emph{The Handbook of Science and Technology
Studies}, 3 ed., Org. Edward J. Hackett et al. Cambridge, Mass.: \versal{MIT}
Press.

Buchman Daniel Z., et al. 2013. ``Neurobiological Narratives:
Experiences of Mood Disorder Through the Lens of Neuroimaging.''
\emph{Sociology of Health and Illness} 35 (1): 66--81.

Buckner, Randy L., Jessica R. Andrews"-Hanna, e Daniel L. Schachter.2008.
``The Brain's Default Network: Anatomy, Function, and Relevance to
Disease.'' \emph{Annals of the New York Academy of Sciences} 1124:1--38.

Buford, Chris, e Fritz Allhoff. 2005. ``Neuroscience and Metaphysics.''
\emph{American Journal of Bioethics} 5 (2): 34--36, W33--34.

Bulgakov, Mikhail. 1987 {[}1925{]}. \emph{Heart of a Dog}. Tradução de
Mirra Ginsberg. New York: Grove. (trad. port.: \emph{Coração de Cão}.
Lisboa: Alêtheia Editores, 2014.)

Bulow, Hans"-Henrik, et al. 2008. ``The World's Major Religions' Points
of View on End"-of"-Life Decisions in the Intensive Care Unit.''
\emph{Intensive Care Medicine} 34:423--430.

Bumiller, Kristin. 2008. ``Quirky Citizens: Autism, Gender, and
Reimagining Disability.'' \emph{Signs: Journal of Women in Culture and
Society} 33 (4): 967--991.

Burkeman, Oliver. 2015. ``Therapy Wars: The Revenge of Freud.''
\emph{Guardian} (7 Janeiro).
\textless{}\emph{https://bit.ly/1O6gAqM}\textgreater{}.

Burn, Stephen J., e Peter Dempsey, Orgs. 2008. \emph{Intersections:
Essays on Richard Powers}. Champaign, Ill.: Dalkey Archive.

Burton, Robert. 1651 {[}1621{]}. \emph{The Anatomy of Melancholy}.
\textless{}\emph{https://bit.ly/2Oy4kaL}\textgreater{}. (trad.
port.: \emph{Anatomia da Melancolia}. Porto: Porto Editora, 2014.)

Busso, Daniel S., e Courtney Pollack. 2015. ``No Brain Left Behind:
Consequences of Neuroscience Discourse for Education.'' \emph{Learning,
Media, and Technology} 40 (2): 168--186.

Button, Katherine S., et al. 2013. ``Power Failure: Why Small Sample
Size Undermines the Reliability of Neuroscience.'' \emph{Nature Reviews
Neuroscience}14 (5): 365--376.

Byatt, Antonia S. 2006a. ``Observe the Neurones. Between, Above and
Below John Donne.'' \emph{Times Literary Supplement} (22 Setembro).
\textless{}\emph{http://www.thetimes.co.uk/tto/others/article1888544.ece}\textgreater{}.

\_\_\_\_\_. 2006b. ``Feeling Thought: Donne and the Embodied Mind.'' In
\emph{The Cambridge Companion to Donne}, Org. Achsah Gibbory, 247--257.
New York: Cambridge University Press.

Cabral, Joana, et al. 2013. ``Structural Connectivity in Schizophrenia
and Its Impact on the Dynamics of Spontaneous Functional Networks.''
\emph{Chaos: An Interdisciplinary Journal of Nonlinear Science} 23 (4):
046111. doi:10.1063/1.4851117.

Cacioppo, John T., e Gary G. Berntson. 1992. ``Social Psychological
Contributions to the Decade of the Brain: Doctrine of Multilevel
Analysis.'' \emph{American Psychologist} 47:1019--1028.

\_\_\_\_\_, Orgs. 2005. \emph{Social Neuroscience.} New York: Psychology
Press.

Cadigan, Pat, Org. 2002. \emph{The Ultimate Cyberpunk}. New York:
ibooks.

Callard, Felicity, e Daniel S. Margulies. 2011. ``The Subject at Rest:
Novel Conceptualizations of Self and Brain from Cognitive Neuroscience's
Study of the `Resting State.''' \emph{Subjectivity} 4:227--257.

Callard, Felicity, e Des Fitzgerald. 2015. \emph{Rethinking
Interdisciplinarity Across the Social Sciences and Neurosciences}. New
York: Palgrave Macmillan.

Capacchione, Lucia. 2001. \emph{The Power of Your Other Hand: A Course
in Channeling the Inner Wisdom of the Right Brain.} Franklin Lakes,
N.J.: New Page.

Cappelletto, Chiara. 2009. \emph{Neuroestetica. L'arte del cervello}.
Rome: Laterza.

Caramazza, Alfons, Stefano Anzellotti, Lukas Strnad, e Angelika Lingnau.
2014. ``Embodied Cognition and Mirror Neurons: A Critical Assessment.''
\emph{Annual Review of Neuroscience} 37:1--15.

Carey, Benedict. 2005. ``Can Brain Scans See Depression?'' \emph{New
York Times} (18 Outubro).
\textless{}\emph{https://nyti.ms/2IDW3Oy}\textgreater{}.

Carey, Nessa. 2012. \emph{The Epigenetics Revolution: How Modern Biology
Is Rewriting Our Understanding of Genetics, Disease, and Inheritance}.
New York: Columbia University Press.

Carson, Gerald. 1957. \emph{Cornflakes Crusade}. New York: Reinhart.

Carver, Joanna. 2012. ``New Look at Einstein's Brain Pictures Show His
Genius.'' \emph{New Scientist} (20 Novembro).
\textless{}\emph{http://www.newscientist.com/blogs/shortsharpscience/2012/11/einsteins-brain.html}\textgreater{}.

Casati, Roberto, e Alessandro Pignocchi. 2007. ``Mirror and Canonical
Neurons Are Not Constitutive of Aesthetic Response {[}Comment on
Freedberg and Gallese, 2007{]}.'' \emph{Trends in Cognitive Science} 11
(10): 410.

Cascio, Ariel. 2014. ``New Directions in the Social Study of the Autism
Spectrum: A Review Essay.'' \emph{Culture, Medicine, and Psychiatry}
38:306--311.

Cavallaro, Dani. 2004. ``The Brain in a Vat in Cyberpunk, the
Persistence of the Flesh.'' In Gere 2004, 287-- 305.

Cavanagh, Patrick. 2005. ``The Artist as Neuroscientist.'' \emph{Nature}
434:301--307.

\versal{CBS}. 2006. ``Retraining the Brain: Doctors Test Drug"-Free Methods to
Restore Lost Mental Capabilities.'' \emph{\versal{CBS} News} (15 Janeiro).
\textless{}\emph{https://cbsn.ws/3229aAY}\textgreater{}.

Cela"-Conde, Camilo J., et al. 2011. ``The Neural Foundations of
Aesthetic Appreciation.'' \emph{Progress in Neurobiology} 94:39--48.

Chafetz, Michael D. 1992. \emph{Smart for Life: How to Improve Your
Brain Power at Any Age.} New York: Penguin.

Chamak, Brigitte. 2008. ``Autism and Social Movements: French Parents'
Associations and International Autistic Individuals' Organizations.''
\emph{Sociology of Health and Illness} 30 (1): 76--96.

Chamak, Brigitte. 2014. ``Autism as Viewed by French Parents.'' In
\emph{Comprehensive Guide to Autism}, Org., Vinood B. Patel, Victor R.
Preedy, e Colin R. Martin. New York: Springer.

Chamak, Brigitte, e Beatrice Bonniau. 2013. ``Changes in the Diagnosis
of Autism: How Parents and Professionals Act and React in France.''
\emph{Culture, Medicine and Psychiatry} 37:405--426.

Chamak, Brigitte, et al. 2008. ``What Can We Learn About Autism from
Autistic Persons?'' \emph{Psychotherapy and Psychosomatics} 77 (5):
271--279.

Charlton, James. 2000. \emph{Nothing About Us Without Us: Disability
Oppression and Empowerment}. Berkeley: University of California Press.

Charman, Tony. 2006. ``Autism at the Crossroads: Determining the
Phenotype Matters for Neuroscience.'' \emph{Nature Neuroscience} 9 (10):
1197.

Charney, Dennis, et al. 2002. ``Neuroscience Research Agenda to Guide
Development of a Pathophysiologically Based Classification System.'' In
\emph{A Research Agenda for \versal{DSM"-V}}, Org. David J. Kupfer, Michael B.
First, e Darrel A. Regier, 31--84. Washington, D.C.: American
Psychiatric Association.

Chatterjee, Anjan. 2010. ``Neuroaesthetics: A Coming of Age Story.''
\emph{Journal of Cognitive Neuroscience} 23:53--62.

\_\_\_\_\_. 2012. ``Neuroaesthetics: Growing Pains of a New Discipline.''
In \emph{Aesthetic Science: Connecting Minds, Brains, and Experience},
Org. Arthur P. Shimamura e Stephen E. Palmer. New York: Oxford
University Press.

Chatterjee, Anjan, e Martha J. Farah, Orgs. 2013. \emph{Neuroethics in
Practice: Medicine, Mind, and Society}. New York: Oxford University
Press.

Cheek, Joanna. 2012. ``Myth: Reframing Mental Illness as a `Brain
Disease' Reduces Stigma.'' \emph{Canadian Foundation for Healthcare
Improvement}.
\textless{}\emph{http://www.cfhi-fcass.ca/SearchResultsNews/12-06-04/a078ceca-a41-4d14-82b5-b60f5a8bb991.aspx}\textgreater{}.

Cheon, Bobby K., et al. 2011. ``Cultural Influences on Neural Basis of
Intergroup Empathy.'' \emph{Neuroimage} 57 (2): 642--650.

\_\_\_\_\_. 2013. ``Constraints, Catalysts, and Coevolution in Cultural
Neuroscience: Reply to Commentaries.'' \emph{Psychological Inquiry} 24
(1): 71--79.

Cheu, Johnson. 2004. ``De"-gene"-erates, Replicants, and Other Aliens:
(Re)defining Disability in Futuristic Film.'' In Corker e French 1999,
198--212.

Chiao, Joan Y., Org. 2009a. \emph{Cultural Neuroscience: Cultural
Influences on Brain Function}. Progress in Brain Research 178. New York:
Elsevier.

\_\_\_\_\_.2009b. ``Cultural Neuroscience: A Once and Future
Discipline.'' In Chiao 2009a.

\_\_\_\_\_.2011. ``Cultural Neuroscience: Visualizing Culture"-Gene
Influences on Brain Function.'' In \emph{The Oxford Handbook of Social
Neuroscience}, Org. Jean Decety e John T. Cacioppo. Oxford: Oxford
University Press.

Chiao, Joan Y., e Nalini Ambady. 2007. ``Cultural Neuroscience: Parsing
Universality and Diversity Across Levels of Analysis.'' In
\emph{Handbook of Cultural Psychology}, Org. Shinobu Kitayama e Dov
Cohen. New York: Guilford.

Chiao, Joan Y., e Katherine D. Blizinsky. 2010. ``Culture"-Gene
Coevolution of Individualism"-Collectivism and the Serotonin Transporter
Gene.'' \emph{Proceedings of the Royal Society B} 277:529--537.

Chiao, Joan Y., e Bobby K. Cheon. 2012. ``Cultural Neuroscience as
Critical Neuroscience in Practice.'' In Choudhury e Slaby 2012a.

Chiao, Joan Y., et al. 2008. ``Cultural Specificity in Amygdale Response
to Fear Faces.'' \emph{Journal of Cognitive Neuroscience} 20 (12):
2167--2174.

\_\_\_\_\_. 2009. ``Neural Basis of Individualistic and Collectivistic
Views of Self.'' \emph{Human Brain Mapping} 30 (9): 2813--2820.

\_\_\_\_\_. 2010. ``Theory and Methods in Cultural Neuroscience.''
\emph{Social Cognitive and Affective Neuroscience} 5 (2/3): 356--361.

\_\_\_\_\_. 2013. ``Cultural Neuroscience: Progress and Promise.''
\emph{Psychological Inquiry} 24 (1): 1--19.

Choudhury, Suparna. 2010. ``Culturing the Adolescent Brain: What Can
Neuroscience Learn from Anthropology?'' \emph{Social Cognitive and
Affective Neuroscience} 5 (2/3): 159--167.

Choudhury, Suparna, e Laurence J. Kirmayer. 2009. ``Cultural
Neuroscience and Psychopathology: Prospects for Cultural Psychiatry.''
In Chiao 2009a.

Choudhury, Suparna, e Kelly A. McKinney. 2013. ``Digital Media, the
Developing Brain, and the Interpretive Plasticity of Neuroplasticity.''
\emph{Transcultural Psychiatry} 50 (2): 192--215.

Choudhury, Suparna, e Jan Slaby, Orgs. 2012a. \emph{Critical
Neuroscience: A Handbook of the Social and Cultural Contexts of
Neuroscience}. Malden, Mass.: Blackwell.

\_\_\_\_\_. 2012b. ``Introduction. Critical Neuroscience --- Between
Lifeworld and Laboratory.'' In Choudhury e Slaby 2012a.

Choudhury, Suparna, Kelly A. McKinney, e Moritz Merten. 2012.
``Rebelling Against the Brain: Public Engagement with the `Neurological
Adolescent.''' \emph{Social Science and Medicine} 74:565--573.

Choudhury, Suparna, Saskia Kathi Nagel, e Jan Slaby. 2009. ``Critical
Neuroscience: Linking Science and Society Through Critical Practice.''
\emph{BioSocieties} 4 (1): 61--77.

Churchland, Paul M. 1981. ``Eliminative Materialism and the
Propositional Attitudes.'' \emph{Journal of Philosophy} 78 (2): 67--90.

Cikara, Mina, e Jay J. Van Bavel. 2014. ``The Neuroscience of Intergroup
Relations: An Integrative Review.'' \emph{Perspectives on Psychological
Science} 9 (3):245--274.

Clarke, Basil. 1987. \emph{Arthur Wigan and the Duality of Mind}.
Cambridge: Cambridge University Press.

Clarke, Edwin, e Kenneth Dewhurst. 1996. \emph{An Illustrated History of
Brain Function}. Berkeley: University of California Press.

Clarke, Edwin, e L. Stephen Jacyna. 1987. \emph{Nineteenth"-Century
Origins of Neuroscientific Concepts}. Berkeley: University of California
Press.

Clarke, Juanne, e Gudrun van Amerom. 2007. ```Surplus Suffering':
Differences Between Organizational Understandings of Asperger's Syndrome
and Those People Who Claim the `Disorder.' '' \emph{Disability and
Society} 22 (7): 761--776.

\_\_\_\_\_. 2008. ``Asperger's Syndrome: Differences Between Parents'
Understanding and Those Diagnosed.'' \emph{Social Work in Health Care}
46 (3): 85--106.

Clausen, Jens, e Neil Levy, Orgs. 2015. \emph{Handbook of Neuroethics}.
Dordrecht: Springer.

Clifford, Jim. 1988. \emph{The Predicament of Culture: Twentieth"-Century
Ethnography, Literature, and Art} Cambridge, Mass.: Harvard University
Press.

Cohen, Adam. B. 2009. ``Many Forms of Culture.'' \emph{American
Psychologist} 6:194--204.

Cohen, Alex, e Oye Gureje. 2007. ``Making Sense of Evidence.''
\emph{International Review of Psychiatry} 19 (5): 583--591.

Cohen, Alex, Vikram Patel, e Harry Minas. 2014. ``A Brief History of
Global Mental Health.'' In Patel, Minas, Cohen, e Prince 2014.

Cohen, Isabel, e Marcelle Goldsmith. 2002. \emph{Hands On: How to Use
Brain Gym in the Classroom}. Ventura, Calif.: Edu Kinesthetics, Inc.

Cohen, Patricia. 2010. ``Next Big Thing in English, Knowing They Know
That You Know.'' \emph{New York Times} (1 Abril).
\textless{}\emph{https://nyti.ms/312AhKI}\textgreater{}.

Cohn, Simon. 2010. ``Picturing the Brain Inside, Revealing the Illness
Outside: A Comparison of the Different Meanings Attributed to Brain
Scans by Scientists and Patients.'' In \emph{Technologized Images,
Technologized Bodies}, Org. J. Edwards, P. Harvey, e P. Wade, 65--84.
New York: Berghahn.

\_\_\_\_\_. 2012. ``Disrupting Images: Neuroscientific Representations in
the Lives of Psychiatric Patients.'' In Choudhury e Slaby 2012a.

Coleman, Gabriella E. 2010. ``Ethnographic Approaches to Digital
Media.'' \emph{Annual Review of Anthropology} 39:487--505.

Coles, Romand. 2013. ``The Neuropolitical \emph{Habitus} of Resonant
Receptive Democracy.'' In Vander Valk 2012a, 178--197.

Coltheart, Max. 2006. ``Perhaps Functional Neuroimaging Has Not Told Us
Anything About the Mind (So Far).'' \emph{Cortex} 42:422--427.

\_\_\_\_\_.2013. ``How Can Functional Neuroimaging Inform Cognitive
Theories?'' \emph{Perspectives on Psychological Scien}ce 8 (1): 98--103.

Combe, Andrew. 1836--1837. ``Remarks on the Possibility of Increasing
the Development of the Cerebral Organs by Adequate Exercise of the
Mental Faculties.'' \emph{Phrenological Journal} 10:414--426.

Combe, George. 1828. \emph{The Constitution of Man Considered in
Relation to External Objects}. Edinburgh: Mclachlan \& Stewardt e John
Anderson.

Connolly, William. 2002. \emph{Neuropolitics: Thinking, Culture, Speed}.
Minneapolis: University of Minnesota Press.

Conrad, Erin C., e Raymond De Vries. 2011. ``Field of Dreams: A Social
History of Neuroethics.'' In Pickersgill e Van Keulen 2011.

Constable, Catherine. 2012. ``Withdrawal of Artificial Nutrition and
Hydration for Patients in a Permanent Vegetative State: Changing Tack.''
\emph{Bioethics} 26 (3): 157--163.

Conway, Bevil R., e Alexander Rehding. 2013. ``Neuroaesthetics and the
Trouble with Beauty.'' \emph{\versal{PLOS} Biology} 11 (3): e1001504.
doi:10.1371/journal.pbio.1001504.

Cook, Richard, Geoffrey Bird, Caroline Catmur, Clare Pressa, e Cecilia
Heyes. 2014. ``Mirror Neurons: From Origin to Function.'' \emph{Brain
and Behavioral Sciences} 37 (2): 177--192.

Cooper, Rachel. 2015. ``Must Disorders Cause Harm? The Changing Stance
of the \versal{DSM}.'' In \emph{The \versal{DSM}"-5 in Perspective: Philosophical
Reflections on the Psychiatric Babel}, Org. Steeves Demazeux e
PatrickSingy, 83--96. New York: Springer.

Cooper, Sara. 2016. ``Global Mental Health and Its Critics: Moving
Beyond the Impasse.'' \emph{Critical Public Health} 26 (4): 355--358.

Cooter, Roger. 1984. \emph{The Cultural Meaning of Popular Science:
Phrenology and the Organisation of Consent in Nineteenth"-Century
Britain}. Cambridge: Cambridge University Press.

Cooter, Roger. 2014. ``Neural Veils and the Will to Historical Critique:
Why Historians of Science Need to Take the Neuro"-Turn Seriously.''
\emph{Isis} 105 (1):145--154.

Corker, Mairian. 1999. ``New Disability Discourse, the Principle of
Optimization, and Social Change.'' In Corker e French 1999.

Corker, Mairian, e Sally French, Orgs. 1999. \emph{Disability
Discourse}. Philadelphia: Open University Press.

Corker, Mairian, e Tom Shakespeare, Orgs. 2004.
\emph{Disability/Postmodernity: Embodying Disability Theory}. London:
Continuum.

Corrigan, W. Patrick, et al. 2002. ``Challenging Two Mental Illness
Stigmas: Personal Responsibility and Dangerousness.''
\emph{Schizophrenia Bulletin} 28 (2):293--309.

Cotman, W. Carl, e Nicole C. Berchtold. 2002. ``Exercise: A Behavioral
Intervention to Enhance Brain Health and Plasticity.'' \emph{Trends in
Neurosciences} 25 (6): 295--301.

Couser, G. Thomas. 2004. \emph{Vulnerable Subjects: Ethics and Life
Writing}. Ithaca, N.Y.: Cornell University Press.

Cowen, Phillip J. 2013. ``Classification of Depressive Disorders.'' In
\emph{Behavioral Neurobiology of Depression and Its Treatment}, Org.
Philip J. Cowen, Trevor Sharp, e Jennifer Y. F. Lau, 3--13 Berlin:
Springer.

Crane, Mary Thomas, e Alan Richardson. 1999. ``Literary Studies and
Cognitive Science: Toward a New Interdisciplinarity.'' \emph{Mosaic}
32:123--140.

Crichton. Michael. 1972. \emph{The Terminal Man}. New York: Ballantine.

Crick, Francis. 1994. \emph{The Astonishing Hypothesis: The Scientific
Search for the Soul.} New York: Touchstone. (trad. port.: \emph{A
Hipótese Espantosa: Busca Científica da Alma}. Lisboa: Instituto Piaget,
1998.).

Cromby, John, e Simon J. Williams. 2011. ``Neuroscience and
Subjectivity.'' \emph{Subjectivity} 4:215--226.

Crossley, Nick. 1998. ``R. D. Laing and the British Antipsychiatry
Movement.'' \emph{Social Science and Medicine} 47 (7): 877--889.

\_\_\_\_\_. 2006. \emph{Contesting Psychiatry: Social Movements in Mental
Health.} London: Routledge.

Cunningham, John P., e Byron M. Yu. 2014. ``Dimensionality Reduction for
Large"-Scale Neural Recordings.'' \emph{Nature Neuroscience}
17:1500--1509.

D'Alembert, Jean. 1986 {[}1767{]}. ``Eclaircissements sur différents
endroits des \emph{Eléments de philosophie}''. In \emph{Essai sur les
eléments de philosophie}. Paris: Fayard.

Damousi, Joy, e Mariano Ben Plotkin, Orgs. 2009. \emph{The Transnational
Unconscious}: \emph{Essays in the History of Psychoanalysis and
Transnationalism.} London: Palgrave Macmillan.

Danto, Arthur C. 1964. ``The Artworld.'' \emph{Journal of Philosophy} 61
(19): 571--584.

\_\_\_\_\_. 1981. \emph{The Transfiguration of the Commonplace}.
Cambridge, Mass.: Harvard University Press.

\_\_\_\_\_. 1993. ``Andy Warhol: Brillo Box.'' \emph{Artforum} 32 (1):
128--129.

\_\_\_\_\_. 1997. \emph{After the End of Art: Contemporary Art and the
Pale of History}. Princeton, N.J.: Princeton University Press.

Daston, Lorraine, e Otto Sibum. 2003. ``Introduction: Scientific
Personae and Their Histories.'' \emph{Science in Context} 16 (1/2):
1--8.

Davidson, Joyce. 2007. ```In a World of Her Own . . .': Re"-presenting
Alienation and Emotion in the Lives and Writings of Women with Autism.''
\emph{Gender, Place, and Culture} 14 (6): 659--677.

\_\_\_\_\_. 2008. ``Autistic Culture Online: Virtual Communication and
Cultural Expression on the Spectrum.'' \emph{Social and Cultural
Geography} 9 (7):791--806.

Davidson, Richard J., et al. 2002a. ``Depression: Perspectives from
Affective Neuroscience.'' \emph{Annual Review of Psychology} 53:
545--574.

\_\_\_\_\_. 2003. ``Alterations in Brain and Immune Function Produced by
Mindfulness Meditation.'' \emph{Psychosomatic Medicine} 65:564--570.

Davidson, Richard J., Diego Pizzagalli, e Jack Nitschke. 2002b. ``The
Representation and Regulation of Emotion in Depression: Perspectives
from Affective Neuroscience.'' In \emph{Handbook of Depression}, Org.
Ian Gotlib e Constance Hammen, 219--244. New York: Guilford.

\_\_\_\_\_. 2009. ``Representation and Regulation of Emotion in
Depression: Perspectives from Affective Neuroscience.'' In
\emph{Handbook of Depression}, Org. Ian Gotlib e Constance Hammen,
218--248. New York: Guilford.

Davidson, Richard J., e Bruce McEwen. 2012. ``Social Influences on
Neuroplasticity: Stress and Interventions to Promote Well"-Being.''
\emph{Nature Neuroscience} 15 (5): 689--695.

Davies, David. 2014. ```This Is Your Brain on Art.' What Can Philosophy
of Art Learn from Neuroscience?'' In \emph{Aesthetics and the Sciences
of Mind}, Org. Gregory Currie, Matthew Kieran, Aaron Meskin, e Jon
Robson. New York: Oxford University Press.

Davies, Stephen. 2009. ``Evolution, Art, and Aesthetics.'' In \emph{A
Companion to Aesthetics}, 2ed., Org. Stephen Davies, Kathleen Marie
Higgins, Robert Hopkins, Robert Stecker, e David E. Cooper. Oxford:
Blackwell.

Davis, Lennard J. 1995. \emph{Enforcing Normalcy: Disability, Deafness,
and the Body}. London: Verso.

\_\_\_\_\_. 2002. \emph{Bending Over Backwards: Disability, Dismodernism,
and Other Difficult Positions}. New York: New York University Press.

Dawson, Michelle. 2004. ``The Misbehavior of Behaviorists. Ethical
Challenges to the Autism"-\versal{ABA} Industry''.
\textless{}\emph{https://bit.ly/1re6Hs8}.

De Almeida, Jorge C., e Mary Louise Phillips. 2013. ``Distinguishing
Between Unipolar Depression and Bipolar Depression: Current and Future
Clinical and Neuroimaging Perspectives.'' \emph{Biological Psychiatry}
73:111--118.

De Beaugrande, Robert. 1987. ``Schemas for Literary Communication.'' In
\emph{Literary Discourse: Aspects of Cognitive and Social Psychological
Approaches}, Org. Laszlo Halász, 49--99. Berlin: de Gruyter.

De Giustino, David. 1975. \emph{Conquest of Mind: Phrenology and
Victorian Social Thought}. London: Croom Helm.

De Grazia, David. 2011. ``The Definition of Death.'' \emph{Stanford
Encyclopedia of Philosophy}.
\textless{}\emph{https://stanford.io/2M3bnqd}\textgreater{}.

De Vignemont, Frédérique, e Tania Singer. 2006. ``The Empathic Brain:
How, When, and Why?'' \emph{Trends in Cognitive Sciences} 10 (10):
435--441.

De Vos, Jan, e Ed Pluth, Orgs. 2016. \emph{Neuroscience and Critique:
Exploring the Limits of the Neurological Turn}. New York: Routledge.

Dekker, Martijn. 2006. ``On Our Own Terms: Emerging Autistic Culture.''\\
\textless{}\emph{https://bit.ly/2p1Ea5r}\textgreater{}.

Denkhaus, Ruth, e Mathias Bos. 2012. ``How Cultural is `Cultural
Neuroscience'? Some Comments on an Emerging Research Paradigm.''
\emph{BioSocieties} 7 (4): 433--458.

Dennett, Daniel C. 2008. ``Astride the Two Cultures: A Letter to Richard
Powers, Upadated.'' In Burn e Dempsey 2008, 151--161.

Dennison, Gail E., Paul E. Dennison, e Jerry V. Teplitz. 1994.
\emph{Brain Gym for Business. Instant Brain Boosters for On"-The"-Job
Success}. Ventura, Calif.: Edu"-Kinesthetics, Inc.

Deresiewicz, William. 2006. ``Science Fiction.'' \emph{The Nation} (9
Outubro).
\textless{}\emph{https://bit.ly/33icmss}\textgreater{}.

Deshpande, Gopikrishna, et al. 2013. ``Identification of Neural
Connectivity Signatures of Autism Using Machine Learning.''
\emph{Frontiers in Human Neuroscience} 17 (7), art. 670: 1--15.

Deville, James. 1841. ``Account of a Number of Cases in Which a Change
Had Been Produced on the Form of the Head by Education and Moral
Training.'' \emph{Phrenological Journal} 14:32--38.

Dewhurst, Kenneth, trans. 1980. \emph{Thomas Willis's Oxford Lectures}.
Oxford: Sandford.

Di Dio, Cinzia, Emiliano Macaluso, e Giacomo Rizzolatti. 2007. ``The
Golden Beauty: Brain Response to Classical and Renaissance Sculptures.''
\emph{\versal{PL}o\versal{S ONE}} 11 (Novembro): 1--9.

Diamond, Marian C., et al. 1985. ``On the Brain of a Scientist: Albert
Einstein.'' \emph{Experimental Neurology} 88: 198--204.

Dick, Philip K. 1991 {[}1977{]}. \emph{A Scanner Darkly}. New York:
Vintage. (trad. port.: \emph{O Homem Duplo}. Rio de Janeiro: Rocco,
2007.)

Dinello, Daniel. 2006. \emph{Technophobia! Science Fiction Visions of
Posthuman Technology}. Austin: University of Texas Press.

Dobbs, David. 2006. ``A Depression Switch?'' \emph{New York Times
Magazine} (2 Abril).
\textless{}\emph{https://nyti.ms/2M4uUqp}\textgreater{}.

Doidge, Norman. 2007. \emph{The Brain That Changes Itself: Stories of
Personal Triumph from the Frontiers of Brain Science}. New York:
Penguin. (trad. port.: \emph{O Cérebro que se Transforma: Como a
Neurociência Pode Curar as Pessoas}. Tradução de Ryta Vinagre. 10 ed.
Rio de Janeiro: Record, 2011.)

\_\_\_\_\_. 2015. \emph{The Brain's Way of Healing: Remarkable
Discoveries and Recoveries from the Frontiers of Neuroplasticity}. New
York: Viking.

Dolan, Brian. 2007. ``Soul Searching: A Brief History of the Mind/Body
Debate in the Neurosciences.'' \emph{Neurosurgical Focus} 23:1--7.

Dominguez Duque, Juan F. 2012. ``Neuroanthropology and the Dialectical
Imperative.'' \emph{Anthropological Theory} 12 (1): 5--27.

\_\_\_\_\_. 2015. ``Toward a Neuroanthropology of Ethics: Introduction.''
In \emph{Handbook of Neuroethics}, Org. Jens Clausen e Neil Levy.
Dordrecht: Springer.

Dominguez Duque, Juan F., et al. 2009. ``The brain in Culture and
Culture in the Brain: A Review of Core Issues in Neuroanthropology.'' In
Chiao 2009a.

\_\_\_\_\_. 2010. ``Neuroanthropology: A Humanistic Science for the Study
of the Culture"-Brain Nexus.'' \emph{\versal{SCAN}} {[}\emph{Social Cognitive and
Affective Neuroscience}{]} 5 (2/3): 38--147.

Doucet, Hubert. 2005. ``Imagining a Neuroethics Which Would Go Further
Than Genethics.'' \emph{American Journal of Bioethics} 5 (2): 29--31,
W23--24.

Downey, Greg. 2012a. ``Neuroanthropology.'' In \emph{The \versal{SAGE} Handbook
of Social Anthropology}, v. 2., Org. Richard Fardon, Oliva Harris,
Trevor H. J. Marchand, Cris Shore, Veronica Strang, Richard Wilson, e
Mark Nuttall. London: Sage.

\_\_\_\_\_. 2012b. ``Culture Variation in Rugby Skills: A Preliminary
Neuroanthropological Report.'' \emph{Annals of Anthropological Practice}
36 (1): 26--44.

\_\_\_\_\_. 2012c. ``Balancing Across Cultures: Equilibrium in
Capoeira.'' In Lende e Downey 2012a.

Downey, Greg, e Daniel H. Lende. 2012. ``Neuroanthropology and the
Encultured Brain.'' In Lende e Downey 2012a.

Draaisma, Douwe. 2009. ``Echos, Doubles, and Delusions: Capgras
Syndromein Science and Literature.'' \emph{Style} 43 (3): 429--441.

Dresler, Martin, Org. 2009. \emph{Neuroästhetik.
Kunst---Gehirn---Wissenschaft}. Leipzig: Seemann.

Drevets, Wayne. 1998. ``Functional Neuroimaging Studies of Depression:
The Anatomy of Melancholia.'' \emph{Annual Review of Medicine}
49:341--361.

Droz, Marion M. 2011. ``La plasticité cérébrale de Cajal à Kandel.
Cheminement d'une notion constitutive du sujet cérébral.'' \emph{Revue
d'Histoire des Sciences} 63 (2): 331--367.

Dudley, Kevin J., et al. 2011. ``Epigenetic Mechanisms Mediating
Vulnerability and Resilience to Psychiatric Disorders.''
\emph{Neuroscience and Biobehavioral Reviews} 35:1544--1551.

Dumit, Joseph. 2003. ``Is It Me or My Brain? Depression and
Neuroscientific Facts.'' \emph{Journal of Medical Humanities} 24 (12):
35--46.

\_\_\_\_\_. 2004. \emph{Picturing Personhood. Brain Scans and Biomedical
Identity}. Princeton, N.J.: Princeton University Press.

\_\_\_\_\_. 2012. \emph{Drugs for Life: How Pharmaceutical Companies
Define Our Health}. Durham, N.C.: Duke University Press.

Dunsmoor, Joseph E., et al. 2015. ``Emotional Learning Selectively and
Retroactively Strengthens Memories for Related Events.'' \emph{Nature}
520:345--348.

Eagleton, Terry. 2000. \emph{The Idea of Culture}. Malden, Mass.:
Blackwell (trad. port.: \emph{A Ideia de Cultura}. Sao Paulo: Unesp,
2011).

Eaton, William R. 2005. \emph{Boyle on Fire: The Mechanical Revolution
in Scientific Explanation}. New York: Continuum.

Ecker, Christine, et al. 2010. ``Describing the Brain in Autism in Five
Dimensions---Magnetic Resonance Imaging--Assisted Diagnosis of Autism
Spectrum Disorder Using a Multiparameter Classification Approach.''
\emph{Journal of Neuroscience} 30 (32): 10612--10623.

Ecks, Stefan. 2013. \emph{Eating Drugs: Psychopharmaceutical Pluralism
in India}. New York: \versal{NYU} Press.

Ecks, Stefan, e Soumita Basu. 2009. ``The Unlicensed Lives of
Antidepressants in India: Generic Drugs, Unqualified Practitioners, and
Floating Prescriptions.'' \emph{Transcultural Psychiatry} 46:86--106.

Eco, Umberto. 2007. \emph{On Ugliness}. Traduzido por Alastair McEwen.
New York: Rizzoli (trad. port.: \emph{História da Feiura}. Rio de
Janeiro: Record, 2014).

Edwards, Betty. 1979. \emph{Drawing on the Right Side of the Brain.} Los
Angeles: J. P. Tarcher. (trad. port.: \emph{Desenhando com o Lado
Direito do Cérebro}. Rio de Janeiro: Ediouro Publicações, 2003.).

Edwards, Jeannette, Penny Harvey, e Peter Wade, Orgs. 2010.
\emph{Technologized Images, Technologized Bodies}, 65--84. New York:
Berghahn.

Ehrenberg, Alain. 2004. ``Le sujet cérébral.'' \emph{Esprit}
309:130--155.

Ehrenwald, Jan. 1984. \emph{Anatomy of Genius: Split Brains and Global
Minds.} New York: Human Sciences.

Eichenbaum, Howard. 2012. \emph{The Cognitive Neuroscience of Memory: An
Introduction}. 2 ed. New York: Oxford University Press.

Eijkholt, Marleen, James A. Anderson, e Judy Illes. 2012. ``Picturing
Neuroscience Research Through a Human Rights Lens: Imaging First Episode
Schizophrenic Treatment"-Naïve Individuals.'' \emph{International Journal
of Law and Psychiatry} 35:146--152.

Eklund, Anders, Thomas E. Nichols, e Hans Knutsson. 2016. ``Cluster
Failure: Why f\versal{MRI} Inferences for Spatial Extent Have Inflated
False"-PositiveRates.'' \emph{\versal{PNAS}} 113 (28): 7900--7905.

Erk, Susanne, Henrik Walter, e Manfred Spitzer. 2002. ``Functional
Neuroimaging of Depression.'' \emph{Advances in Biological Psychiatry}
21:63--69.

Esch, Tobias. 2014. ``The Neurobiology of Meditation and Mindfulness.''
In \emph{Meditation: Neuroscientific Approaches and Philosophical
Implications}, Org. Stefan Schmidt e Harald Walach. New York: Springer.

Evans, Warren F. 1874. \emph{Mental Medicine: A Theoretical and
Practical Treatise on Mental Psychology}.3 ed. Boston: Carter \& Pettee.

Eyal, Gil, et al. 2010. \emph{The Autism Matrix: The Social Origins of
the Autism Epidemic}. Cambridge: Polity.

Falk, Dean, Frederick E. Lepore, e Adrianne Noe. 2012. ``The Cerebral
Cortex of Albert Einstein: A Description and Preliminary Analysis of
Unpublished Photographs.'' \emph{Brain}. doi:10.1093/brain/aws295.

Falk, John H., e John D. Balling. 2010. ``Evolutionary Influence on
Human Landscape Preference.'' \emph{Environment and Behavior}
42:479--493.

Farah, Martha J., Org. 2010a. \emph{Neuroethics: An Introduction with
Readings}. Cambridge, Mass.: \versal{MIT} Press.

\_\_\_\_\_. 2010b. ``Neuroethics: An Overview.'' In Farah 2010a.

\_\_\_\_\_. 2014. ``Brain Images, Babies, and Bathwater: Critiquing
Critiques of Functional Neuroimaging.'' In \emph{Interpreting
Neuroimages: An Introduction to the Technology and Its Limits, Hastings
Center Report} 45 (2): S19-S30. doi:10.1002/hast.295.

Farah, Martha J., e Cayce J. Hook. 2013. ``The Seductive Allure of
`Seductive Allure.''' \emph{Perspectives on Psychological Science} 8
(1): 88--90.

Fernandez"-Duque, Diego, Jessica Evans, Colton Christian, e Sara D.
Hodges. 2015. ``Superfluous Neuroscience Information Makes Explanations
of Psychological Phenomena More Appealing.'' \emph{Journal of Cognitive
Neuroscience} 27 (5): 926--944.

Ferrari, Alize J., et al. 2013. ``Burden of Depressive Disorders by
Country, Sex, Age, and Year: Findings from the Global Burden of Disease
Study 2010.'' \emph{\versal{PL}o\versal{S} Medicine} 10 (11): e1001547.
doi:10.1371/journal.pmed.1001547.

Ferret, Stéphane. 1993. \emph{Le philosophe et son scalpel. Le problème
de l'identité personnelle}. Paris: Minuit.

Fimiani, Filippo. 2009. ``Simulations incorporées et tropismes
empathiques. Notes sur la neuro"-esthétique'' \emph{Images Re"-vues.
Histoire, Anthropologie et Théorie de l'Art} 6 (Junho).
\textless{}\emph{http://imagesrevues.revues.org/426}\textgreater{}.

Finger, Stanley. 2000. \emph{Minds Behind the Brain: A History of the
Pioneers and Their Discoveries}. New York: Oxford University Press.

Finger, Stanley, Dahlia W. Zaidel, François Boller, e Julien
Bogousslavsky, Orgs. 2013. \emph{The Fine Arts, Neurology, and
Neuroscience. New Discoveries and Changing Landscapes}. Progress in
Brain Research 204. Amsterdam: Elsevier.

Finn, Emily S. 2015. ``Brain Activity Is as Unique---and
Identifying---as a Fingerprint.'' \emph{The Conversation} (12 Outubro).
\textless{}\emph{https://bit.ly/322j8SS}\textgreater{}.

Finn, Emily S., et al. 2015. ``Functional Connectome Fingerprinting:
Identifying Individuals Using Patterns of Brain Connectivity.''
\emph{Nature Neuroscience} 18 (11): 1664--1671.

Fitch, Tecumseh W., Antje von Graevenitz, e Eric Nicolas. 2009.
``Bio"-Aesthetics and the Aesthetic Trajectory: A Dynamic Cognitive and
Cultural Perspective.'' In Skov e Vartanian 2009a.

Fitzgerald, Des e Felicity Callard. 2014. ``Social Science and
Neuroscience Beyond Interdisciplinarity: Experimental Entanglements.''
\emph{Theory, Culture, and Society} 32 (1): 3--32.

Fitzgerald, Des, et al. 2014. ``Ambivalence, Equivocation, and the
Politics of Experimental Knowledge: A Transdisciplinary Neuroscience
Encounter.'' \emph{Social Studies of Science} 44 (5): 701--721.

Fitzgerald, Des, Nikolas Rose, e Ilina Singh. 2016a. ``Revitalizing
Sociology: Urban Life and Mental Illness Between History and the
Present.'' \emph{British Journal of Sociology} 67 (1): 138--160.

\_\_\_\_\_. 2016b. ``Living Well in the \emph{Neuropolis}.''
\emph{Sociological Review Monographs} 64:221--237.

Fitzgerald, Paul B., Angela R. Laird, Jerome Maller, e Zafiris J.
Daskalakis. 2008. ``A Meta"-Analytic Study of Changes in Brain Activation
in Depression.'' \emph{Human Brain Mapping} 29 (6): 683--695.

Fitzpatrick, Susan M. 2012. ``Functional Brain Imaging. Neuro"-Turn or
Wrong Turn?'' In Littlefield e Johnson 2012, 180--198.

Foley, Debra L., e Katherine I. Morley. 2011. ``Systematic Review of
Early Cardiometabolic Outcomes of the First Treated Episode of
Psychosis.'' \emph{Archives of General Psychiatry} 68:609--616.

Fombonne, Eric. 2003. ``Modern Views on Autism.'' \emph{Canadian Journal
of Psychiatry} 48 (8): 503--506.

Fonagy, Peter, et al. 2015. ``Pragmatic Randomized Controlled Trial of
Long"-Term Psychoanalytic Psychotherapy for Treatment"-Resistant
Depression: The Tavistock Adult Depression Study (\versal{TADS}).'' \emph{World
Psychiatry} 14:312--321.

Ford, Andrew. 1999. ``Performing Interpretation: Early Allegorical
Exegesis of Homer.'' In \emph{Epic Traditions in the Contemporary World:
The Poetics of Community}, Org. Margaret Beissinger, Jane Tylus, e
Susanne Wofford, 33--53. Berkeley: University of California Press.

Forest, Denis. 2014. \emph{Neuroscepticisme. Les sciences du cerveau
sous le scalpel de l'épistémologue}. Paris: Ithaque.

Forry, Steven Earl. 1990. \emph{Hideous Progenies: Dramatizations of
Frankenstein from Mary Shelley to the Present}. Philadelphia: University
of Pennsylvania Press.

Foucault, Michel. 1969. \emph{A Arqueologia do Saber}. Rio de Janeiro:
Forense Universitária, 2008.

\_\_\_\_\_. 1983. ``Afterword: The Subject and Power.'' In \emph{Michel
Foucault: Beyond Structuralism and Hermeneutics}, Org. Hubert L. Dreyfus
e Paul Rabinow. Brighton: Harvester. (trad. port.: O Sujeito e o Poder.
In \emph{Michel Foucault: Uma Trajetória Filosófica -- Para Além do
Estruturalismo e da Hermenêutica}, Org. Hubert L. Dreyfus e Paul
Rabinow. 2 ed. Rio de Janeiro: Forense Universitária, 2012.)

\_\_\_\_\_. 1988. ``Technologies of the Self.'' In \emph{Technologies of
the Self}, Org. Luther H. Martin, Huck Gutman, e Patrick H. Hutton.
Amherst: University of Massachusetts Press.

\_\_\_\_\_. 1990. \emph{The Use of Pleasure}. Tradução de R. Hurley. New
York: Vintage. (trad. port.: \emph{História da Sexualidade: O Uso dos
Prazeres}. Tradução de Maria Thereza da Costa Albuquerque. 1. ed, v. 2.
São Paulo: Editora Paz e Terra, 2014.)

Fournier, Jay C., et al. 2010. ``Antidepressant Drug Effects and
Depression Severity: A Patient"-Level Meta"-analysis.'' \emph{\versal{JAMA}} 303
(1): 47--53.

Fraenkel, Béatrice. 2007. ``L'invention de l'art pariétal pré
historique. Histoire d'une expérience visuelle.'' \emph{Gradhiva. Revue
d'Anthropologie et d'Histoire des Arts} 6:18--31.

Frances, Allen J. 2013. \emph{Saving Normal: An Insider's Revolt Against
Out"-of"-Control Psychiatric Diagnosis, \versal{DSM}"-5, Big Pharma, and the
Medicalization of Ordinary Life}. Nova York: William Morrow.

Franzen, Jonathan. 2001. \emph{The Corrections}. New York: Picador.
(trad. port.: \emph{As Correções}. Tradução de Sergio Flaksman. 2 ed.
São Paulo: Companhia das Letras, 2011.)

\_\_\_\_\_. 2002. ``My Father's Brain.'' In \emph{How to Be Alone},
7--38. New York: Farrar, Straus \& Giroux. (trad. port.: \emph{Como
Ficar Sozinho}. Tradução de Oscar Pilagallo. São Paulo: Companhia das
Letras, 2012.)

Frazzetto, Giovanni, e Suzanne Anker. 2009. ``Neuroculture.''
\emph{Nature Reviews Neuroscience} 10:815--821.

Freedberg, David. 1985. \emph{Iconoclasts and Their Motives}. Maarssen:
Gary Schwartz.

\_\_\_\_\_. 1989. \emph{The Power of Images: Studies in the History and
Theory of Response}. Chicago: University of Chicago Press.

\_\_\_\_\_. 2007. ``Empathy, Motion, e Emotion.'' In \emph{Wie sich
Gefühle Ausdruckverschaffen. Emotionen in Nahsicht}, Org. Klaus Herding
e Antje Krause"-Wahl, 17--51. Berlin: Driesen.

\_\_\_\_\_. 2008. ``Antropologia e storia dell'arte: la fine delle
discipline?'' \emph{Ricerche di Storia dell'arte} 94:5--18.

\_\_\_\_\_. 2009a. ``Immagini e risposta emotiva: la prospettiva
neuroscientifica.'' In \emph{Prospettiva Zeri}, Org. Anna Ottani Cavina.
Turin: Umberto Allemandi.

\_\_\_\_\_. 2009b. ``Choirs of Praise: Some Aspects of Action
Understanding in Fifteenth"-Century Painting and Sculpture.'' In
\emph{Medieval Renaissance Baroque: A Cat's Cradle for Marilyn Aronberg
Lavin}, Org. David A. Levine e Jack Freiberg. New York: Italica.

\_\_\_\_\_. 2009c. ``Movement, Embodiment, Emotion.'' In \emph{Histoire
de l'art et anthropologie}. Paris, \versal{INHA} / Musée du quai Branly.
\textless{}\emph{http://actesbranly.revues.org/330}\textgreater{}.

Freedberg, David, e Vittorio Gallese. 2007. ``Motion, Emotion, and
Empathy in Aesthetic Experience.'' \emph{Trends in Cognitive Science} 11
(5): 197--203.

Freeman, Jonathan B. 2013. ``Within"-Cultural Variation and the Scope of
Cultural Neuroscience.'' \emph{Psychological Inquiry} 24:26--30.

Freeman, Jonathan B., e Pegeen Cronin. 2002. ``Diagnosing Autism
Spectrum Disorder in Young Children: An Update.'' \emph{Infants and
Young Children}, 14 (3): 1--10.

Freeman, Jonathan B., et al. 2009. ``Culture Shapes a Mesolimbic
Response to Signals of Dominance and Subordination That Associates with
Behavior.'' \emph{NeuroImage} 47:353--359.

Frith, Chris. 2007. \emph{Making up the Mind: How the Brain Creates Our
Mental World}. Hoboken, N.J.: Wiley.

Fuller, Robert C. 1982. \emph{Mesmerism and the American Cure of Souls}.
Philadelphia: University of Pennsylvania Press.

\_\_\_\_\_. 1989. \emph{Alternative Medicine and American Religious
Life}. New York: Oxford University Press.

\_\_\_\_\_. 2001. \emph{Spiritual, But Not Religious: Understanding
Unchurched America.} New York: Oxford University Press.

Furey, Maura L., et al. 2013. ``Potential of Pretreatment Neural
Activity in the Visual Cortex During Emotional Processing to Predict
Treatment Response to Scopolamine in Major Depressive Disorder.''
\emph{\versal{JAMA} Psychiatry} 70 (3): 280--290.

Gabriel, Markus. 2015. \emph{Ich ist nich Gehirn. Philosophie des
Geistes für das 21. Jahrhundert}. Hamburg: Ullstein.

Gabrieli, John D., Satrajit S. Ghosh, e Susan Whitfield"-Gabrieli. 2015.
``Prediction as a Humanitarian and Pragmatic Contribution from Human
Cognitive Neuroscience.'' \emph{Neuron} 85 (1): 11--26.

Gaddy, James. 2007. ``Shadow Boxer.'' \emph{Print} (Julho/Agosto).
\textless{}\emph{http://www.printmag.com/Article/ShadowBoxer}\textgreater{}.

Gainer, Ruth S., e Harold Gainer. 1977. ``Educating Both Halves of the
Brain: Fact or Fancy?'' \emph{Art Education} 30 (5): 20--22.

Galchen, Rivka. 2008. \emph{Atmospheric Disturbances}. London: Harper
Perennial.

Gall, Franz J. 1835. \emph{On the Functions of the Brain and of Each of
Its Parts: With Observations on the Possibility of Determining the
Instincts, Propensities, and Talents, or the Moral and Intellectual
Dispositions of Men and Animals, by the Configuration of the Brain and
Head.} 6 vols. Tradução de Winslow Lewis. Boston: Marsh, Capen \& Lyon.

Gall, Franz J., e Johann"-Caspar Spurzheim. 1809. \emph{Recherches sur le
système nerveux en général, et sur celui du cerveau en particulier.
Mémoire présenté à l'Institut de France, le 14 mars 1808, suivi
d'observations sur le rapport qui en a été fait à cette compagnie par
ses commissaires}. Paris: F. Schoelle e H. Nicolle.

Gallagher, Shaun. 2010. ``The Body's Architecture.'' Lecture delivered
at the Third International Arakawa and Gins: Architecture and Philosophy
Conference.
\textless{}\emph{https://bit.ly/311rQ2z}\textgreater{}.

Gallagher, Shaun, e Dan Zahavi. 2008. \emph{The Phenomenological Mind:
An Introduction to Philosophy of Mind and Cognitive Science}. London:
Routledge.

Gallese, Vittorio. 2007. ``Before and Below Theory of Mind: Embodied
Simulation and the Neural Correlates of Social Cognition.''
\emph{Philosophical Transactions of the Royal Society B} 362:659--669.

\_\_\_\_\_. 2008. ``Empathy, Embodied Simulation, and the Brain:
Commentary on Aragno and Zepf/Hartmann.'' \emph{Journal of the American
Psychoanalytic Association} 56:769--781.

\_\_\_\_\_. 2009. ``Motor Abstraction: A Neuroscientific Account of How
Action Goals and Intentions Are Mapped and Understood.''
\emph{Psychological Research} 73 (4): 486--498.

\_\_\_\_\_. 2011. ``Embodied Simulation Theory: Imagination and
Narrative.'' \emph{Neuropsychoanalysis: An Interdisciplinary Journal for
Psychoanalysis and the Neurosciences} 13 (2): 196--200.

Gallese, Vittorio, e David Freedberg. 2007. ``Mirror and Canonical
Neurons Are Crucial Elements in Aesthetic Response {[}Reply to Casati
and Pignocchi, 2007{]}.'' \emph{Trends in Cognitive Science} 11 (10):
411.

Garland, David. 2014. ``What Is a `History of the Present'? On
Foucault's Genealogies and Their Critical Preconditions.''
\emph{Punishment and Society} 16(4): 365--384.

Gazzaniga, Michael S. 1967. ``The Split Brain in Man.'' \emph{Scientific
American} 217 (2): 24--29.

\_\_\_\_\_. 2005. \emph{The Ethical Brain}. New York: Dana.

Genette, Gérard. 1999 {[}1997{]}. \emph{The Aesthetic Relation}.
Tradução de G. M. Goshgarian. Ithaca, N.Y.: Cornell University Press.

Gennero, Valeria. 2008. ``Gli inganni del cervello. Intervista a Richard
Powers.'' \emph{Acoma, Rivista Internazionale di Studi Nord"-Americani}
37:91--96.

\_\_\_\_\_. 2011. ``Larger Than Our Biologies. Identity and Consciousness
in Contemporary Fiction.'' In Ortega e Vidal 2011, 307--323.

Geraci, Robert M. 2010. \emph{Apocalyptic \versal{AI}: Visions of Heaven in
Robotics, Artificial Intelligence, and Virtual Reality}. New York:
Oxford University Press.

Gere, Cathy, Org. 2004. \emph{The Brain in a Vat}. Special issue of
\emph{Studies in History and Philosophy of Biology and the Biomedical
Sciences} 35.

\_\_\_\_\_. 2011. ```Nature's Experiment:' Epilepsy, Localization of
Brain Function, and the Emergence of the Cerebral Subject.'' In Ortega e
Vidal 2011.

Gibbon, Sahra, e Carlos Novas. 2008a. ``Introduction: Biosocialities,
Genetics, and the Social Sciences.'' In Gibbon e Novas 2008b, 1--18.

\_\_\_\_\_, Orgs. 2008b. \emph{Biosocialities, Genetics, and the Social
Sciences: Making Biologies and Identities}. London: Routledge.

Gibbons, Robert D., et al. 2012. ``Benefits from Antidepressants:
Synthesis of 6-Week Patient"-Level Outcomes from Double"-Blind
Placebo"-Controlled Randomized Trials of Fluoxetine and Venlafaxine.''
\emph{Archives of General Psychiatry} 69 (6): 572--579.

Gibbons, Robert V., et al. 1998. ``A Comparison of Physicians' and
Patients' Attitudes Toward Pharmaceutical Industry Gifts.''
\emph{Journal of General Internal Medicine} 13 (3): 151--154.

Gilbert, Scott F. 1995. ``Resurrecting the Body: Has Postmodernism Had
Any Effect on Biology?'' \emph{Science in Context} 8 (4): 563--577.

Gilmore, Jonathan. 2006. ``Brain Trust.'' \emph{Artforum} (1 Julho):
121--122.

Giordano, James, e Bert Gordijn, Orgs. 2010. \emph{Scientific and
Philosophical Perspectives in Neuroethics}. New York: Oxford University
Press.

Glannon, Walter, Org. 2007. \emph{Defining Right and Wrong in Brain
Science: Essential Readings in Neuroethics}. Washington, D.C.: Dana.

Glannon, Walter. 2011. ``Brain, Behavior, and Knowledge {[}Commentary on
Pardo and Patterson 2011{]}.'' \emph{Neuroethic}s 4: 191--194.

Goggin, Gerard, e Christopher Newell. 2003. \emph{Digital Disability:
The Social Construction of Disability in New Media.} Lanham, Md.: Rowman
\& Littlefield.

Goggin, Gerard, e Tim Noonan. 2006. ``Blogging Disability: The Interface
Between New Cultural Movements and Internet Technology.'' In \emph{Uses
of Blogs}, Org. Axel Bruns e Joanne Jacobs, 161--172. New York: Peter
Lang.

Goh, Joshua O., et al. 2010. ``Culture Differences in Neural Processing
of Faces and Houses in the Ventral Visual Cortex.'' \emph{Social
Cognitive and Affective Neuroscience} 5:227--235.

Goldacre, Ben. 2013. \emph{Bad Pharma: How Drug Companies Mislead
Doctors and Harm Patients}. New York: Faber \& Faber.

Goldberg, Elkhonon. 2001. \emph{The Executive Brain: Frontal Lobes and
the Civilized Mind}. Oxford: Oxford University Press. (trad. port.:
\emph{O Cérebro Executivo: Lobos Frontais e a Mente Civilizada.} Rio de
Janeiro: Imago, 2002).

Goldman, Corrie. 2012. ``This Is Your Brain on Jane Austen, and Stanford
Researchers Are Taking Notes.'' \emph{Stanford Report} (7 Setembro).
\textless{}\emph{https://stanford.io/1iZNOoE}\textgreater{}.

Gombrich, Ernst. 1990. ``The Edge of Delusion {[}Review of Freedberg
1989{]}.'' \emph{New York Review of Books} (15 Fevereiro).
\textless{}\emph{https://bit.ly/35ke6Dk}\textgreater{}.

Gong, Qiyong, e Yong He. 2015. ``Depression, Neuroimaging, and
Connectomics: A Selective Overview.'' \emph{Biological Psychiatry}
77:223--235.

Good, Byron J. 2010. ``The Complexities of Psychopharmaceutical
Hegemonies in Indonesia.'' In \emph{Pharmaceutical Self: The Global
Shaping of Experience in an Age of Psychopharmacology}, Org. Janis H.
Jenkins. Santa Fe, N.M.: School for Advanced Research Press.

Goscilo, Helena. 1981. ``Lermontov's Debt to Lavater and Gall.''
\emph{Slavonic and East European Review} 59 (4): 500--515.

Gotlib, Ian, e Paul J. Hamilton. 2008. ``Neuroimaging and Depression:
Current Status and Unresolved Issues.'' \emph{Current Directions in
Psychological Science} 17:159--163.

Gotlib, Ian, e Constance Hammen, Orgs. 2014. \emph{Handbook of
Depression}. 3 ed. New York: Guilford.

Gould, Stephen J., e Richard C. Lewontin. 1979. ``The Spandrels of San
Marco and the Panglossian Paradigm: A Critique of the Adaptationist
Programme.'' \emph{Proceedings of the Royal Society of London, Series B}
205 (1161):581--598.

Goupil, Georgette, et al. 2014. ``L'utilisation d'internet par les
parents d'enfants ayant un trouble du spectre de l'autisme / Internet
Use by Parents of Children with Autism Spectrum Disorders.''
\emph{Canadian Journal of Learning and Technology / La Revue Canadienne
de l'Apprentissage et de la Technologie} 40:1--18.

Graby, Steven. 2015. ``Neurodiversity: Bridging the Gap Between the
Disabled People's Movement and the Mental Health System Survivors'
Movement?'' In \emph{Madness, Distress, and the Politics of
Disablement}, Org. Helen Spandler, Jill Anderson, e Bob Sapey. Bristol:
Policy.

Graham, Daniel J., e David J. Field. 2007. ``Statistical Regularities of
Art Images and Natural Scenes: Spectra, Sparseness, and
Nonlinearities.'' \emph{Spatial Vision} 21 (1/2): 149--164.

Graham, Julia, et al. 2013. ``Meta"-analytic Evidence for Neuroimaging
Models of Depression: State or Trait?'' \emph{Journal of Affective
Disorders} 151:423--431.

Grande, David. 2010. ``Limiting the Influence of Pharmaceutical Industry
Gifts on Physicians: Self"-Regulation or Government Intervention?''
\emph{Journal of General and Internal Medicine} 25 (1): 79--83.

Grande, David, Judy Shea, e Katrina Armstrong. 2012. ``Pharmaceutical
Industry Gifts to Physicians: Patient Beliefs and Trust in Physicians
and the Health Care System.'' \emph{Journal of General Internal
Medicine} 27:274--279.

Grandin, Temple. 1995. \emph{Thinking in Pictures and Other Reports from
My Life with Autism}. New York: Vintage.

Granello, Darcy Haag, e Todd A. Gibbs. 2016. ``The Power of Language and
Labels: `The Mentally Ill' Versus `People with Mental Illnesses.'''
\emph{Journal of Counseling and Development} 94:31--40.

Gray, Kurt, et al. 2011. ``More Dead Than Dead: Perceptions of Persons
in the Persistent Vegetative State.'' \emph{Cognition} 121 (2):
275--280.

Green, Michael J., et al. 2012. ``Do Gifts from the Pharmaceutical
Industry Affect Trust in Physicians?'' \emph{Family Medicine}
44:325--331.

Greenberg, Gary. 2010. \emph{Manufacturing Depression: The Secret
History of a Modern Disease}. New York: Simon \& Shuster.

\_\_\_\_\_. 2013. \emph{The Book of Woe: The \versal{DSM} and the Unmaking of
Psychiatry}. New York: Blue Rider.

Greenberg, Harvey, e Krin Gabbard. 1999. ``Reel Recollection: Notes on
the Cinematic Depiction of Memory.'' \emph{PsyArt: A Hyperlink Journal
for the Psychological Study of the Arts}.
\textless{}\emph{http://www.psyartjournal.com/article/show/greenberg-reelrecollectionnotesonthecinematic}\textgreater{}.

Greicius, Michael D., et al. 2007. ``Resting"-State Functional
Connectivity in Major Depression: Abnormally Increased Contributions
from Subgenual Cingulate Cortex and Thalamus.'' \emph{Biological
Psychiatry} 62 (5): 429--437.

Grimm, Simone, et al. 2009. ``Increased Self"-Focus in Major Depressive
Disorder Is Related to Neural Abnormalities in Subcortical"-Cortical
Midline Structures.'' \emph{Human Brain Mapping} 30:2617--2627.

Gross, Sky. 2011. ``A Stone in a Spaghetti Bowl: The Biological and
Metaphorical Brain in Neuro"-Oncology.'' In Pickersgill e Van Keulen
2011.

Guidotti, Francesca. 2003. \emph{Cyborg e dintorni, Le formule della
fantascienza}. Bergamo: Bergamo University Press.

Gupta, Akhil, e James Ferguson. 1992. ``Beyond `Culture': Space,
Identity, and the Politics of Difference.'' \emph{Cultural Anthropology}
7:6--23.

Gupta, Mona. 2014. \emph{Is Evidence"-Based Psychiatry Ethical?} Oxford:
Oxford University Press.

Gusfield, Joseph R. 1992. ``Nature's Body and the Metaphors of Food.''
In \emph{Cultivating Differences: Symbolic Boundaries and the Making of
Inequality}, Org. Michèle Lamont e Marcel Fournier. Chicago: University
of Chicago Press. (trad. port.: ``O Corpo da Natureza e as Metáforas do
Alimento.'' In \emph{Cultivando Diferenças: Fronteiras Simbólicas e a
Formação da Desigualdade,} Org. Michèle Lamont e Marcel Fournier.
Tradução de Renata Lucia Bottini. São Paulo: Edições Sesc, 2015.)

Gutchess, Angela H., et al. 2010. ``Neural Differences in the Processing
of Semantic Relationships Across Cultures.'' \emph{Social Cognitive and
Affective Neuroscience} 5:254--263.

Gutchess, Angela H., e Joshua O. Goh. 2013. ``Refining Concepts and
Uncovering Biological Mechanisms for Cultural Neuroscience.''
\emph{Psychological Inquiry} 24:31--36.

Hacking, Ian. 1995. ``The Looping Effects of Human Kinds.'' In
\emph{Causal Cognition: A Multidisciplinary Approach}, Org. Dan Sperber,
David Premack, e Ann J. Premack, 351--383. Oxford: Clarendon.

\_\_\_\_\_. 2002. ``Making Up People.'' In \emph{Historical Ontology},
99--114. Cambridge, Mass.: Harvard University Press. (trad. port.:
\emph{Ontologia Histórica}. Tradução de Leila Mendes. Rio Grande do Sul:
Unisinos, 2009.)

\_\_\_\_\_. 2006. ``What Is Tom Saying to Maureen?'' \emph{London Review
of Books} 28(9).
htttp://www.lrb.co.uk/v28/n09/ian-hacking/what-is-tom-saying-to-maureen.

\_\_\_\_\_. 2009. ``Autistic Autobiography.'' \emph{Philosophical
Transactions of the Royal Society B} 364:1467--1473.

Hagner, Michael. 2001. ``Cultivating the Cortex in German
Neuroanatomy.'' \emph{Science in Context} 14:541--564.

\_\_\_\_\_. 2004. \emph{Geniale Gehirne: Zur Geschichte der
Elitenhirnforschung}. Berlin: Wallstein.

\_\_\_\_\_. 2009 {[}2006{]}. ``The Mind at Work: The Visual
Representation of Cerebral Processes.'' Tradução de U. Froese. In
\emph{Body Within: Art, Medicine, and Visualization}, Org. Renée van de
Vall e Robert Zwijnenberg, 67--90. Leiden: Brill.

Hagner, Michael, e Cornelius Borck. 2001. ``Mindful Practices: On the
Neurosciences in the Twentieth Century.'' \emph{Science in Context}
14:507--510.

Hahn, Torsten. 2005. ``Risk Communication and Paranoid Hermeneutics,
Towards a Distinction Between `Medical Thrillers' and `Mind"-Control
Thrillers' in Narrations on Biocontrol.'' \emph{New Literary History} 36
(2):187--204.

Hallett, Ronald E., e Kristen Barber. 2014. ``Ethnographic Research in a
Cyber Era.'' \emph{Journal of Contemporary Ethnography} 43 (3):
306--330.

Hamilton, Paul J., et al. 2012. ``Functional Neuroimaging of Major
Depressive Disorder: A Meta"-analysis and New Integration of Baseline
Activation and Neural Response Data.'' \emph{American Journal of
Psychiatry} 169:693--703.

Han, Shihui. 2013. ``Culture and Brain: A New Journal.'' \emph{Culture
and Brain} 1(1): 1--2.

Han, Shihui, e Yina Ma. 2014. ``Cultural Differences in Human Brain
Activity: A Quantitative Meta"-analysis.'' \emph{NeuroImage} 99:
293--300.

Han, Shihui, e Georg Northoff. 2008. ``Culture"-Sensitive Neural
Substrates of Human Cognition: A Transcultural Neuroimaging Approach.''
\emph{Nature Reviews Neuroscience} 9:646--654.

Han, Shihui, e Ernst Poppel, Orgs. 2011. \emph{Culture and Neural Frames
of Cognition and Communication}. Berlin: Springer.

Han, Shihui, et al. 2013. ``A Cultural Neuroscience Approach to the
Biosocial Nature of the Human Brain.'' \emph{Annual Review of
Psychology} 64:335--359.

Han, Ying, et al. 2009. ``Gray Matter Density and White Matter Integrity
in Pianists' Brains: A Combined Structural and Diffusion Tensor \versal{MRI}
Study.'' \emph{Neuroscience Letters} 459:3--6.

Hanakawa, Takashi, et al. 2003. ``Neural Correlates Underlying Mental
Calculation in Abacus Experts: Functional Magnetic Resonance Imaging
Study.'' \emph{Neuroimage} 19:296--307.

Hanegraaf, Wouter. 1998. \emph{New Age Religion and Western Culture:
Esotericism in the Mirror of Secular Thought}. Albany: \versal{SUNY} Press.

Hanlon, Charlotte, Abebaw Fekadu, e Vikram Patel. 2014. ``Interventions
for Mental Disorders.'' In Patel, Minas, Cohen, e Prince 2014.

Hansen, Helena, e Mary Skinner. 2012. ``From White Bullets to Black
Markets and Greened Medicine: The Neuroeconomics and Neuroracial
Politics of Opioid Pharmaceuticals.'' \emph{Annals of Anthropological
Practice} 36 (1):167--182.

Hanson, Allan F. 1992. \emph{Testing Testing: Social Consequences of the
Examined Life}. Berkeley: University of California Press.

Hardcastle, Valerie G., e Matthew C. Stewart. 2002. ``What Do Brain Data
Really Show?'' \emph{Philosophy of Science} 69:S72--S82.

Harmon, Amy. 2004a. ``Adults and Autism; An Answer, but Not a Cure, for
a Social Disorder.'' \emph{New York Times} (29 Abril).

\_\_\_\_\_. 2004b. ``Neurodiversity Forever: The Disability Movement
Turns to Brains.'' \emph{New York Times} (9 Maio).

\_\_\_\_\_. 2004c. ``How About Not `Curing' Us, Some Autistics Are
Pleading.'' \emph{New York Times} (20 Dezembro).

Harrington, Anne. 1987. \emph{Mind, Medicine, and the Double Brain: A
Study in Nineteenth"-Century Thought}. Princeton, N.J.: Princeton
University Press.

\_\_\_\_\_. 1991. ``Beyond Phrenology: Localization Theory in the Modern
Era.'' In \emph{The Enchanted Loom}: \emph{Chapters in the History of
Neuroscience}, Org. Pietro Corsi. New York: Oxford University Press.

\_\_\_\_\_. 2008. \emph{The Cure Within: A History of Mind"-Body
Medicine}. New York: Norton.

Harrington, Anne, e Godehard Oepen. 1989. ``Whole Brain Politics and
Brain Laterality Research.'' \emph{European Archives of Psychiatry and
Neurological Science} 239 (3): 141--143.

Harrington, Jean, e Christine Hauskeller. 2014. ``Translational
Research: An Imperative Shaping the Spaces in Biomedicine.''
\emph{\versal{TECNOSCIENZA}: Italian Journal of Science and Technology Studies} 5
(1): 191--201.

Harris, Charles B. 2008. ``The Story of the Self, \emph{The Echo Maker},
and Neurological Realism.'' In Burn e Dempsey 2008, 230--259.

Harris, Lauren J. 1980. ``Left"-Handedness: Early Theories, Facts, and
Fancies.'' In \emph{Neuropsychology of Left"-Handedness}, Org. Jeannine
Herron. New York: Academic Press.

\_\_\_\_\_. 1985. ``Teaching the Right Brain: Historical Perspective on a
Contemporary Educational Fad.'' In \emph{Hemispheric Function and
Collaboration in the Child}, Org. Catherine T. Best. New York: Academic
Press.

Harris, Paul, e Alison Flood. 2010. ``Literary Critics Scan the Brain to
Find Out Why We Love to Read.'' \emph{The Observer} (11 Abril).
\textless{}\emph{https://bit.ly/2Mv629Y}\textgreater{}.

Hart, F. Elizabeth. 2001. ``The Epistemology of Cognitive Literary
Studies.'' \emph{Philosophy and Literature} 25 (2): 314--334.

Hart, Sarah, et al. 2013. ``Altered Fronto"-limbic Activity in Children
and Adolescents with Familial High Risk for Schizophrenia.''
\emph{Psychiatry Research: Neuroimaging} 212:19--27.

Harvey, Ruth. 1975. \emph{The Inward Wits: Psychological Theory in the
Middle Ages and the Renaissance}. London: Warburg Institute.

Hasler, Felix. 2009. ``Stoppt den Neurowahn!'' \emph{Das Magazin} (23
Outubro).\\
\textless{}\emph{https://bit.ly/2MvZ7xD}\textgreater{}.

\_\_\_\_\_. 2013. \emph{Neuromythologie. Eine Streitschrift gegen die
Deutungsmacht der Hirnforschung}. Bielefeld: transcript.

Hasler, Gregor, 2010. ``Pathophysiology of Depression: Do We Have Any
Solid Evidence of Interest to Clinicians?'' \emph{World Psychiatry}
9:155--161.

Healy, David. 1997. \emph{The Antidepressant Era}. Cambridge, Mass.:
Harvard University Press.

\_\_\_\_\_. 2002. \emph{The Creation of Psychopharmacology}. Cambridge,
Mass.: Harvard University Press.

\_\_\_\_\_. 2004. \emph{Let Them Eat Prozac: The Unhealthy Relationship
Between the Pharmaceutical Industry and Depression}. New York: New York
University Press.

\_\_\_\_\_. 2008. \emph{Mania: A Short History of Bipolar Disorder}.
Baltimore, Md.: Johns Hopkins University Press.

\_\_\_\_\_. 2013. \emph{Pharmageddon}. Berkeley: University of California
Press.

Healy, Melissa. 2013. ``Einstein's Brain a Wonder of Connectedness.''
\emph{Los Angeles Times} (10 Outubro).
\textless{}\emph{https://lat.ms/310Y11Z}\textgreater{}.

Hedden, Trey, et al. 2008. ``Cultural Influences on Neural Substrates of
Attentional Control.'' \emph{Psychological Science} 19:12--17.

Heim, Christine, e Elisabeth B. Binder. 2012. ``Current Research Trends
in Early Life Stress and Depression: Review of Human Studies on
Sensitive Periods, Gene"-Environment Interactions, and Epigenetics.''
\emph{Experimental Neurology} 233 (1): 102--111.

Heinz, Andreas, et al. 2014. ``The Uncanny Return of the Race Concept.''
\emph{Frontiers in Human Neuroscience} 8, art. 836.
doi:10.3389/fnhum.2014.00836.

Hendrickx, Sarah. 2010. \emph{The Adolescent and Adult Neuro"-Diversity
Handbook: Asperger Syndrome, \versal{ADHD}, Dyslexia, Dyspraxia, and Related
Conditions}. London: Jessica Kingsley.

Hendrix, Scott E., e Christopher J. May. 2012. ``Neuroscience and the
Quest for God.'' In Littlefield e Johnson 2012.

Herman, Luc, e Bart Vervaeck. 2009. ``Capturing Capgras, \emph{The Echo
Maker} by Richard Powers.'' \emph{Style} 43 (3): 407--428.

Hickok, Gregory. 2009. ``Eight Problems for the Mirror Neuron Theory of
Action Understanding in Monkeys and Humans.'' \emph{Journal of Cognitive
Neuroscience} 21 (7): 1229--1243.

Hodges, Brian. 1995. ``Interactions with the Pharmaceutical Industry:
Experiences and Attitudes of Psychiatry Residents, Interns, and
Clerks.'' \emph{Canadian Medical Association Journal} 153 (5): 553--559.

Hofmann, Bjorn. 2015. ``Exit Exceptionalism: Mental Disease Is Like Any
Other Medical Disease.'' \emph{Journal of Psychiatry and Neuroscience}
45 (6): E36.

Hofstadter, Douglas R., e Daniel C. Dennett, Orgs. 1981. \emph{The
Mind's I: Fantasies and Reflections on Self and Soul.} Toronto: Bantam.

Holland, Norman. 1988. \emph{The Brain of Robert Frost}. New York:
Routledge.

Holland, Stephen, Celia Kitzinger, e Jenny Kitzinger. 2014. ``Death,
Treatment Decisions and the Permanent Vegetative State: Evidence from
Families and Experts.'' \emph{Medicine, Health Care, and Philosophy} 17
(3): 413--423.

Holtzheimer, Paul E., e Helen S. Mayberg. 2011. ``Stuck in a Rut:
Rethinking Depression and Its Treatment.'' \emph{Trends in
Neurosciences} 34 (1): 1--9.

Horgan, John. 2014. ``Much"-Hyped Brain"-Implant Treatment for Depression
Suffers Setback.'' \emph{Scientific American} (11 Março).
\textless{}\emph{https://bit.ly/2MoFUxE}\textgreater{}.

Horwitz, Allan, e Jerome Wakefield. 2007. \emph{The Loss of Sadness: How
Psychiatry Transformed Normal Sorrow Into Depressive Disorder}. New
York: Oxford University Press. (trad. port.: \emph{Tristeza Perdida:
Como a Psiquiatria Transformou a Depressão em Moda.} São Paulo: Summus,
2010.)

Hoyer, Armin. 2010. \emph{Neurotechnologie, Philosophie und
Hirnforschung: Zur Entstehung und Institutionalisierung der Neuroethik}.
Diss. Mestrado, Johann Wolfgang Goethe"-Universität, Frankfurt am Main.

Hsu, Chung"-Ting, et al. 2015. ``The Magical Activation of Left Amygdala
when Reading Harry Potter: Na f\versal{MRI} Study on How Descriptions of
Supra"-Natural Events Entertain and Enchant.'' \emph{\versal{PL}o\versal{S ONE}}.
doi:10.1371/ journal.pone.0118179.

Huarte de San Juan, Juan. 1698. \emph{The tryal of wits. Discovering the
great difference of wits among men, and what sort of learning suits best
with each genius}. Tradução de ``Mr. Bellamy.'' London: Printed for
Richard Sare.

Hubbard, Ruth, e Elijah Wald. 1993. \emph{Exploding the Gene Myth}.
Boston: Beacon.

Hughes, Jane. 2010. ``New Brain Scan to Diagnose Autism.'' \emph{\versal{BBC}
News Health} (10 Agosto).
\textless{}\emph{https://bbc.in/3213PtL}\textgreater{}.

Hultman, Rainbo, Stephen D. Mague, Qiang Li, et al. 2016.
``Dysregulation of Prefrontal Cortex"-Mediated Slow"-Evolving Limbic
Dynamics Drives Stress"-Induced Emotional Pathology.'' \emph{Neuron}
91:439--452.

Hunter, Madeleine. 1976. ``Right"-Brained Kids in Left"-Brained Schools.''
\emph{Today's Education} 65 (4): 45--48.

Hyde, Luke W., et al. 2015. ``Cultural Neuroscience: New Directions as
the Field Matures. What Do Cultural Neuroscience Findings Mean?''
\emph{Culture and Brain} 3:75--92.

Hyman, John. 2006. ``Art and Neuroscience.''
\textless{}\emph{https://bit.ly/2OCaN4n}\textgreater{}.

Hyman, Steven E. 2007. ``Can Neuroscience Be Integrated Into the \versal{DSMV}?''
\emph{Nature Reviews Neuroscience} 8:725--732.

\_\_\_\_\_. 2008. ``A Glimmer of Light for Neuropsychiatric Disorders.''
\emph{Nature} 455:890--893.

\_\_\_\_\_. 2009. ``How Adversity Gets Under the Skin.'' \emph{Nature
Neuroscience} 12 (3): 241--243.

Illes, Judy, Org. 2006. \emph{Neuroethics: Defining the Issues in
Theory, Practice, and Policy}. New York: Oxford University Press.

Illes, Judy, e Eric Racine. 2005. ``Imaging or Imagining? A Neuroethics
Challenge Informed by Genetics.'' \emph{American Journal of Bioethics} 5
(2): 5--18.

Illes, Judy, Eric Racine, e Matthew P. Kirschen. 2006. ``A Picture Is
Worth a Thousand Words, but Which One Thousand?'' In \emph{Illes} 2006.

Illes, Judy, et al. 2008. ``In the Mind's Eye: Provider and Patient
Attitudes on Functional Brain Imaging.'' \emph{Journal of Psychiatric
Research} 43 (2): 107--114. doi:10.1016/j.jpsychires.2008.02.008.

Illes, Judy, e Barbara J. Sahakian, Orgs. 2011. \emph{Oxford Handbook of
Neuroethics}. New York: Oxford.

Ingram, Rick E., Org. 2009. \emph{International Encyclopedia of
Depression}. New York: Springer.

Insel, Thomas R. 2012. {[}Entrevista com{]}. \emph{Psychiatric Annals}
42 (9): 350--351.

\_\_\_\_\_. 2013. ``Transforming Diagnosis.''
\textless{}\emph{http://www.nimh.nih.gov/about/director/2013/ transforming-iagnosis.shtml}\textgreater{}.

Insel, Thomas R., e Remi Quirion. 2005. ``Psychiatry as a Clinical
Neuroscience Discipline.'' \emph{\versal{JAMA}} 294 (17): 2221--2224.

Insel, Thomas, et al. 2010. ``Research Domain Criteria (\versal{RD}o\versal{C}): Toward a
New Classification Framework for Research on Mental Disorders.''
\emph{American Journal of Psychiatry} 167:748--751.

Ioannidis, John P. A. 2015. ``Translational Research May Be Most
Successful When It Fails.'' \emph{Hastings Center Report} 45 (2):
39--40.

Ione, Amy. 2003. ``Examining Semir Zeki's `Neural Concept Formation and
Art: Dante, Michelangelo, Wagner.' '' \emph{Journal of Consciousness
Studies} 10 (2): 58--66.

Ishizu, Tomohiro, e Semir Zeki. 2011. ``Toward a Brain"-Based Theory of
Beauty.'' \emph{\versal{PL}o\versal{S ONE}} 6 (7): 1--10.

Jackson, John. 1905. \emph{Ambidexterity or Two"-Handedness and
Two"-Brainedness: An Argument for Natural Development and Rational
Education}. London: Kegan Paul, Trench, Trübner \& Co.

Jackson, Stanley W. 1986. \emph{Melancholia and Depression: From
Hippocratic Times to Modern Times}. New Haven, Conn.: Yale University
Press.

Jacob, K. Stanly, e Vikram Patel. 2014. ``Classification of Mental
Disorders: A Global Mental Health Perspective.'' \emph{The Lancet}
383:1433--1435.

Jacob, Pierre. 2008. ``What Do Mirror Neurons Contribute to Human Social
Cognition?'' \emph{Mind and Language} 23 (2): 190--223.

Jacobsen, Thomas, Ricarda I. Schubotz, Lea Hofel, e D. Yves von Cramon.
2006. ``Brain Correlates of Aesthetic Judgment of Beauty.''
\emph{NeuroImage} 29: 276--285.

Jaeger, Paul T. 2012. \emph{Disability and the Internet: Confronting a
Digital Divide}. Boulder, Colo.: Lynne Rienner.

Jaroff, Leon. 1989. ``The Gene Hunt.'' \emph{Time} (20 Março).
\textless{}\emph{https://bit.ly/31ZdyRi}\textgreater{}.

Joel, Daphna, et al. 2015. ``Sex Beyond the Genitalia: The Human Brain
Mosaic.'' \emph{\versal{PNAS}.}
\textless{}\emph{https://bit.ly/2OuyhIK}\textgreater{}.

Johnsen, Tom J., e Oddgeir Friborg. 2015. ``The Effects of Cognitive
Behavioral Therapy as an Anti"-Depressive Treatment Is Falling: A
Meta"-analysis.'' \emph{Psychological Bulletin} 141 (4): 747--768.

Johnson, Davi. 2008. `` `How Do You Know Unless You Look?'': Brain
Imaging, Biopower, and Practical Neuroscience.'' \emph{Journal of
Medical Humanities} 29:147--161.

Johnson, Gary. 2008. ``Consciousness as Content: Neuronarratives and the
Redemption of Fiction.'' \emph{Mosaic} 41 (1): 169--184.

Jones, Gareth D. 1989. ``Brain Birth and Personal Identity.''
\emph{Journal of Medical Ethics} 15:173--178.

\_\_\_\_\_. 1998. ``The Problematic Symmetry Between Brain Birth and
Brain Death.'' \emph{Journal of Medical Ethics} 24:237--242.

Jones, Nev, e Timothy Kelly. 2015. ``Inconvenient Complications: On the
Heterogeneities of Madness and Their Relationship to Disability.'' In
\emph{Madness, Distress, and the Politics of Disablement}, Org. Helen
Spandler, Jill Anderson e Bob Sapey. Bristol: Policy.

Jones, Rachel. 2012. ``What Makes a Human Brain?'' \emph{Nature Reviews
Neuroscience} 13 (10): 655.

Jones, Robert S. P., e Tor O. Meldal. 2001. ``Social Relationships and
Asperger's Syndrome. A Qualitative Analysis of First"-Hand Accounts.''
\emph{Journal of Intellectual Disabilities} 5 (1): 35--41.

Jones, Robert S. P., Andrew Zahl, e Haci C. Huws. 2001. ``First"-Hand
Accounts of Emotional Experiences in Autism: A Qualitative Analysis.''
\emph{Disability and Society} 16 (3): 393--401.

Jones, Simon R., e Charles Fernyhough. 2007. ``A New Look at the Neural
Diathesis"-Stress Model of Schizophrenia: The Primacy of
Social"-Evaluative and Uncontrollable Situations.'' \emph{Schizophrenia
Bulletin} 33 (5): 1171--1177.

Joyce, Kelly A. 2008\emph{. Magnetic Appeal: \versal{MRI} and the Myth of
Transparency}. Ithaca, N.Y.: Cornell University Press.

Jurecic, Ann. 2007. ``Neurodiversity.'' \emph{College English} 69 (5):
421--442.

Kabat"-Zinn, Jon, e Richard J. Davidson. 2015. ``A Confluence of Streams
and a Flowering of Possibilities.'' \emph{In The Mind's Own Physician: A
Scientific Dialogue with the Dalai Lama on the Healing Power of
Meditation}, Org. Jon Kabat"-Zinn e Richard J. Davidson, 1--19. Oakland,
Calif.: Mind and Life Institute / New Harbinger Publications.

Kachka, Boris. 2012. ``Proust Wasn't a Neuroscientist. Neither Was Jonah
Lehrer.'' \emph{New York Magazine} (28 Out.).
\textless{}\emph{https://nym.ag/2brzM4H}\textgreater{}.

Kaitaro, Timo. 2004. ``Brain"-Mind Identities in Dualism and Materialism:
A Historical Perspective.'' \emph{Studies in History and Philosophy of
Biology and Biomedical Sciences} 35:627--645.

Kapur, Shitij, Anthony G. Phillips, e Thomas Insel. 2012. ``Why Has It
Taken So Long for Biological Psychiatry to Develop Clinical Tests and
What to Do About It?'' \emph{Molecular Psychiatry} 17 (12): 1174--1179.

Kaufman, Sharon R., e Lynn M. Morgan. 2005. ``The Anthropology of the
Beginnings and Ends of Life.'' \emph{Annual Review of Anthropology}
34:317--341.

Kawabata, Hideaki, e Semir Zeki. 2004. ``Neural Correlates of Beauty.''
\emph{Journal of Neurophysiology} 91:1699--1705.

Keedwell, Paul. 2009. ``Brain Circuitry.'' In Ingram 2009.

Keim"-Malpass, Jessica, Richard H. Steeves, e Christine Kennedy. 2014.
``Internet Ethnography: A Review of Methodological Considerations for
Studying Online Illness Blogs.'' \emph{International Journal of Nursing
Studies} 51(12): 1686--1692.

Kelley, William M., C. Neil Macrae, Carrie L. Wyland, Sali Caglar,
Souheil Inati, e Todd F. Heatherton. 2002. ``Finding the Self? An
Event"-Related f\versal{MRI} Study.'' \emph{Journal of Cognitive Neurosciences}
14:785--794.

Kellogg, John Harvey. 1887. \emph{First Book in Physiology and Hygiene}.
New York: Harper \& Brothers.

Kemp, Simon. 1990. \emph{Medieval Psychology}. New York: Greenwood.

Kempton, Matthew J., John R. Geddes, Ulrich Ettinger, Simon C. Williams,
e Paul M. Grasby. 2008. ``Meta"-analysis, Database, and Meta"-regression
of 98 Structural Imaging Studies in Bipolar Disorder.'' \emph{Archives
of General Psychiatry} 65 (9): 1017--1032.

Kempton, Matthew J., et al. 2011. ``Structural Neuroimaging Studies in
Major Depressive Disorder: Meta"-analysis and Comparison with Bipolar
Disorder.'' \emph{Archives of General Psychiatry} 68 (7): 675--690.

Kenway, Ian M. 2009. ``Blessing or Curse? Autism and the Rise of the
Internet.'' \emph{Journal of Religion, Disability, and Health} 13 (2):
94--103.

Keysers, Christian. 2011. \emph{The Empathic Brain:} \emph{How Mirror
Neurons Help You Understand Others}. Amsterdam: Social Brain.

Khamsi, Roxanne. 2013. ``Brain Scans Could Become \versal{EKG}s for Mental
Disorders.'' \emph{Time} (28 Jun.).
\textless{}\emph{https://bit.ly/1StueGC}\textgreater{}

Kieseppä, Tuula, et al. 2009. ``Major Depressive Disorder and White
Matter Abnormalities: A Diffusion Tensor Imaging Study with Tract"-Based
Spatial Statistics\emph{.'' Journal of Affective Disorders} 120 (1):
240--244.

Kilner, James M., e Roger N. Lemon. 2013. ``What We Know Currently About
Mirror Neurons.'' \emph{Current Biology} 23:R1057--R1062.

Kim, Heejung S., e Joni Y. Sasaki. 2014. ``Cultural Neuroscience:
Biology of the Mind in Cultural Contexts.'' \emph{Annual Review of
Psychology} 65:487--514.

Kirby, David. 2003. ``Scientists on the Set: Science Consultants and
Communication of Science in Visual Fiction.'' \emph{Public Understanding
of Science} 12:261--278.

Kirmayer, Laurence J. 2002. ``Psychopharmacology in a Globalizing World:
The Use of Antidepressants in Japan.'' \emph{Transcultural Psychiatry}
39:295--322.

Kirmayer, Laurence J., e Daina Crafa. 2014. ``What Kind of Science for
Psychiatry.'' \emph{Frontiers in Human Neuroscience} 8, art. 435:1--12.

Kirmayer, Laurence J., e Eugene Raikhel. 2009. ``Editorial: From Amrita
to Substance D: Psychopharmacology, Political Economy, and Technologies
of the Self.'' \emph{Transcultural Psychiatry} 46:5--15.

Kirsch, Irving, et al. 2008. ``Initial Severity and Antidepressant
Benefits: A Meta"-analysis of Data Submitted to the Food and Drug
Administration.'' \emph{\versal{PL}o\versal{S}} \emph{Medicine} 5 (2): e45, 0260--0268.

Kirsch, Irving. 2009. \emph{The Emperor's New Drugs: Exploding the
Antidepressant Myth}. London: Bodley Head.

Kitanaka, Junko. 2011. \emph{Depression in Japan: Psychiatric Cures for
a Society in Distress}. Princeton, N.J.: Princeton University Press.

Kitayama, Shinobu, Sean Duffy, Tadashi Kawamura, e Jeff T. Larsen. 2003.
``Perceiving an Object and Its Context in Different Cultures: A Cultural
Look at New Look.'' \emph{Psychological Science} 14:201--206.

Kitayama, Shinobu, e Jiyoung Park. 2010. ``Cultural Neuroscience of the
Self: Understanding the Social Grounding of the Brain.'' \emph{Social
Cognitive and Affective Neuroscience} 5 (2/3): 111--129.

Kitayama, Shinobu, e Sarah Huff. 2015. ``Cultural Neuroscience:
Connecting Culture, Brain, and Genes.'' \emph{In Emerging Trends in the
Social and Behavioral Sciences: An Interdisciplinary, Searchable, and
Linkable Resource}, Org. Robert A. Scott e Stephen M. Kosslyn. Wiley.
doi:10.1002/9781118900772.

Kiverstein, Julian, e Mark Miller. 2015. ``The Embodied Brain: Towards a
Radical Embodied Cognitive Neuroscience.'' \emph{Frontiers in Human
Neuroscience} 9, art.237. doi:10.3389/fnhum.2015.00237.

Klein, Colin. 2010. ``Images Are Not the Evidence in Neuroimaging.''
\emph{British Journal for the Philosophy of Science} 61:265--278.

Kleinman, Arthur. 2009. ``Global Mental Health: A Failure of Humanity.''
\emph{The Lancet} 374:603--604.

\_\_\_\_\_. 2012. ``Medical Anthropology and Mental Health. Five
Questions for the Next Fifty Years.'' \emph{In Medical Anthropology at
the Intersections: Histories, Activisms, and Futures}, Org. Marcia C.
Inhorn e Emily A. Wentzell. Durham, N.C.: Duke University Press.

Klibansky, Raymond, Erwin Panofsky, e Fritz Saxl. 1964. \emph{Saturn and
Melancholy: Studies in the History of Natural Philosophy, Religion, and
Art}. New York: Basic Books.

Koenigs, Michael, e Daniel Tranel. 2008. ``Prefrontal Cortex Damage
Abolishes Brand"-Cued Changes in Cola Preference.'' \emph{\versal{SCAN}} 3:1--6.

Koolschijn, Cédric, et al. 2009. ``Brain Volume Abnormalities in Major
Depressive Disorder: A Meta"-analysis of Magnetic Resonance Imaging
Studies.'' \emph{Human Brain Mapping} 30 (11): 3719--3735.

Kosslyn, Stephen M. 1999. ``If Neuroimaging Is the Answer, What Is the
Question?'' \emph{Philosophical Transactions of the Royal Society of
London} 354:1283--1294.

Kozinets, Robert T. 2010. \emph{Netnography: Doing Ethnographic Research
Online}. Los Angeles: Sage. (trad. port.: \emph{Netnografia: Realizando
Pesquisa Etnográfica Online.} Tradução de Daniel Bueno. Porto Alegre,
\versal{RS}: Penso, 2014.)

Krementsov, Nikolai. 2009. ``Off with Your Heads: Isolated Organs in
Early Soviet Science and Fiction.'' \emph{Studies in History and
Philosophy of Biological and Biomedical Sciences} 40:87--100.

\_\_\_\_\_. 2014. \emph{Revolutionary Experiments: The Quest for
Immortality in Bolshevik Science and Fiction}. New York: Oxford
University Press.

Kroeber, Alfred L., e Clyde Kluckhohn. 1952. \emph{Culture: A Critical
Review of Concepts and Definitions}. Harvard University Peabody Museum
of American Archeology and Ethnology Papers 47. Cambridge, Mass.
{[}Reproduzido de forma variada.{]}

Krow"-Lucal, Martha G. 1983. ``Balzac, Galdós, and Phrenology.''
\emph{Anales Galdosianos} 18:7--14.

Kupferschimdt, Kai. 2013. ``Concentrating on Kindness\emph{.'' Science}
341 (20 Setembro): 1336--1339.

Kwint, Marius, e Richard Wingate. 2012. \emph{Brains: The Mind as
Matter}. London: Profile.

Lacasse, Jeffrey R., e Jonathan Leo. 2005. ``Serotonin and Depression: A
Disconnect Between the Advertisements and the Scientific Literature.''
\emph{\versal{PL}o\versal{S} Medicine} 2 (12): 1211--1216.

Lacey, Simon, et al. 2011. ``Art for Reward's Sake: Visual Art Recruits
the Ventral Striatum.'' \emph{Neuroimage} 55 (1): 420--433.

Lage, Andrey. 2006. ``Autistas usam remédios para controlar aspectos da
doença.'' \emph{Folha OnLine} (27 Jul.).
\textless{}\emph{https://bit.ly/2LYQ9K6}\textgreater{}.

Lakoff, Andrew. 2005. \emph{Pharmaceutical Reason: Medication and
Psychiatric Knowledge in Argentina}. Cambridge: Cambridge University
Press.

\_\_\_\_\_. 2006. ``High Contact: Gifts and Surveillance in Argentina.''
In Petryna, Lakoff, e Kleinman 2006.

Lakoff, George. 2008. \emph{The Political Mind: A Cognitive Scientist's
Guide to Your Brain and Its Politics}. New York: Penguin.

Lamberton, Robert. 1986. \emph{Homer the Theologian: Neoplatonist
Allegorical Reading and the Growth of the Epic Tradition}. Berkeley:
University of California Press.

Landsberg, Alison. 2004. \emph{Prosthetic Memory: The Transformation of
American Remembrance in the Age of Mass Culture.} New York: Columbia
University Press.

Landi, Anne. 2009. ``Brain Wave.'' \emph{\versal{ART}news} (Jun.): 88--93.

Lane, Christopher. 2007. \emph{Shyness: How Normal Behavior Became a
Sickness}. New Haven, Conn.: Yale University Press.

Lardreau, Guy. 1988. \emph{Fictions philosophiques et science fiction}.
Paris: Actes Sud.

Larkin, Warren, e John Read. 2008. ``Childhood Trauma and Psychosis:
Evidence, Pathways, and Implications.'' \emph{Journal of Postgraduate
Medicine} 54 (4): 287--293.

Latour, Bruno. 2004. ``How to Talk About the Body? The Normative
Dimension of Science Studies.'' \emph{Body and Society} 10 (2/3):
205--229.

Lauer, Gerhard. 2009. ``Going Empirical: Why We Need Cognitive Literary
Studies.'' \emph{Journal of Literary Theory} 3:145--154.

Laureys, Steven, et al. 2010. ``Unresponsive Wakefulness Syndrome: A New
Name for the Vegetative State or Apallic Syndrome.'' \emph{\versal{BMC} Medicine}
8:68. \textless{}\emph{https://bit.ly/2IBGxmq}\textgreater{}.

Lauring, Jon O, Org. 2015. \emph{An Introduction to Neuroaesthetics: The
Neuroscientific Approach to Aesthetic Experience, Artistic Creativity,
and Arts Appreciation}. Copenahgen: Museum Tusculanum Press.

Lazar, Sara W., Catherine E. Kerr, Rachel H. Wasserman, et al. 2005.
``Meditation Experience Is Associated with Increased Cortical
Thickness.'' \emph{Neuroreport} 16 (17): 1893--1897.

Leary, Timothy. 1980. \emph{The Politics of Ecstasy}. Berkeley: Ronin,
1998.

Leary, Timothy, Robert Anton Wilson e George A. Koopman. 1977.
\emph{Neuropolitics: The Sociobiology of Human Metamorphosis.} Los
Angeles: Starseed/Peace.

Leder, Helmut. 2001. ``Determinants of Preference. When Do We Like What
We Know?'' \emph{Empirical Studies of the Arts} 19 (2): 201--211.

Lederer, E. Susan, Org. 2002. \emph{Frankenstein: Penetrating the
Secrets of Nature.} New Brunswick, N.J.: Rutgers University Press.

Lega, Bradley C. 2006. ``An Essay Concerning Human Understanding: How
the Cerebri Anatome of Thomas Willis Influenced John Locke.''
\emph{Neurosurgery} 58:567--576.

Legrenzi, Paolo, e Carlo Umiltà. 2009. \emph{Neuro"-mania: Il cervello
non spiega chi siamo}. Bologna: Il Mulino.

Lehrer, Jonah. 2007. \emph{Proust Was a Neuroscientist}. New York:
Houghton Mifflin Harcourt. (trad. port.: \emph{Proust foi um
Neurocientista: Como a Arte Antecipa a Ciência}. São Paulo: Best Seller,
2010.)

Lei, Miaomei, Hiroyuki Akama, e Brian Murphy. 2014. ``Neural Basis of
Language Switching in the Brain: f\versal{MRI} Evidence from Korean"-Chinese Early
Bilinguals.'' \emph{Brain and Language} 138:12--18.

Leibing, Annette. 2009. ``Tense Prescriptions? Alzheimer Medications and
the Anthropology of Uncertainty.'' \emph{Transcultural Psychiatry}
46:180--206.

Leichsenring, Falk, e Susanne Klein. 2014. ``Evidence for Psychodynamic
Psychotherapy in Specific Mental Disorders: A Systematic Review.''
\emph{Psychoanalytic Psychotherapy} 28 (1): 4--32.

Leichsenring, Falk, e Sven Rabung. 2008. ``Effectiveness of Long"-Term
Psychodynamic Psychotherapy.'' \emph{Journal of the American Medical
Association} 300:1151--1565.

\_\_\_\_\_. 2011. ``Long"-Term Psychodynamic Psychotherapy in Complex
Mental Disorders: Update of a Meta"-analysis.'' \emph{British Journal of
Psychiatry} 199 (1): 15--22.

Lende, Daniel H., e Greg Downey, Orgs. 2012a. \emph{The Encultured
Brain: An Introduction to Neuroanthropology.} Cambridge, Mass.: \versal{MIT}
Press.

\_\_\_\_\_. 2012b. ``The Encultured Brain---Toward the Future.'' In Lende
e Downey 2012a.

Leo, Jonathan, e Jeffrey R. Lacasse. 2008. ``The Media and the Chemical
Imbalance Theory of Depression.'' \emph{Society} 45:35--45.

Lepore, Frederick E. 2001. ``Dissecting Genius. Einstein's Brain and the
Search for the Neural Basis of Intellect.'' \emph{Cerebrum} 3 (1).
\textless{}\emph{http://www.dana.org/Cerebrum/Default.aspx?id​=39337}\textgreater{}.

LeVay, Simon. 1997. \emph{Albrick's Gold}. London: Headline Book.

Levine, Timothy R., Mary J. Bresnahan, Hee S. Park, et al. 2003.
``Self"-Construal Scales Lack Validity.'' \emph{Human Communication
Research} 29 (2): 210--252.

Levy, Neil. 2007. \emph{Neuroethics: Challenges for the Twenty"-First
Century}. New York: Cambridge University Press.

Lichtenstein, Jacqueline, Carole Maigné, e Pierre Arnauld, Orgs.
2013.\emph{Vers La science de l'art. L'esthétique scientifique en France
1857--1937.} Paris: Presses de l'Université Paris Sorbonne.

Littlefield, Melissa M., 2011. \emph{The Lying Brain: Lie Detection in
Science and Science Fiction}. Michigan: University of Michigan Press.

Littlefield, Melissa M., Des Fitzgerald, Kasper Knudsen, James Tonks, e
Martin J. Dietz. 2014. ``Contextualizing Neuro"-Collaborations:
Reflections on a Transdisciplinary f\versal{MRI} Lie Detection Experiment.''
\emph{Frontiers in Human Neuroscience}. doi:10.3389/fnhum.2014.00149.

Littlefield, Melissa M., e Jenell M. Johnson, Orgs. 2012. \emph{The
Neuroscientific Turn: Transdisciplinarity in the Age of the Brain.} Ann
Arbor: University of Michigan Press.

Livingstone, Margaret S. 2000. ``Is It Warm? Is It Real? Or Just Low
Spatial Frequency?'' \emph{Science} 290 (17 Nov.): 1299.

Livingstone, Margaret S., e Bevil R. Conway. 2004. ``Was Rembrandt
Stereoblind?'' \emph{New England Journal of Medicine} 351:1264--1265.

Lock, Margaret. 2002. \emph{Twice Dead: Organ Transplants and the
Reinvention of Death}. Berkeley: University of California Press.

Locke, John. 1690. ``Second Treatise of Government.'' \emph{In Two
Treatises of Government}, Org. Peter Laslett. New York: Cambridge
University Press. (trad. port.: ``Segundo Tratado sobre o Governo.'' In
\emph{Dois Tratados Sobre o Governo}. Tradução de Julio Fischer. São
Paulo: Martins Fontes, 1998.)

\_\_\_\_\_. 1988 {[}1694{]}. \emph{An Essay Concerning Human
Understanding}. 2 ed., Org. Peter H. Nidditch. Oxford: Clarendon. (trad.
port.: \emph{Ensaio sobre o Entendimento Humano,} Org. Pedro Paulo G.
Pimenta. São Paulo: Martins Fontes, 2012).

Lodge, David. 1988. \emph{Nice Work}. London: Secker \& Warburg.

\_\_\_\_\_. 2001. \emph{Thinks} . . . New York: Viking (trad. port:
\emph{Pense\ldots{}} São Paulo: Best Seller, 2001).

Löfholm, Cecilia Andrée, Lars Brännström, Martin Olsson, e Kjell
Hansson. 2013. ``Treatment"-as"-Usual in Effectiveness Studies: What Is It
and Does It Matter?'' \emph{International Journal of Social Welfare}
22:25--34.

Loftus, Elizabeth, e Katherine Ketcham. 1994. \emph{The Myth of
Repressed Memory: False Memories and Allegations of Sexual Abuse.} New
York: St. Martin's.

Logothetis, Nikos K. 2008. ``What We Can Do and What We Cannot Do with
f\versal{MRI}.'' \emph{Nature} 453:869--878.

Lohmann, Gabriele, Johannes Stelzer, Jane Neumann, Nihat Ay, e Robert
Turner. 2013. ```More Is Different' in Functional Magnetic Resonance
Imaging: A Review of Recent Data Analysis Techniques.'' \emph{Brain
Connectivity} 3 (3): 223--239.

Lopez"-Ibor, J. Juan. 2002. ``The \versal{WPA} and the Fight Against Stigma
Because of Mental Illness.'' \emph{World Psychiatry} 1:30--31.

Lord, Catherine, e Rebecca Jones. 2012. ``Annual Research Review:
Re"-thinking the Classification of Autism Spectrum Disorders.''
\emph{Journal of Child Psychology and Psychiatry} 53:490--509.

Luhrmann, Tanya Marie. 2000. \emph{Of Two Minds: An Anthropologist Looks
at American Psychiatry}. New York: Knopf.

\_\_\_\_\_. 2007. ``Social Defeat and the Culture of Chronicity; Or, Why
Schizophrenia Does So Well Over There and So Badly Here.''
\emph{Culture, Medicine, and Psychiatry} 31:135--172.

\_\_\_\_\_. 2012. ``Beyond the Brain.'' Wilson Quarterly (Verão): 28--34.

Luria, Alexander Romanovich. 1966. ``Vygotski et l'étude des fonctions
psychiques supérieures.'' \emph{Recherches Internationales à La Lumière
Du Marxisme} 51:93--103.

\_\_\_\_\_. 1979. \emph{The Making of Mind: A Personal Account of Soviet
Psychology}, Org. Michael Cole e Sheila Cole. Cambridge, Mass.: Harvard
University Press.

Lutz, Amy S. F. 2013. ``You Do Not Have Asperger's: What Psychiatry's
New Diagnostic Manual Means for People on the Autism Spectrum.''
\emph{Slate} (22 Maio).
\textless{}\emph{http://www.slate.com/articles/\_and\_science/medical\_examiner/2013/05/autism\_spectrum\_diagnoses\_the\_dsm\_5\_eliminates\_asperger\_s\_and\_pdd\_nos.html}\textgreater{}.

Maasen, Sabine, e Barbara Sutter, Orgs. 2007. \emph{On Willing Selves:
Neoliberal Politics and the Challenge of Neuroscience.} Basingstoke:
Macmillan.

MacKinnon, Katherine C. 2014. ``Contemporary Biological Anthropology in
2013: Integrative, Connected, and Relevant.'' \emph{American
Anthropologist} 116 (2): 352--365.

Macpherson, Crawford Brough. 1962. \emph{The Political Theory of
Possessive Individualism: Hobbes to Locke}. Oxford: Oxford University
Press. (trad. port.: \emph{A Teoria Política do Individualismo
Possessivo: De Hobbes a Locke}. São Paulo: Paz e Terra, 1979).

Maguire, Eleanor A., David G. Gadian, Ingrid S. Johnsrude, et al. 2000.
``Navigation"-Related Structural Change in the Hippocampi of Taxi
Drivers.'' \emph{\versal{PNAS}} 97 (8): 4398--4403.

Malabou, Catherine. 2008. \emph{What Should We Do with Our Brain}?
Tradução de Sebastian Rand. New York: Fordham University Press.

Malane, Rachel Ann. 2005. \emph{Sex in Mind:} \emph{The Gendered Brain
in Nineteenth"-Century Literature and Mental Sciences.} New York: Peter
Lang.

Marcus, Gary. 2013. ``The Problem with the Neuroscience Backlash.''
\emph{New Yorker} (19 Jun.).
\textless{}\emph{http://www.newyorker.com/online/blogs/elements/2013/06/the-problem-with-theneuroscience-backlash.html}\textgreater{}.

Marcus, Joseph A. 1997. ``Neuroanthropology.'' \emph{In The Dictionary
of Anthropology}, Org. Thomas Barfield, 340--342. Malden, Mass.:
Blackwell.

Mark, Vernon H., e Jeffrey P. Mark. 1991. \emph{Brain Power: A
Neurosurgeon's Complete Program to Maintain and Enhance Brain Fitness
Throughout Your Life.} Boston: Houghton Mifflin.

Marquardt, Wolfgang. 2015. \emph{Human Brain Project Mediation Report}.
Juelich: Mediation of the Human Brain Project c/o Forschungszentrum
Juelich GmbH.
\textless{}\emph{http://www.fzjuelich.de/SharedDocs/Pressemitteilungen/UK/EN/2015/15-03-09hbp-mediation.html}\textgreater{}.

Marsen, Sky. 2004. ``Against Heritage: Invented Identities in Science
Fiction Film.'' \emph{Semiotica} 152 (1/4): 141--157.

Martín"-Aragúz, Antonio, et al., Orgs. 2010. \emph{Neuroestética}.
Madrid: Saned.

Martin, Emily. 2000. ``Mind"-Body Problems.'' \emph{American Ethnologist}
27:569--590.

\_\_\_\_\_. 2007. \emph{Bipolar Expeditions: Mania and Depression in
American Culture.} Princeton, N.J.: Princeton University Press.

\_\_\_\_\_. 2009. ``Identity, Identification, and the Brain.''
Apresentação no workshop ``Neurocultures.'' Max Planck Institute of the
History of Science. Berlin, \emph{20--22} Fevereiro.

\_\_\_\_\_. 2010. ``Self"-Making and the Brain.'' \emph{Subjectivity} 3
(4): 366--381.

Martindale, Colin, Paul Locher, e Vladimir Petrov, Orgs. 2007.
\emph{Evolutionary and Neurocognitive Approaches to Aesthetics:
Creativity and the Arts.} Amityville, N.Y.: Baywood.

Massaro, Davide, Federica Savazzi, Cinzia Di Dio, et al. 2012. ``When
Art Moves the Eyes: A Behavioral and Eye"-Tracking Study.'' \versal{PL}o\versal{S ONE} 7
(5): e37285.doi:10.1371/journal.pone.0037285.

Mateo, Marina Martínez, Maurice Cabanis, Nicole Cruz de Echeverría
Loebell, e Sören Krach. 2012. ``Concerns About Cultural Neuroscience. A
Critical Analysis.'' \emph{Neuroscience and Biobehavioral Reviews} 36
(1): 152--161.

Mateo, Marina Martínez, Maurice Cabanis, Julian Stenmanns, e Sören
Krach. 2013. ``Essentializing the Binary Self: Individualism and
Collectivism in Cultural Neuroscience.'' \emph{Frontiers in Human
Neuroscience} 7, art. 289: 1--4.

Max, Daniel. T. 2007. ``Swann's Hypothesis.'' \emph{New York Times} (4
Nov.).
\textless{}\emph{https://nyti.ms/317jvug}\textgreater{}.

Mayberg, Helen S. 2007. ``Defining the Neural Circuitry of Depression:
Towards a New Nosology with Therapeutic Implications.'' \emph{Biological
Psychiatry} 61:729--730.

\_\_\_\_\_. 2014. ``Neuroimaging and Psychiatry: The Long Road from Bench
to Bedside.'' \emph{Hastings Center Report} 44:S31--S36.

Mayberg, Helen, et al. 2005. ``Deep Brain Stimulation for Clinical Study
of Treatment"-Resistant Depression.'' \emph{Neuron} 45:651--660.

McCabe, David P., e Alan D. Castel. 2008. ``Seeing Is Believing: The
Effect of Brain Images on Judgments of Scientific Reasoning.''
\emph{Cognition} 107:343--352.

McCarthy, Margaret M. 2015. Sex Differences in the Brain,'' \emph{The
Scientist} (1 Out.).
\textless{}\emph{https://bit.ly/1Q4OEmI}\textgreater{}

McClure, Samuel M., Jian Li, Damon Tomlin, et al. 2004. ``Neural
Correlates of Behavioral Preference for Culturally Familiar Drinks.''
\emph{Neuron} 44:379--387.

McEwan, Ian. 2004 {[}1997{]}. \emph{Enduring Love}. London: Vintage.
(trad. port.: \emph{Amor Sem Fim}. Tradução de Jorio Dauster. São Paulo:
Companhia das Letras, 2011).

\_\_\_\_\_. 2005. \emph{Saturday}. New York: Doubleday. (trad. port.:
\emph{Sábado}. Tradução de Rubens Figueiredo. São Paulo: Companhia das
Letras, 2013).

McGee, Micki. 2005. \emph{Self"-Help, Inc.: Makeover Culture in American
Life}. New York: Oxford University Press.

McGrath, Callie L., et al. 2013. ``Toward a Neuroimaging Treatment
Selection Biomarker for Major Depressive Disorder.'' \emph{\versal{JAMA}
Psychiatry} 70 (8): 821--829.

McKinley, Marc. 2011. ``Avoiding a Collapse in Thinking: Commentary on
Jonathan Shedler's `\emph{The Efficacy of Psychodynamic
Psychotherapy}.'''
\textless{}\emph{http://www.apadivisions.org/division39/publications/review/2011/01/psychodynamic-psychotherapy.aspx}\textgreater{}.

Mellor, Felicity. 2009. ``The Politics of Accuracy in Judging Global
Warming Films.'' \emph{Environmental Communication} 3 (2): 134--150.

Meloni, Maurizio. 2011. ``The Cerebral Subject at the Junction of
Naturalism and Antinaturalism.'' In Ortega e Vidal 2011.

\_\_\_\_\_. 2012. ``On the Growing Intellectual Authority of Neuroscience
for Political and Moral Theory: Sketch for a Genealogy.'' In Vander Valk
2012a, 25--49.

\_\_\_\_\_. 2013. ``Biology Without Biologism: Social Theory in a
Postgenomic Age.'' \emph{Sociology}. doi:10.1177/0038038513501944.

\_\_\_\_\_. 2014a. ``How Biology Became Social, and What It Means for
Social Theory.'' \emph{Sociological Review} 62 (3): 593--614.

\_\_\_\_\_. 2014b. ``The Social Brain Meets the Reactive Genome:
Neuroscience, Epigenetics, and the New Social Biology.'' \emph{Frontiers
in Human Neuroscience} 8, art. 309.

Men, Weiwei, Dean Falk, Tao Sun, et al. 2013. ``The Corpus Callosum of
Albert Einstein's Brain: Another Clue to His High Intelligence?''
\emph{Brain}. doi:10.1093/brain/awt252.

Menand, Louis. 2002. ``What Comes Naturally.'' \emph{The New Yorker} (22
Nov.).
\textless{}\emph{https://bit.ly/2MrfTxJ}\textgreater{}.

Menninghaus, Winfried. 2008. \emph{Kunst als ``Beförderung des Lebens'':
Perspektiven transzendentaler und evolutionärer Ästhetik.} Munich: Carl
Friedrich von Siemens Stiftung.

Merzenich, Michael, Mor Nahum, e Thomas M. van Vleet, Orgs. 2013.
\emph{Changing Brains: Applying Brain Plasticity to Advance and Recover
Human Ability.} Amsterdam: Elsevier.

Merzenich, Michael, Thomas M. van Vleet, e Mor Nahum. 2014. ``Brain
Plasticity"-Based Therapeutics.'' \emph{Frontiers in Human Neuroscience}
8, art. 385.

Metzinger, Thomas. 2009. \emph{The Ego Tunnel: The Science of the Mind
and the Myth of the Self.} New York: Perseus.

Meyerding, Jane. 1998. ``Thoughts on Finding Myself Differently
Brained.''\\ \textless{}\emph{https://bit.ly/1bgjr1o}\textgreater{}

\_\_\_\_\_. 2003. ``The Great `Why Label?' Debate.''
\textless{}\emph{https://bit.ly/30ZALkQ}\textgreater{}

Michael, Emily. 2000. \emph{``}Renaissance Theories of Body, Soul, and
Mind\emph{.'' In Psyche and Soma: Physicians and Metaphysicians on the
Mind"-Body Problem from Antiquity to Enlightenment}, Org. John P. Wright
e Paul Potter. Oxford: Clarendon.

Miller, Gavin. 2014. ``Is the Agenda for Global Mental Health a Form of
Cultural Imperialism?'' \emph{Medical Humanities} 40 (2): 131--134.

Miller, Greg. 2016. ``Brain Scans Are Prone to False Positives, Study
Says.'' \emph{Science} 353 (6296): 208--209.

Mills, China. 2014. \emph{Decolonizing Global Mental Health: The
Psychiatrization of the Majority World.} London: Routledge.

Mitchell, Philip B. 2009. ``Winds of Change: Growing Demands for
Transparency in the Relationship Between Doctors and the Pharmaceutical
Industry.'' \emph{Medical Journal of Australia} 191:273--275.

Mlodinow, Leonard. 2012. ``Why People Choose Coke Over Pepsi: How Our
Brains Create Our Consumer Experience.''
\textless{}\emph{https://bit.ly/2nvkXbS}\textgreater{}

Molnár, Zoltán. 2004. ``Thomas Willis (1621--1675), the Founder of
Clinical Neuroscience.'' \emph{Nature Reviews Neuroscience} 5:329--335.

Moncrieff, Joanna. 2008. \emph{The Myth of the Chemical Cure: A Critique
of Psychiatric Drug Treatment.} Houndmills: Palgrave.

Montanini, Daniel, e Cláudio E. M. Banzato. 2012. ``Do estigma da
psicose maníaco"-depressiva ao incentivo ao tratamento do transtorno
bipolar: a evolução da abordagem em dois veículos midiáticos nos últimos
40 anos.'' \emph{Jornal brasileiro de psiquiatria} 61 (2): 84--88.

Moran, Joseph M., e Jamil Zaki. 2013. ``Functional Neuroimaging and
Psychology: What Have You Done for Me Lately?'' \emph{Journal of
Cognitive Neuroscience} 25 (6): 834--842.

Morioka, Masahiro. 2001. ``Reconsidering Brain Death: A Lesson from
Japan's Fifteen Years of Experience.'' \emph{Hastings Center Report} 31
(4): 41--46.

Morrison P. Anthony, Paul Hutton, David Shiers, e Douglas Turkington.
2012. ``Antipsychotics: Is It Time to Introduce Patient Choice?''
\emph{British Journal of Psychiatry} 201:83-84.

Morrison, P. Anthony, Douglas Turkington, Melissa Pyle, et al. 2014.
``Cognitive Therapy for People with Schizophrenia Spectrum Disorder Not
Taking Antipsychotic Medication: A Single"-Blind Randomised Controlled
Trial.'' \emph{The Lancet} 383 (9926): 1395--1403.

Mowaljarlai, David, Patricia Vinnicombe, Graeme K. Ward, e Christopher
Chippindale. 1988. ``Repainting of Images in Australia and the
Maintenance of Aboriginal Culture.'' \emph{Antiquity} 62:690--696.

Mrazek, Alissa. J., Tokiko Harada, e Joan Y. Chiao. 2014. ``Cultural
Neuroscience of Identity Development.'' \emph{In The Oxford Handbook of
Identity Development}, Org. Kate C. McLean e Moin Syed. Oxford: Oxford
University Press.

Munro, Geoffrey D., e Cynthia A. Munro. 2014. ```Soft' Versus `Hard'
Psychological Science: Biased Evaluations of Scientific Evidence That
Threatens or Supports a Strongly Held Political Identity.'' \emph{Basic
and Applied Social Psychology} 36 (6): 533--543.

Muzur, Amir, e Iva Rinčić. 2013. ``Neurocriticism: A Contribution to the
Study of the Etiology, Phenomenology, and Ethics of the Use and Abuse of
the Prefix Neuro-.'' \emph{\versal{JAHR}--European Journal of Bioethics} 4 (7):
545--554.

Nadal, Marcos. 2013. ``The Experience of Art: Insights from
Neuroimaging.'' In Finger, Zaidel, Boller, e Bogousslavsky 2013.

Nadal, Marcos, e Marcus T. Pearce. 2011. ``The Copenhagen Neuroesthetics
Conference: Prospects and Pitfalls for an Emerging Field.'' \emph{Brain
and Cognition} 76:172--183.

Nadesan, Majia H. 2005. \emph{Constructing Autism: Unravelling the
``Truth'' and Understanding the Social.} London: Routledge.

Nalbantian, Suzanne. 2008. ``Neuroesthetics, Neuroscientific Theory, and
Illustration from the Arts'' \emph{Interdisciplinary Science Reviews} 33
(4): 357--368.

Nan, Yun, Thomas R. Knösche, Stefan Zysset, e Angela D. Friederici.
2008. ``Cross"-Cultural Music Phrase Processing: An f\versal{MRI} Study.''
\emph{Human Brain Mapping} 29:312--328.

Nantel"-Vivier, Amélie, e Robert Pihl. 2008. ``Biological Vulnerability
of Depression.'' \emph{In Handbook of Depression in Children and
Adolescents}, Org. John R. Z. Abela e Benjamin L. Hankin, 103--123. New
York: Guilford.

Nelkin, Dorothy, e M. Susan Lindee. 1995. \emph{The \versal{DNA} Mystique: The
Gene as a Cultural Icon.} New York: Freeman.

Nelson, Amy. 2004. ``Declaration from the Autism Community That They Are
a Minority Group.'' (18 Nov.).
\textless{}\emph{https://bit.ly/2Vx2vfM}\textgreater{}

Netherland, Julie. 2011. ```We Haven't Sliced Open Anyone's Brain Yet':
Neuroscience, Embodiment, and the Governance of Addiction.'' In
Pickersgill e Van Keulen 2011.

Ng, Brandon W., James P. Morris, e Shigehiro Oishi. 2013. ``Cultural
Neuroscience: The Current State of Affairs.'' \emph{Psychological
Inquiry} 24:53--57.

Ng, Sik Hung, Shihui Han, Lihua Mao, e Julian C. Lai. 2010. ``Dynamic
Bicultural Brains: f\versal{MRI} Study of Their Flexible Neural Representation of
Self and Significant Others in Response to Culture Primes.'' \emph{Asian
Journal of Social Psychology} 13 (2): 83--91.

Nissenbaum, Stephen. 1980. \emph{Sex, Diet, and Debility in Jacksonian
America: Sylvester Graham and Health Reform.} Westport, Conn.:
Greenwood.

Noë, Alva. 2009. \emph{Out of Our Heads: Why You Are Not Your Brain, and
Other Lessons from the Biology of Consciousness.} New York: Hill and
Wang.

\_\_\_\_\_. 2015. \emph{Strange Tools: Art and Human Nature.} New York:
Hill and Wang.

Northoff, Georg. 2013a. ``Gene, Brains, and Environment---Genetic
Neuro"-imaging of Depression.'' \emph{Current Opinion in Neurobiology}
23:133--142.

\_\_\_\_\_. 2013b. ``What Is Culture? Culture Is Context"-Dependence!''
\emph{Culture and Brain} 1 (2/4): 77--99.

Northoff, Georg, Christine Wiebking, Todd Feinberg, e Jaak Panksepp.
2011. ``The `Resting"-State Hypothesis' of Major Depressive Disorder---A
Translational Subcortical"-Cortical Framework for a System Disorder.''
\emph{Neuroscience and Biobehavioral Reviews} 35 (9): 1929--1945.

Novas, Carlos, e Nikolas Rose. 2000. ``Genetic Risk and the Birth of the
Somatic Individual.'' \emph{Economy and Society} 29:485--513.

Nozick, Robert. 1981. \emph{Philosophical Explanations}. Cambridge,
Mass.: Harvard Univeristy Press.

O'Connor, Cliodhna, e Helene Joffe. 2013. ``How Has Neuroscience
Affected Lay Understandings of Personhood? A Review of the Evidence.''
\emph{Public Understanding of Science} 22 (3): 254--268.

Ochs, Elinor, e Olga Solomon. 2010. ``Autistic Sociality.'' \emph{Ethos}
38:69--92.

Oehler"-Klein, Sigrid. 1990. \emph{Die Schadellehre Franz Joseph Galls in
Literatur und Kritik des 19. Jahrhunderts: Zur Rezeptionsgeschichte
einer medizinisch"-biologisch begründeten Theorie der Physiognomik und
Psychologie.} Stuttgart: Gustav Fischer.

Oliver, Mike. 1990. \emph{The Politics of Disablement}. London:
Macmillan.

Olney, Jennifer. 2006. ``Exercise May Be Key to Keeping Your Brain
Fit.''
\textless{}\emph{http://www.brainhq.com/media/news/exercise-may-be-key-keeping-your-brain-fit}\textgreater{}

Olson, Gary. 2008. ``We Empathize, Therefore We Are: Toward a Moral
Neuropolitics.'' \versal{ZN}et (26 Jul.).
\textless{}\emph{https://bit.ly/2IAEIpG}\textgreater{}

\_\_\_\_\_. 2013. \emph{Empathy} \emph{Imperiled: Capitalism, Culture,
and the Brain.} New York: Springer.

\versal{OMS}. 2001. \emph{The World Health Report 2001---Mental Health: New
Understanding, New Hope.} Geneva: World Health Organization. (trad.
port.: \emph{Relatório Mundial da Saúde---Saúde Mental: Nova Concepção,
Nova esperança.} Lisboa: Direção Geral da Saúde, 2002.
\textless{}\emph{https://bit.ly/2SAZnMS)}\textgreater{}.

\_\_\_\_\_. 2008. \emph{Mental Health Gap Action Programme (mh\versal{GAP}):
Scaling Up Care for Mental, Neurological, and Substance Abuse
Disorders.} Geneva: \versal{WHO}.

\_\_\_\_\_. 2013. ``Mental Health Action Plan 2013--2020.'' Sixty"-Sixth
World Health Assembly. Resolution \emph{\versal{WHA}66/8}.

Onians, John. 2008a. \emph{Neuroarthistory: From Aristotle and Pliny to
Baxandall and Zeki.} New Haven, Conn.: Yale University Press.

\_\_\_\_\_. 2008b. ``Neuro Ways of Seeing {[}Entrevista com Eric
Fernie{]}.'' Tate Etc. 13 (Verão).
\textless{}\emph{https://bit.ly/2OzFuHC}\textgreater{}.

Open Science Collaboration. 2015. ``Estimating the Reproducibility of
Psychological Science.'' \emph{Science} 349 (6251): aac4716.
doi:10.1126/science.aac4716.

Orsini, Michael. 2009. ``Contesting the Autistic Subject: Biological
Citizenship and the Autism/Autistic Movement.'' \emph{In Critical
Interventions in the Ethics of Health Care}, Org. Stuart Murray e Dave
Holmes. London: Ashgate.

\_\_\_\_\_. 2012. ``Autism, Neurodiversity, and the Welfare State: The
Challenges of Accommodating Neurological Difference.'' \emph{Canadian
Journal of Political Science} 45:805--882.

Ortega, Francisco. 2011. ``Toward a Genealogy of Neuroacesis.'' In
Ortega e Vidal 2011.

\_\_\_\_\_. 2014. \emph{Corporeality, Medical Technologies, and
Contemporary Culture.} New York: Routledge.

Ortega, Francisco, e Fernando Vidal, Orgs. 2011. \emph{Neurocultures:
Glimpses Into an Expanding Universe.} Berlin: Peter Lang.

Ortega, Francisco, Rafaela Zorzanelli, Lilian Kozslowski Meierhoffer, et
al. 2013. ``A Construção do Diagnóstico do Autismo em uma Rede Social
Virtual Brasileira.'' \emph{Interface---Comunicação, Saúde, Educação}
17:119--132.

Osteen, Mark, Org. 2008. \emph{Autism and Representation.} New York:
Routledge.

Owen, Adrian M., Adam Hampshire, Jessica A. Grahn, et al. 2010.
``Putting Brain Training to the Test.'' \emph{Nature} 465 (7299):
775--778.

Padden, Carol, e Tom Humphries. 2006\emph{. Inside Deaf Culture}.
Cambridge, Mass.: Harvard University Press.

Painter, Nell I. 2010. \emph{The History of White People.} New York:
Norton.

Pardo, Michael S., e Dennis Patterson. 2011. ``Minds, Brains, and
Norms.'' \emph{Neuroethics} 4:179--190.

Parlette, Snowdon. 1997. \emph{The Brain Workout Book}. New York: M.
Evans e Co.

Patel, Vikram. 2012. ``Global Mental Health: From Science to Action.''
\emph{Harvard Review of Psychiatry} 20 (1): 6--12.

Patel, Vikram, e Mark Winston. 1994. ```Universality of Mental Illness'
Revisited: Assumptions, Artefacts, and New Directions.'' \emph{British
Journal of Psychiatry} 165:437--440.

Patel, Vikram, Helen A. Weiss, Neerja Chowdhary, et al. 2011. ``Lay
Health Worker"-Led Intervention for Depressive and Anxiety Disorders in
India: Impact on Clinical and Disability Outcomes Over 12 Months.''
\emph{British Journal of Psychiatry} 199:459--466.

Patel, Vikram, Harry Minas, Alex Cohen, e Martin J. Prince, Orgs. 2014.
\emph{Global Mental Health: Principles and Practice.} New York: Oxford
University Press.

Paterniti, Michael. 2000. Driving Mr. Albert: \emph{A Trip Across
America with Einstein's Brain.} New York: Dial (trad. port.:
\emph{Conduzindo o Sr. Albert}. Sao Paulo: Companhia das Letras, 2003).

Pedersen, David Budtz. 2011. ``Revisiting the Neuro"-Turn in the
Humanities and Natural Sciences.'' \emph{Pensamiento} 67 (254):
767--786.

Pepperell, Robert. 2011. ``Connecting Art and the Brain: An Artist's
Perspective on Visual Indeterminacy.'' \emph{Frontiers in Human
Neuroscience} 5, art. 84: 1--12.

Peters, June A., Luba Djurdjinovic, e Diane Baker. 1999. ``The Genetic
Self: The Human Genome Project, Genetic Counseling, and Family
Therapy.'' \emph{Families, Systems, and Health} 17 (1): 5--25.

Pethes, Nicolas. 2005. ``Terminal Men, Biotechnological Experimentation,
and the Reshaping of `the Human' in Medical Thrillers.'' \emph{New
Literary History} 36 (2): 161--185.

Petryna, Adriana, Andrew Lakoff, e Arthur Kleinman, Orgs. 2006.
\emph{Global Pharmaceuticals: Ethics, Markets, Practices.} Durham, N.C.:
Duke University Press.

Petryna, Adriana, e Arthur Kleinman. 2006. ``The Pharmaceutical Nexus.''
In Petryna, Lakoff, e Kleinman 2006.

Phelan, C. Jo. 2005. ``Geneticization of Deviant Behavior and
Consequences for Stigma: The Case of Mental Illness.'' \emph{Journal of
Health and Social Behavior} 46 (4): 307--322.

Phillips, Kristopher G., Alan Beretta, e Harry A. Whitaker. 2015. ``Mind
and Brain: Toward an Understanding of Dualism\emph{.'' In Brain, Mind,
and Consciousness in the History of Neuroscience}, Org. C. U. M. Smith e
Harry Whitaker, 355--369. Dordrecht: Springer.

Pickering, Andrew. 2011. \emph{The Cybernetic Brain: Sketches of Another
Future.} Chicago: University of Chicago Press.

Pickersgill, Martyn, Sarah Cunningham"-Burley, e Paul Martin. 2011.
``Constituting Neurologic Subjects: Neuroscience, Subjectivity, and the
Mundane Significance of the Brain.'' \emph{Subjectivity} 4:346--365.

Pickersgill, Martyn, e Ira Van Keulen, Orgs. 2012. \emph{Sociological
Reflections on the Neurosciences.} Bingley: Emerald.

Pickersgill, Martyn, Paul Martin, e Sarah Cunningham"-Burley. 2015. ``The
Changing Brain: Neuroscience and the Enduring Import of Everyday
Experience\emph{.'' Public Understanding of Science} 24:878--892.

Pitts"-Taylor, Victoria. 2010. ``The Plastic Brain: Neoliberalism and the
Neuronal Self.'' \emph{Health} 14 (6): 635--652.

Podgorny, Irina. 2005. ``La derrota del genio. Cráneos y cérebros en la
filogenia argentina.'' \emph{Saber y tiempo. Revista de historia de la
ciencia} 5 (20): 63--106.

Poldrack, Russell A. 2008. ``The Role of f\versal{MRI} in Cognitive
Neuro"-science: Where Do We Stand?'' \emph{Current Opinion in
Neurobiology} 2:223--227.

Posner, Jonathan, Virginia Rauh, Allison Gruber, et al. 2013.
``Dissociable Attentional and Affective Circuits in Medication"-Naïve
Children with Attention"-Deficit/Hyperactivity Disorder.''
\emph{Psychiatry Research: Neuroimaging} 213:24--30.

Posner, Jonathan, Christine Park, e Zhishun Wang. 2014. ``Connecting the
Dots: A Review of Resting Connectivity \versal{MRI} Studies in
Attention"-Deficit/Hyperactivity Disorder.'' \emph{Neuropsychology
Review} 24:3--15.

Powers, Richard. 1996 {[}1995{]}. \emph{Galatea} 2.2. New York: Harper
Perennial.

\_\_\_\_\_. 2006. \emph{The Echomaker}. New York: Farrar, Straus e
Giroux. (trad. port.: \emph{Ecos da Mente}. Tradução de Marilene
Tombini. São Paulo: Record, 2013).

\_\_\_\_\_. 2007. ``The Brain Is the Ultimate Storytelling Machine, and
Consciousness is the Ultimate Story. Interview with Richard Powers.''
\emph{Believer} (Fevereiro).
\textless{}\emph{https://bit.ly/2ICIzTe}\textgreater{}

Presidential Commision 2015. Gray Matters: \emph{Topics at the
Intersection of Neuroscience, Ethics, and Society.} Vol. 2. Washington,
D.C.: Presidential Commission for the Study of Bioethical Issues.

Press Release. 2011. ``Mindfulness Meditation Training Changes Brain
Structure in 8 Weeks.''
\textless{}\emph{http://www.massgeneral.org/about/pressrelease.aspx?id​=1329}\textgreater{}.
{[}Hölzel et al. 2011{]}

Prévost, Bertrand. 2003. ``Pouvoir ou efficacité symbolique des
images.'' \emph{L'Homme. Revue Française d'Anthropologie} 165:275--282.

Price, Joseph L., e Wayne C. Drevets. 2010. ``Neurocircuitry of Mood
Disorders.'' \emph{Neuropsychopharmacology Reviews} 35:192--216.

Prince"-Hughes, Dawn. 2004. \emph{Songs of the Gorilla Nation: My Journey
Through Autism.} New York: Harmony.

Prince, Dawn Eddings. 2010. ``An Exceptional Path: An Ethnographic
Narrative Reflecting on Autistic Parenthood from Evolutionary, Cultural,
and Spiritual Perspectives.'' \emph{Ethos} 38:56--68.

Prince, Martin, Vikram Patel, Shekhar Saxena, et al. 2007. ``No Health
Without Mental Health.'' \emph{The Lancet} 370:859--877.

Prince, Martin, Atif Rahman, Rosie Mayston, e Benedict Weobong. 2014.
``Mental Health and the Global Health and Development Agendas.'' In
Patel, Minas, Cohen, e Prince 2014.

Protevi, John. 2009. \emph{Political Affect: Connecting the Social and
the Somatic.} Minneapolis: University of Minnesota Press.

Puccetti, Roland. 1969. ``Brain Transplantation and Personal Identity.''
\emph{Analysis} 29:65--77.

\_\_\_\_\_. 1973. ``Brain Bisection and Personal Identity.''
\emph{British Journal for the Philosophy of Science} 24:339--355.

Pugliese, Joseph. 2010. \emph{Biometrics: Bodies, Technologies,
Biopolitics.} New York: Routledge.

Putnam, Hilary. 1981. \emph{Reason, Truth, and History.} Cambridge,
Mass.: Harvard University Press (trad. port.: \emph{Razão},
\emph{verdade} e \emph{história}. Lisboa: Dom Quixote, 1992).

Rachul, Christen, e Amy Zarzeczny. 2012. ``The Rise of
Neuroskepticism.'' \emph{International Journal of Law and Psychiatry}
35:77--81.

Racine, Eric. 2010. \emph{Pragmatic Neuroethics: Improving Treatment and
Understanding of the Mind"-Brain.} Cambridge, Mass.: \versal{MIT} Press.

Racine, Eric, Ofek Bar"-Ilan, e Judy Illes.2005. ``f\versal{MRI} in the Public
Eye\emph{.'' Nature Reviews Neuroscience} 6:159--164.

Radstone, Susannah. 2010. ``Cinema and Memory.'' \emph{In Memory:
Histories, Theories, Debates}, Org. Susannah Radstone e Bill Schwartz,
325--342. New York: Fordham University Press.

Rafter, Nicole. 2008. \emph{The Criminal Brain: Understanding Biological
Theories of Crime.} New York: New York University Press.

Raichle, Marcus E., Ann M. MacLeod, Abraham Z. Snyder, et al. 2001. ``A
Default Mode of Brain Function.'' \emph{\versal{PNAS}} 98 (2): 676--682.

Raichle, Marcus E., e Abraham Z. Snyder. 2007. ``A Default Mode of Brain
Function: A Brief History of an Evolving Idea.'' \emph{NeuroImage}
37:1083--1090.

Ramachandran, Vilayanur Subramanian, e William Hirstein. 1999. ``The
Science of Art: A Neurological Theory of Aesthetic Experience.''
\emph{Journal of Consciousness Studies} 6 (6/7): 15--51.

Randall, Kevin. 2015. ``Neuropolitics, Where Campaigns Try to Read Your
Mind.'' \emph{New York Times} (3 Nov.).
\textless{}\emph{https://nyti.ms/2ouP17V}\textgreater{}

Rapp, Rayna. 2011. ``A Child Surrounds This Brain: The Future of
Neuro"-logical Difference According to Scientists, Parents, and Diagnosed
Young Adults.'' In Pickersgill e Van Keulen 2011.

Rawlings, Charlie E., e Eugene Rossitch Jr. 1994. ``Franz Josef Gall and
His Contribution to Neuroanatomy with Emphasis on the Brain Stem.''
\emph{Surgical Neurology} 42:272--275.

Ray, Rebecca D., Amy L. Shelton, Nick G. Hollon, et al. 2010.
``Interdependent Self"-Construal and Neural Representations of Self and
Mother.'' \emph{Social Cognitive and Affective Neuroscience} 5:318--323.

Ray, Wayne A., Cecilia P. Chung, Katherine T. Murray, et al. 2009.
``Atypical Antipsychotic Drugs and the Risk of Sudden Cardiac Death.''
\emph{New England Journal of Medicine} 360:225--235.

Read, John. 2005. ``The Bio"-bio"-bio Model of Madness.'' \emph{The
Psychologists} 18 (10): 596--597.

Read, John, Richard Bentall, e Roar Fosse. 2009. ``Time to Abandon the
Bio"-bio"-bio Model of Psychosis: Exploring the Epigenetic and
Psychological Mechanisms by Which Adverse Life Events Lead to Psychotic
Symptoms.'' \emph{Epidemiologia e psichiatria sociale} 18 (4): 299--310.

Read, John, e Niki Harré. 2001. ``The Role of Biological and Genetic
Causal Beliefs in the Stigmatization of `Mental Patients.'''
\emph{Journal of Mental Health} 10 (2): 223--235.

Redies, Christoph. 2007. ``A Universal Model of Aesthetic Perception
Based on the Sensory Coding of Natural Stimuli.'' \emph{Spatial Vision}
21 (1/2): 97--117.

Redies, Christoph, Jan Hänisch, Marko Blickhan, e Joachim Denzler. 2007.
``Artists Portray Human Faces with the Fourier Statistics of Complex
Natural Scenes.'' \emph{Network: Computation in Neural Systems} 18 (3):
235--248.

Redies, Christoph, Jens Hasenstein, e Joachim Denzler. 2007.
``Fractal"-like Image Statistics in Visual Art: Similarity to Natural
Scenes.'' \emph{Spatial Vision} 21 (1/2): 137--148.

Redwood, Daniel. 2007. ``Meditation, Positive Emotions, and Brain
Science: Interview with Richard Davidson Ph.D.''
\textless{}\emph{https://bit.ly/322irsM}\textgreater{}

Rees, Tobias. 2010. ``Being Neurologically Human Today: Life and Science
and Adult Cerebral Plasticity (an Ethical Analysis).'' \emph{American
Ethnologist} 37 (1): 150--166.

\_\_\_\_\_. 2011. ``So Plastic a Brain: On Philosophy, Fieldwork in
Philosophy, and the Rise of Adult Cerebral Plasticity.''
\emph{BioSocieties} 6 (2): 263--267.

Regalado, Antonio. 2015. ``Why America's Top Mental Health Researcher
Joined Alphabet {[}Interview with Thomas Insel{]}.'' \emph{\versal{MIT}
Technology Review} (21 Setembro).
\textless{}\emph{https://bit.ly/1FpLI3t}\textgreater{}

Reichle, Ingeborg. 2009. \emph{Art in the Age of Technoscience: Genetic
Engineering, Robotics, and Artificial Life in Contemporary Art.}
Tradução de Gloria Custance. New York: Springer.

Reid, Ian C. 2013. ``Are Antidepressants Overprescribed? No.''
\emph{British Medical Journal} 346. doi:10.1136/bmj.f190.

Renard, Maurice. 1921. \emph{Les mains d'Orlac}. Paris: Nilsson.

Rengachary, Setti S., Andrew Xavier, Sunil Manjila, et al. 2008. ``The
Legendary Contributions of Thomas Willis (1621--1675): The Arterial
Circle and Beyond.'' \emph{Journal of Neurosurgery} 109:765--775.

Renneville, Marc. 2000. \emph{Le langage des crânes. Une histoire de la
phrénologie}. Paris: Les Empêcheurs de tourner en rond.

Richards, Graham. 2002. ``The Psychology of Psychology: A Historically
Grounded Sketch.'' \emph{Theory and Psychology} 12:7--36.

Richardson, Alan. 2004. ``Studies in Literature and Cognition: A Field
Map.'' \emph{In The Work of Fiction: Cognition, Culture, and
Complexity}, Org. Alan Richardson e Ellen Spolsky, 1--29. Aldershot:
Ashgate.

Rios, Clarice, e Barbara C. Andrada. 2015. ``The Changing Face of Autism
in Brazil.'' \emph{Culture, Medicine, and Psychiatry} 39 (2): 213--234.

Rizzolatti, Giacomo, e Laila Craighero. 2004. ``The Mirror"-Neuron
System.'' \emph{Annual Review of Neuroscience} 27:169--192.

Rizzolatti, Giacomo, e Maddalena Fabbri"-Destro. 2010. ``Mirror Neurons:
From Discovery to Autism.'' \emph{Experimental Brain Research}
200:223--237.

Rizzolatti, Giacomo, e Corrado Sinigaglia. 2010. ``The Functional Role
of the Parietofrontal Mirror Circuit: Interpretations and
Misinterpretations.'' \emph{Nature Reviews Neuroscience} 11 (4):
264--274.

Rocca, Julius. 2003. \emph{Galen on the Brain: Anatomical Knowledge and
Physiological Speculation in the Second Century \versal{AD}}. Leiden: Brill.

Rodriguez, Paul. 2006. ``Talking Brains: A Cognitive Semantic Analysis
of an Emerging Folk Neuropsychology.'' \emph{Public Understanding of
Science} 15 (3): 301--330.

Roepstorff, Andreas. 2011. ``Culture: A Site of Relativist Energy in the
Cognitive Sciences.'' \emph{Common Knowledge} 17:37--41.

Roepstorff, Andreas, e Chris Frith. 2012. ``Neuroanthropology or Simply
Anthropology? Going Experimental as Method, as Object of Study, and as
Research Aesthetic.'' \emph{Anthropological Theory} 12 (1): 101--111.

Roepstorff, Andreas, e Kai Vogeley. 2009. ``Contextualising Culture and
Social Cognition.'' \emph{Trends in Cognitive Science} 13:511--516.

Roepstorff, Andreas, Jörg Niewöhner, e Stefan Beck. 2010. ``Enculturing
Brains Through Patterned Practices.'' \emph{Neural Networks}
23:1051--1059.

Rose, Nikolas. 1990. \emph{Governing the Soul: The Shaping of the
Private Self}. London: Routledge. (trad. port.: ``Governando a Alma: a
Formação do Eu Privado''. In Silva, Tadeu, Org. \emph{Liberdades
Reguladas}. Petrópolis: Vozes, p. 30-45).

\_\_\_\_\_. 1996. \emph{Inventing Our Selves: Psychology, Power, and
Personhood}. New York: Cambridge University Press. (trad. port.:
\emph{Inventando nossos Selfs: Psicologia, Poder e Subjetividade}.
Petrópolis: Vozes, 2011).

\_\_\_\_\_. 2003. ``The Neurochemical Self and Its Anomalies.'' \emph{In
Risk and Morality}, Org. Richard Ericson e Aaron Doyle. Toronto:
University of Toronto Press.

\_\_\_\_\_. 2004. ``Becoming Neurochemical Selves.'' \emph{In
Biotechnology: Between Commerce and Civil Society}, Org. Nico Stehr. New
Brunswick, N.J.: Transaction.

\_\_\_\_\_. 2007. \emph{The Politics of Life Itself: Biomedicine, Power,
and Subjectivity in the Twenty"-First Century.} Princeton, N.J.:
Princeton University Press. (trad. port.: \emph{A Política da Própria
Vida: Biomedicina, Poder e Subjetividade no Século \versal{XXI}}. São Paulo:
Paulus, 2013.)

\_\_\_\_\_. 2013a. ``The Human Sciences in a Biological Age.''
\emph{Theory, Culture, and Society} 30 (1): 3--34.

\_\_\_\_\_. 2013b. ``What Is Diagnosis For?'' Conference paper, ``\versal{DSM}"-5
and the Future of Diagnosis.''
\textless{}\emph{https://bit.ly/32dZvqQ}\textgreater{}

Rose, Nikolas, e Joelle M. Abi"-Rached. 2013. \emph{Neuro: The New Brain
Sciences and the Management of the Mind.} Princeton, N.J.: Princeton
University Press.

\_\_\_\_\_. 2014. ``Governing Through the Brain: Neuropolitics,
Neuroscience, and Subjectivity.'' \emph{Cambridge Anthropology} 32 (1):
3--23.

Rosen, Bruce R., e Robert L. Savoy. 2012. ``f\versal{MRI} at 20: Has It Changed
the World?'' \emph{NeuroImage} 62:1316--1324.

Rosenbaum, Bent, Susanne Harder, Per Knudsen, et al. 2012. ``Supportive
Psychodynamic Psychotherapy Versus Treatment as Usual for First"-Episode
Psychosis: Two"-Year Outcome.'' \emph{Psychiatry: Interpersonal and
Biological Processes} 75 (4): 331--341.

Roskies, Adina L. 2002. ``Neuroethics for the New Millenium.'' Neuron
35:21--23.

\_\_\_\_\_. 2007. ``Are Neuroimages Like Photographs of the Brain?''
\emph{Philosophy of Science} 74:860--872.

\_\_\_\_\_. 2009. ``Brain"-Mind and Structure"-Function Relationships: A
Methodological Response to Coltheart.'' \emph{Philosophy of Science}
76:927--939.

\_\_\_\_\_. 2010. ``Saving Subtraction: A Reply to Van Orden and Paap.''
British Journal of the \emph{Philosophy of Science} 61:635--665.

Roth, Marco. 2009. ``The Rise of the Neuronovel.'' N+1 8 (19 Out.).
\textless{}\emph{https://bit.ly/2OwLnVJ}\textgreater{}

Roth, Michael S. 1981. ``Foucault's `History of the Present.'''
\emph{History and Theory} 20 (1): 32--46.

Rousseau, George Sebastian. 2007. ```Brainomania': Brain, Mind, and Soul
in the Long Eighteenth Century.'' \emph{British Journal for
Eighteenth"-Century Studies} 30:161--191.

Rowland, Margaret. 2015. ``Angry and Mad: A Critical Examination of
Identity Politics, Neurodiversity, and the Mad Pride Movement.''
\emph{Journal of Ethics in Mental Health} 1:1--3.

Rozenblit, Leonid, e Frank Keil. 2002. ``The Misunderstood Limits of
Folk Science: An Illusion of Explanatory Depth.'' \emph{Cognitive
Science} 26:521--562.

Rubin, Sue. 2005. ``Acceptance Versus Cure.''
\textless{}\emph{https://cnn.it/2ICJ3Zy}\textgreater{}

Rugg, Michael D., e Sharon L. Thompson"-Schill. 2013. ``Moving Forward
with f\versal{MRI} Data.'' \emph{Perspectives on Psychological Science} 8 (1):
84--87.

Rusconi, Elena, e Timothy Mitchener"-Nissen. 2014. ``The Role of
Expectations, Hype, and Ethics in Neuroimaging and Neuromodulation
Futures.'' \emph{Frontiers in Systems Neuroscience} 8, art. 214.

Sacher, Julia, Jane Neumann, Tillmann Fünfstück, et al. 2012. ``Mapping
the Depressed Brain: A Meta"-analysis of Structural and Functional
Alterations in Major Depressive Disorder.'' \emph{Journal of Affective
Disorders} 140:142--148.

Sacks, Oliver. 1985. \emph{The Man Who Mistook His Wife for a Hat and
Other Clinical Tales.} New York: Touchstone. (trad. port.: \emph{O Homem
que Confundiu sua Mulher com um Chapéu.} Tradução de Laura Teixeira
Motta. São Paulo: Companhia das Letras, 1997).

\_\_\_\_\_. 1995. \emph{An Anthropologist on Mars.} New York: Vintage.
(trad. port.: \emph{Um Antropólogo em Marte}. Tradução de Bernardo
Carvalho. São Paulo: Companhia de Bolso, 2006).

\_\_\_\_\_. 2013. ``Speak, Memory.'' \emph{The New York Review of Books}
(21 Fev.).
\textless{}\emph{https://bit.ly/1A6KkOE}\textgreater{}

Sahlins, Marshall. 2000. ``Sentimental Pessimism and Ethnographic
Experience; Or, Why Culture Is Not a Disappearing `Object.''' \emph{In
Biographies of Scientific Objects}, Org. Lorraine Daston. Chicago:
University of Chicago Press. (trad. port.: Sahlins, Marshall. ``O
`pessimismo sentimental' e a experiência etnográfica: Por que a cultura
não é um `objeto' em via de extinção.'' \emph{Mana}, 3(1): 41-73, abr.
1997(parte I) e \emph{Mana}, 3(2): 103-150 (parte \versal{II}).

Sass, Hans"-Martin. 1989. ``Brain Life and Brain Death: A Proposal for a
Normative Agreement.'' \emph{Journal of Medicine and Philosophy}
14:45--59.

Sbriscia"-Fioretti, Beatrice, Cristina Berchio, David Freedberg, et al.
2013. ``\versal{ERP} Modulation During Observation of Abstract Paintings by Franz
Kline.'' \versal{PL}o\versal{S ONE} 8 (10): e75241. doi:10.1371/journal.pone.0075241.

Schaeffer, Jean"-Marie. 1997. ``La relation esthétique comme fait
anthropologique.'' \emph{Critique} 53:691--708.

\_\_\_\_\_. 2009. \emph{Adieu à l'esthétique.} Paris: \versal{PUF}.

\_\_\_\_\_. 2010. \emph{Théorie des signaux coûteux, esthétique et art.}
Trois"-Rivières: Tangence.

Scharinger, Christian, Ulrich Rabl, Lukas Pezawas, e Siegfried Kasper.
2011. ``The Genetic Blueprint of Major Depressive Disorder:
Contributions of Imaging Genetics Studies.'' \emph{World Journal of
Biological Psychiatry} 12:474--488.

Scheper"-Hugues, Nancy. 1984. ``The Margaret Mead Controversy: Culture,
Biology, and Anthropological Inquiry.'' \emph{Human Organization} 43
(1): 85--93.

Schick, Ari. 2005. ``Neuro Exceptionalism?'' \emph{American Journal of
Bioethics} 5 (2): 36--38.

Schlaepfer, Thomas E., Bettina H. Bewernick, Sarah Kayser, et al. 2014.
``Deep Brain Stimulation of the Human Reward System for Major
Depression---Rationale, Outcomes, and Outlook.''
\emph{Neuropsychopharmacology} 39:1303--1314.

Schleim, Stephan, e Jonathan P. Roiser. 2009. ``f\versal{MRI} in Translation: The
Challenges Facing Real"-World Applications.'' \emph{Frontiers in Human
Neuroscience} 3, art. 63: 1--7.

Schnittker, Jason. 2008. ``An Uncertain Revolution: Why the Rise of a
Genetic Model of Mental Illness Has Not Increased Tolerance.''
\emph{Social Science and Medicine} 67 (9): 1370--1381.

Schreyach, Michael. 2007. ```I Am Nature': Science and Jackson
Pollock.'' \emph{Apollo} 7:35--43.

Schwartz, Jeffrey M., e Sharon Begley. 2002. \emph{The Mind and the
Brain: Neuroplasticity and the Power of Mental Force.} New York:
HarperCollins.

Senn, Bryan, e John Johnson. 1992. \emph{Fantastic Cinema Subject Guide:
A Topical Index to 2,500 Horror, Science Fiction, and Fantasy Films}.
Jefferson, N.C.: McFarland.

Shakespeare, Tom. 2006. \emph{Disability Rights and Wrongs}. Abingdon:
Routledge.

Shapin, Steven. 2008. \emph{The Scientific Life: A Moral History of a
Late Modern Vocation.} Chicago: University of Chicago Press.

Shapiro, Joseph P. 1993. \emph{No Pity: People with Disabilities Forging
a New Civil Rights Movement}. New York: Random House.

\_\_\_\_\_. 2006. ``Autism Movement Seeks Acceptance, Not Cures.''
\textless{}\emph{http://www.npr.org/templates/story/story.php?storyId​=5488463}\textgreater{}

Shedler, Jonathan. 2010. ``The Efficacy of Psychodynamic
Psychotherapy.'' \emph{American Psychologist} 65 (2): 98--109.

Sherwood, Katherine. 2009. \emph{Golgi's Door} {[}catálogo da
exibiçao{]}. Washington, D.C.: National Academy of Sciences\emph{.}

Shoemaker, Sidney. 1963. \emph{Self"-Knowledge and Self"-Identity}.
Ithaca, N.Y.: Cornell University Press.

Shorter, Edward. 2013. \emph{How Everyone Became Depressed: The Rise and
Fall of the Nervous Breakdown.} New York: Oxford University Press.

Shuttleworth, Sally. 1996. \emph{Charlotte Brontë and Victorian
Psychology}. New York: Cambridge University Press.

Shweder, Richard A. 1991. \emph{Thinking Through Cultures: Expeditions
in Cultural Psychology.} Cambridge, Mass.: Harvard University Press.

\_\_\_\_\_. 2001. ``Culture: Contemporary Views.'' \emph{In International
Encyclopedia of the Social and Behavioral Sciences}, Org. Neil J.
Smelser e Paul B. Baltes. Oxford: Elsevier.

Siegle, Greg J., Wesley K. Thompson, Amanda Collier, et al. 2012.
``Toward Clinically Useful Neuroimaging in Depression Treatment.''
\emph{Archives of General Psychiatry} 69 (9): 913--924.

Silberman, Steve. 2015. \emph{NeuroTribes: The Legacy of Autism and the
Future of Neurodiversity.} New York: Avery.

Silverman, Chloe. 2008a. ``Brains, Pedigrees, and Promises: Lessons from
the Politics of Autism Genetics.'' \emph{In Gibbon e Novas} 2008b,
38--55.

\_\_\_\_\_. 2008b. ``Fieldwork on Another Planet: Social Science
Perspectives on the Autism Spectrum.'' \emph{BioSocieties} 3 (3):
325--341.

\_\_\_\_\_. 2012. \emph{Understanding Autism: Parents, Doctors, and the
History of a Disorder}. Princeton, N.J.: Princeton University Press.

Simon, Herbert. 1994. ``Literary Criticism: A Cognitive Approach.''
\emph{Stanford Humanities Review} 4 (1).

Simpson, Donald. 2005. ``Phrenology and the Neurosciences: Contributions
of F. J. Gall and J. G. Spurzheim.'' \emph{\versal{ANZ} Journal of Surgery}
75:475--482.

Sinclair, Jim. 1993. ``Don't Mourn for Us.'' \emph{Voice} 1 (3).
\textless{}\emph{https://bit.ly/1TOWAKz}\textgreater{}.

\_\_\_\_\_. 1999. ``Why I Dislike ``Person First'' Language.''
\textless{}\emph{https://bit.ly/1g4D6MT}\textgreater{}

\_\_\_\_\_. 2005. ``Autism Network International: The Development of a
Community and Its Culture.''
\textless{}\emph{https://bit.ly/1CWiLUU}\textgreater{}

Singel, Ryan. 2003. ``He Thinks, Therefore He Sells.''
\textless{}\emph{https://bit.ly/314UcbT}\textgreater{}

Singer, Judy. 1999. ``Why Can't You Be Normal for Once in Your Life?
From a `Problem with No Name' to the Emergence of a New Category of
Difference.'' In Corker e French 1999, 59--67.

\_\_\_\_\_. 2007. ``Light and Dark. Correcting the Balance.''
\textless{}\emph{https://archive.is/pu1O6}\textgreater{}

Singh, Ilina. 2013. ``Brain Talk: Power and Negotiation in Children's
Discourse About Self, Brain, and Behavior.'' \emph{Sociology of Health
and Illness} 35 (6): 813--827.

Singh, Ilina, e Nikolas Rose. 2006. ``Neuro"-Forum: An Introduction.''
\emph{BioSocieties} 1:97--102.

\_\_\_\_\_. 2009. ``Biomarkers in Psychiatry: Promises and Perils in the
Real World.'' \emph{Nature} 460 (7252): 202--207.

Singh, Krish D. 2012. ``Which `Neural Activity' Do You Mean? f\versal{MRI}, \versal{MEG},
Oscillations, and Neurotransmitters.'' \emph{NeuroImage} 62:1121--1130.

Siodmak, Curt. 1992 {[}1942{]}. \emph{Donovan's Brain}. New York:
Leisure Books. (trad. port.: \emph{O Cérebro Assassino}. Rio de Janeiro:
José Olympio, 1969).

Skolnick Weisberg, Deena, Frank C. Keil, Joshua Goodstein, et al. 2008.
``The Seductive Allure of Neuroscience Explanations.'' \emph{Journal of
Cognitive Neuroscience} 20:470--477.

Skov, Martin. 2006. ``A Short Bibliographic Guide to the Emerging Field
of Bioaesthetics.''
\textless{}\emph{https://bit.ly/2nvMUAj}\textgreater{}

Skov, Martin, e Oshin Vartanian, Orgs. 2009a. \emph{Neuroaesthetics}.
Amityville, N.Y.: Baywood.

\_\_\_\_\_. 2009b. ``Introduction: What Is Neuroaesthetics?'' In Skov e
Vartanian 2009a.

Slaby, Jan, Philipp Haueis, e Suparna Choudhury. 2012. ``Neuroscience as
Applied Hermeneutics. Towards a Critical Neuroscience of Political
Theory.'' In Vander Valk 2012a, 50--73.

Smith, Gwenn S., Org. 2015. \emph{Handbook of Depression in Alzheimer's
Disease.} Amsterdam: \versal{IOS}.

Smith, Jennifer. 2009. ``Building a Better Brain.'' \emph{Isthmus} (27
Jul.). \textless{}\emph{http://www.isthmus.com/isthmus/article.php?article​=25405}\textgreater{}

Smith, Martin. 2012. ``Brain Death: Time for an International
Consensus.'' \emph{British Journal of Anaesthesia} 108 (S1): i6--i9.

Smith, Roger. 1997. \emph{The Fontana History of the Human Sciences.}
London: Fontana.

\_\_\_\_\_. 2007. \emph{Being Human: Historical Knowledge and the
Creation of Human Nature.} Manchester: Manchester University Press.

Snodgrass, Jeffrey G. 2014. ``Ethnography of Online Cultures.'' \emph{In
Handbook of Methods in Cultural Anthropology}, Org. H. Russell Bernard e
Clarence C. Gravlee. Lanham, Md.: Rowman \& Littlefield.

Sokolow, Jayme A. 1983. \emph{Eros and Modernization: Sylvester Graham,
Health Reform, and the Origins of Victorian Sexuality in America.}
London: Associated Universities Press.

Solomon, Andrew. 2008. ``The Autism Rights Movement.'' \emph{New York
Magazine} (25 Maio). \textless{}\emph{http://nymag.com/news/features/47225}\textgreater{}

Solso, Robert L. 2000. ``The Cognitive Neuroscience of Art.''
\emph{Journal of Consciousness Studies} 7--8/9:75--81.

\_\_\_\_\_. 2001. ``Brain Activities in a Skilled Versus a Novice Artist:
An f\versal{MRI} Study.'' \emph{Leonardo} 34 (1): 31--34.

Solymosi, Tibor, e John R. Shook, Orgs. 2014. \emph{Neuroscience,
Neurophilosophy, and Pragmatism: Brains at Work with the World.} New
York: Palgrave Macmillan.

Spence, Des. 2013. ``Are Antidepressants Overprescribed? Yes.''
\emph{British Medical Journal} 346 (7907): 16.

Spence, Donald P. 1984. Narrative \emph{Truth and Historical Truth:
Meaning and Interpretation in Psychoanalysis.} New York: Norton.

Spiers, Hugo J., e Daniel Bendor. 2014. ``Enhance, Delete, Incept:
Manipulating Hippocampus"-Dependent Memories.'' \emph{Brain Research
Bulletin} 105:2--7.

Spolsky, Ellen. 2002. ``Darwin and Derrida: Cognitive Literary Theory as
a Species of Post"-Structuralism.'' \emph{Poetics Today} 23:43--62.

Spotts, Dane, e Nancy Atkins. 1999. \emph{Super Brain Power. 28 Minutes
to a Supercharged Brain}. Seattle: LifeQuest.

Starr, Gabrielle. 2012. ``Evolved Reading and the Science(s) of Literary
Study: A Response to Jonathan Kramnick.'' \emph{Critical Inquiry}
38:418--425.

Stein, Dan J., Yanling He, Anthony Phillips, et al. 2015. ``Global
Mental Health and Neuroscience: Potential Synergies.'' \emph{Lancet
Psychiatry} 2:178--185.

Steinberg, Laurence. 2008. ``A Social Neuroscience Perspective on
Adolescent Risk"-Taking.'' \emph{Developmental Review} 28:78--106.

Sterling, Bruce, Org. 1990 {[}1986{]}. \emph{Mirrorshades: The Cyberpunk
Anthology.} Glasgow: Paladin GraftonBooks.

Stern, Madeleine Bettina. 1971. \emph{Heads and Headlines: The
Phrenological Fowlers.} Norman: University of Oklahoma Press.

Stiles, Anne. 2006a. ``Robert Louis Stevenson's Jekyll and Hyde and the
Double Brain.'' \emph{Studies in English Literature}, 1500--1900 46 (4):
879--900.

\_\_\_\_\_. 2006b. ``Cerebral Automatism, the Brain, and the Soul in Bram
Stoker's Dracula.'' \emph{Journal of the History of the Neurosciences}
15 (2): 131--152.

\_\_\_\_\_, Org. 2007. \emph{Neurology and Literature}, 1860--1920. New
York: Palgrave Macmillan.

Stollfuß, Sven. 2014. ``The Rise of the Posthuman Brain: Computational
Neuroscience, Digital Networks, and the `In Silico Cerebral Subject.''
\emph{Trans"-Humanities} 7 (3): 79--102.

Strasser, Peter. 2014. \emph{Diktatur des Gehirns. Für eine Philosophie
des Geistes.} Paderborn: Fink.

Sumeet, Jain, e Sushrut Jadhav. 2009. ``Pills That Swallow Policy:
Clinical Ethnography of a Community Mental Health Program in Northern
India.'' \emph{Transcultural Psychiatry} 46:60--85.

Summerfield, Derek. 2008. ``How Scientifically Valid Is the Knowledge
Base of Global Mental Health?'' \emph{British Medical Journal} 336
(7651): 992--994.

\_\_\_\_\_. 2012. ``Afterword: Against `Global Mental Health.'''
\emph{Transcultural Psychiatry} 49 (3/4): 519.

\_\_\_\_\_. 2014. ``A Short Conversation with Arthur Kleinman About His
Support for the Global Mental Health Movement.'' \emph{Disability and
the Global South} 1 (2): 406--411.

Swain, John, e Colin Cameron. 1999. ``Unless Otherwise Stated:
Discourses of Labeling and Identity in Coming Out.'' In Corker e French
1999.

Tabbi, Joseph. 2008. ``\emph{Afterthoughts on The Echo Maker}.'' In Burn
e Dempsey 2008, 219--229.

Tadd, James Liberty. 1900. \emph{New Methods in Education}. London:
Sampson Low, Marston \& Co.

Tallis, Raymond. 2004. \emph{Why the Mind Is Not a Computer: A Pocket
Lexicon of Neuromythology.} Exeter: Imprint Academic.

\_\_\_\_\_. 2007. ``Not All in the Brain.'' \emph{Brain} 130 (11):
3050--3054.

\_\_\_\_\_. 2008a. ``The Neuroscience Delusion.'' \emph{Times Literary
Supplement} (9 Abr.).

\_\_\_\_\_. 2008b. ``The Limitations of a Neurological Approach to Art
{[}Review of Onians 2007{]}.'' \emph{The Lancet} 372 (5 Jul.): 19--20.

\_\_\_\_\_. 2009. ``Neurotrash.'' \emph{New Humanist} 124 (6).
\textless{}\emph{https://bit.ly/2OAdQu6}\textgreater{}

Tan, Li"-Hai, Angela R. Laird, Karl Li, e Peter T. Fox. 2005.
``Neuroanatomical Correlates of Phonological Processing of Chinese
Characters and Alphabetic Words: A Meta"-analysis.'' \emph{Human Brain
Mapping} 25:83--91.

Tang, Yi"-Yuan, Britta K. Hölzel, e Michael I. Posner. 2015. ``The
Neuroscience of Mindfulness Meditation.'' \emph{Nature Reviews
Neuroscience} 16:213--225.

Tang, Yi"-Yuan, e Michael I. Posner. 2013. ``Editorial: Special Issue on
Mindfulness Neuroscience.'' \emph{\versal{SCAN}} 8:1--3.

Tang, Yi"-yuan, Wutian Zhang, Kewei Chen, et al. 2006. ``Arithmetic
Processing in the Brain Shaped by Cultures.'' \emph{\versal{PNAS}} 103 (28):
10775--10780.

Tannen, Susan. n.d. ``Mental fitness---Exercises for the Brain.''
\textless{}\emph{https://bit.ly/2VubYnZ}\textgreater{}

Tauber, Alfred. 2012. ``The Biological Notion of Self and Non"-self.''
\emph{Stanford Encyclopedia of Philosophy.}
\textless{}\emph{http://plato.stanford.edu/entries/biology-self/}\textgreater{}.

Taylor, Charles. 1989. \emph{Sources of the Self: The Making of the
Modern Identity.} Cambridge: Mass.: Harvard University Press. (trad.
port.: \emph{As Fontes do Self: A Construção da Identidade Moderna.} 4
ed\emph{.} São Paulo: Loyola, 2013).

Taylor, Richard P. 2002. ``Order in Pollock's Chaos.'' \emph{Scientific
American} (Dez.): 116--121.

Taylor, Richard P., Adam P. Micolich, e David Jonas. 1999. ``Fractal
Analysis of Pollock's Drip Paintings.'' \emph{Nature} 399 (3 Jun.): 422.

Teahan, John F. 1979. ``Warren Felt Evans and Mental Healing: Romantic
Idealism and Practical Mysticism in Nineteenth"-Century America.''
\emph{Church History} 48 (1): 63--80.

Temkin, Owsei. 1973. \emph{Galenism: Rise and Decline of a Medical
Philosophy}. Ithaca, N.Y.: Cornell University Press.

Theil, Stefan. 2015. ``Why the Human Brain Project Went Wrong---and How
to Fix It.'' \emph{Scientific American}, (1 Outubro).
\textless{}\emph{https://bit.ly/1KzuwES}\textgreater{}

Thiel, Udo. 2011. \emph{The Early Modern Subject: Self"-Consciousness and
Personal Identity from Descartes to Hume.} New York: Oxford University
Press.

Thoma, Nathan C., Dean McKay, Andrew J. Gerber, et al. 2012. ``A
Quality"-Based Review of Randomized Controlled Trials of
Cognitive"-Behavioural Therapy for Depression: An Assessment and
Metaregression.'' \emph{American Journal of Psychiatry} 169:22--30.

Thomas, Julia Adeney. 2015. ``Who Is the `We' Endangered by Climate
Change?'' \emph{In Endangerment, Biodiversity, and Culture}, Org.
Fernando Vidal e Nélia Dias. New York: Routledge.

Thompson, Paul M., Jason L. Stein, Sarah E. Medland, et al. 2014. ``The
\versal{ENIGMA} Consortium: Large"-Scale Collaborative Analyses of Neuroimaging
and Genetic Data.'' \emph{Brain Imaging and Behavior} 8 (2): 153--182.

Thrailkill, Jane F. 2011. ``Ian McEwan's Neurological Novel.''
\emph{Poetics Today} 32 (1): 171--201.

Tinio, Pablo P. L., e Jeffrey K. Smith, Orgs. 2014. \emph{The Cambridge
Handbook of the Psychology of Aesthetics and the Arts.} New York:
Cambridge University Press.

Tofts, Darren, Annemarie Jonson, e Alessio Cavallaro, Orgs. 2004.
\emph{Prefiguring Cyberculture: An Intellectual History.} Sydney: \versal{MIT}
Press.

Tougaw, Jason. 2012. ``Brain Memoirs, Neuroscience, and the Self: A
Review Article.'' \emph{Literature and Medicine} 30 (1): 171--192.

\_\_\_\_\_. 2016. ``Amnesia and Identity in Contemporary
Literature\emph{.''} In \emph{Memory in the Twenty"-First Century: New
Critical Perspectives from the Arts, Humanities, and Sciences}, Org.
Sebastian Groes, 280--285. New York: Palgrave Macmillan.

Toyokawa, Satoshi, Monica Uddin, Karestan C. Koenen, e Sandro Galea.
2012. ``How Does the Social Environment `Get into the Mind'? Epigenetics
at the Intersection of Social and Psychiatric Epidemiology.''
\emph{Social Science and Medicine} 74:67--74.

Tracy, Harry M. 2016. ``The Neuro Funding Rollecoaster.''
\emph{Cerebrum} (Jun.).
\textless{}\emph{http://www.dana.org/ Cerebrum/2016/The\_Neuro\_Funding\_Rollercoaster/}\textgreater{}

Tsur, Reuven. 1992. \emph{Toward a Theory of Cognitive Poetics.}
Amsterdam: North"-Holland.

Turner, D. Trevor, Mark van der Gaag, Eirini Karyotaki, e Pim Cuijpers.
2014. ``Psychological Interventions for Psychosis: A Meta"-analysis of
Comparative Outcome Studies.'' \emph{American Journal of Psychiatry}
171:523--538.

Turner, Erick H., Annette M. Matthews, Eftihia Linardatos, et al. 2008.
``Selective Publication of Antidepressant Trials and Its Influence on
Apparent Efficacy.'' \emph{New England Journal of Medicine}
358:252--260.

Tylor, Edward B. 1871. \emph{Primitive Culture: Researches into the
Development of Mythology, Philosophy, Religion, Art, and Custom.}
London: John Murray.

Umiltà, Maria Alessandra, Cristina Berchio, Mariateresa Sestito, et al.
2012. ``Abstract Art and Cortical Motor Activation: An \versal{EEG} Study.''
\emph{Frontiers in Human Neuroscience} 6, art. 311: 1--9.

Uttal, William R. 2003. \emph{The New Phrenology: The Limits of
Localizing Cognitive Processes in the Brain.} Cambridge, Mass.: \versal{MIT}
Press.

\_\_\_\_\_. 2015. \emph{Macroneural Theories in Cognitive Neuroscience.}
New York: Psychology Press.

Valenstein, Elliot. S. 1998. \emph{Blaming the Brain: The Truth About
Drugs and Mental Health.} New York: The Free Press.

Valentine, Gill, Tracey Skelton, e Ruth Butler. 2003. ``Coming Out and
Outcomes: Negotiating Lesbian and Gay Identities with, and in, the
Family.'' \emph{Environment and Planning D: Society and Space} 21 (4):
479--499.

Van Orden, Guy C., e Kenneth R. Paap. 1997. ``Functional Neuroimages
Fail to Discover Pieces of the Mind in the Parts of the Brain.''
\emph{Philosophy of Science} 64 (Proceedings): S85--S94.

van Praag, Herman M. 2000. ``Nosologomania: A Disorder of Psychiatry.''
\emph{World Journal of Biological Psychiatry} 1:151--158.

\_\_\_\_\_. 2005. ``Can Stress Cause Depression?'' \emph{World Journal of
Biological Psychiatry} 6 (Suplemento 2): 5--22.

\_\_\_\_\_. 2008. ``Kraepelin, Biological Psychiatry, and Beyond.''
\emph{European Archives of Psychiatry and Clinical Neuroscience} 258
(Suplemento 2): 29--32.

\_\_\_\_\_. 2010. ``Biological Psychiatry: Still Marching Forward in a
Dead End.'' \emph{World Psychiatry} 9 (3): 164--165.

van Praag, Herman M., Rene S. Kahn, Gregory M. Asnis, et al. 1987.
``Denosologization of Biological Psychiatry or the Specificity of 5-\versal{HT}
Disturbances in Psychiatric Disorders.'' \emph{Journal of Affective
Disorders} 13:1--8.

Van Wyhe, John. 2002. ``The Authority of Human Nature: The Schädellehre
of Franz Joseph Gall.'' \emph{British Journal for the History of
Science} 35:17--42.

\_\_\_\_\_. 2004\emph{. Phrenology and the Origins of Victorian
Scientific Naturalism.} Aldershot: Ashgate.

Vander Valk, Frank, Org. 2012a. \emph{Neuroscience and Political Theory:
Thinking the Body Politic.} Routledge: London

\_\_\_\_\_. 2012b. ``Introduction''. In Vander Valk 2012a, 1--22.

Vartanian, Oshin. 2014. ``Empirical Aesthetics: Hindsight and
Foresight.'' In Tinio e Smith 2014.

Veer, Ilya M., Christian F. Beckmann, Marie"-José van Tol, et al. 2010.
``Whole Brain Resting"-State Analysis Reveals Decreased Functional
Connectivity in Major Depression.'' \emph{Frontiers in Systems
Neuroscience} 4, art.41. doi:10.3389/fnsys.2010.00041.

Vidal, Fernando. 2009a. ``Brainhood, Anthropological Figure of
Modernity.'' \emph{History of the Human Sciences} 22 (1): 5--36.

\_\_\_\_\_. 2009b. ``Ectobrains in the Movies.'' \emph{In The Fragment:
An Incomplete History}, Org. William Tronzo, 193--211. Los Angeles:
Getty Research Institute.

\_\_\_\_\_. 2011. \emph{Sciences of the Soul: The Early Modern Origins of
Psychology}. Tradução de Saskia Brown. Chicago: University of Chicago
Press.

\_\_\_\_\_. 2016. ``Frankenstein's Brain: `The Final Touch.'''
\emph{SubStance} 45 (2): 88--117.

Vidal, Fernando, e Francisco Ortega. 2011. ``Approaching the
Neurocultural Spectrum: An Introduction.'' In Ortega e Vidal 2011,
7--27.

Vrecko, Scott. 2006. ``Folk Neurology and the Remaking of Identity.''
\emph{Molecular Interventions} 6:300--303.

Vuilleumier, Patrik, Jorge L. Armony, Jon Driver, e Raymond J. Dolan.
2003. ``Distinct Spatial Frequency Sensitivities for Processing Faces
and Emotional Expressions.'' \emph{Nature Neuroscience} 6 (6): 624--631.

Wajman, José Roberto, Paulo H. Bertolucci, Leticia Mansur, e Serge
Gauthier. 2015. ``Culture as a Variable in Neuroscience and Clinical
Neuropsychology: A Comprehensive Review.'' \emph{Dementia and
Neuropsychologia} 9 (3): 203--218.

Waldman, Paul. 2013. ``David Brooks and the Anti"-Neuroscience
Backlash.'' \emph{The American Prospect} (18 Jun.).
\textless{}\emph{https://bit.ly/2AVZ8oS}\textgreater{}

Walsh, Pat, Mayada Elsabbagh, Patrick Bolton, e Ilina Singh. 2011. ``In
Search of Biomarkers for Autism: Scientific, Social, and Ethical
Challenges.'' \emph{Nature Reviews Neuroscience} 12:603--612.

Walton, Alice. 2015. ``7 Ways Meditation Can Actually Change the
Brain.''\\
\textless{}\emph{https://bit.ly/2Ox6pUk}\textgreater{}.

Waltz, M. 2005. ``Reading Case Studies of People with Autistic Spectrum
Disorders: A Cultural Studies Approach to Issues of Disability
Representation.'' \emph{Disability and Society} 20 (5): 421--435.

Wang, Lin, Daniel F. Hermens, Ian B. Hickie, e Jim Lagopoulos. 2012. ``A
Systematic Review of Resting"-State Functional"-\versal{MRI} Studies in Major
Depression.'' \emph{Journal of Affective Disorders} 142:6--12.

Warnick, Jason E., e Dan Landis, Orgs. 2015. \emph{Neuroscience in
Intercultural Contexts}. New York: Springer.

Watters, Ethan. 2010. \emph{Crazy Like Us: The Globalization of the
American Psyche.} New York: Simon and Schuster.

Wazana, Ashley. 2000. ``Physicians and the Pharmaceutical Industry: Is a
Gift Ever Just a Gift?'' \emph{Journal of the American Medical
Association} 283 (3): 373--380.

Weber, Matthew J., e Sharon L. Thompson"-Schill. 2010. ``Functional
Neuroimaging Can Support Causal Claims About Brain Function.''
\emph{Journal of Cognitive Neuroscience} 22 (11): 2415--2416.

Weinmann, Stefan, John Read, e Volkmar Aderhold. 2009. ``Influence of
Antipsychotics on Mortality in Schizophrenia: Systematic Review.''
\emph{Schizophrenia Research} 113:1--11.

Weintraub, Kit. 2005. ``A Mother's Perspective.''
\textless{}\emph{https://bit.ly/2VrtZmO}\textgreater{}

Wells, Carol G. 1989. \emph{Right Brain Sex: Using Creative
Visualization to Enhance Sexual Pleasure.} New York: Simon \&
Schuster\emph{. }

Wendell, Susan. 1996. \emph{The Rejected Body: Feminist Philosophical
Reflections on Disability}. New York: Routledge.

Western, Drew. 2008. \emph{The Political Brain: The Role of Emotion in
Deciding the Fate of the Nation.} Philadelphia: Public Affairs. (trad.
port.: \emph{Cérebro Político -- O Papel da Emoção na Decisão: O Destino
da Nação.} São Paulo: Unianchieta, 2008).

Whelan, Robert, e Hugh Garavan. 2014. ``When Optimism Hurts: Inflated
Predictions in Psychiatric Neuroimaging.'' \emph{Biological Psychiatry}
75:746--748.

Whitaker, Robert. 2010. \emph{Anatomy of an Epidemic: Magic Bullets,
Psychiatric Drugs, and the Astonishing Rise of Mental Illness in
America.} New York: Crown. (trad. port.: \emph{Anatomia de uma Epidemia:
pílulas mágicas, drogas psiquiátricas e o aumento assombroso da doença
mental.} Rio de Janeiro: Editora Fiocruz, 2017).

White, Ross. 2013. ``The Globalisation of Mental Illness.'' \emph{The
Psychologist} 26 (3): 182--185.

Whorton, James C. 1982. \emph{Crusaders for Fitness: The History of
American Health Reformers.} Princeton, N.J.: Princeton University Press.

Wickelgren, Ingrid. 2005. ``Autistic Brains out of Synch?''
\emph{Science} 308:1856--1858.

Wigan, Arthur L. 1985 {[}1844{]}. \emph{A New View of Insanity: The
Duality of the Mind Proved by the Structure, Functions, and Diseases of
the Brain and by the Phenomena of Mental Derangement, and Shown to Be
Essential to Moral Responsibility.} Malibu: Joseph Simon.

Wijdicks, Eelco F. M. 2012. ``The Transatlantic Divide Over Brain Death
Determination and the Debate.'' \emph{Brain} 135 (4): 1321--1331.

Wilfond, Benjamin S., e Vardit Ravitsky. 2005. ``On the Proliferation of
Bioethics Subdisciplines: Do We Really Need `Genethics' and
`Neuroethics'?'' \emph{American Journal of Bioethics} 5 (2): 20--21.

Wilkes, Kathleen. 1988. \emph{Real People: Personal Identity Without
Thought Experiments.} Oxford: Clarendon.

Williams, J. Simon, Stephen Katz, e Paul Martin. 2011. ``The
Neuro"-Complex: Some Comments and Convergences.'' \emph{MediaTropes} 3
(1): 135--146.

Williams, Raymond. 1985. \emph{Keywords: A Vocabulary of Culture and
Society}. Rev. ed. London: Fontana. (trad. port.: \emph{Palavras"-Chave:
Um Vocabulário de Cultura e Sociedade.} Tradução de Sandra Guardini
Vasconcelos. São Paulo: Boitempo, 2007.)

Willis, Thomas. 1681 {[}1664{]}. \emph{The Anatomy of the Brain and
Nerves.} Tradução de Samuel Pordage. Birmingham, Ala.: The Classics of
Neurology \& Neurosurgery Library, 1983.

\_\_\_\_\_. 1683 {[}1971{]}. \emph{Two Discourses Concerning the Soul of
Brutes . . . {[}De anima brutorum,} 1672\emph{{]}}. Tradução de Samuel
Pordage. Gainsville, Fla.: Scholars' Facsimiles \& Reprints.

Wing, Lorna. 1997. ``The History of Ideas on Autism: Legends, Myths, and
Reality.'' \emph{Autism} 1:13--23.

Winter, Arthur, e Ruth Winter. 1987. \emph{Build Your Brain Power: The
Latest Techniques to Preserve, Restore, and Improve Your Brain's
Potential.} New York: St. Martin's Press. (trad. port.: \emph{Como
Desenvolver o Poder da Mente}. São Paulo: Cultrix, 1997).

Witelson, Sandra F., Debra L. Kigar, e Thomas Harvey. 1999. ``The
Exceptional Brain of Albert Einstein.'' \emph{The Lancet}
353:2149--2153.

Wojciehowski, Hannah Chapelle, e Vittorio Gallese. 2011. ``How Stories
Make Us Feel: Toward an Embodied Narratology.'' \emph{California Italian
Studies} 2 (1). \textless{}\emph{http://escholarship.org/uc/item/3jg726c2}\textgreater{}

Wolbring, Gregor. 2007. ``Neurodiversity, Neuroenhancement,
Neurodisease, and Neurobusiness.'' \emph{Innovation Watch} (15 Maio).
\textless{}\emph{http://www.innovationwatch.com/choiceisyours/choiceisyours-2007-04-30.htm}\textgreater{}

Wolpe, Paul Root. 2002. ``The Neuroscience Revolution.'' \emph{The
Hastings Center Report} 32 (4): 8.

Wright, T. R. 1982. ``From Bumps to Morals: The Phrenological Background
to George Eliot's Moral Framework.'' \emph{Review of English Studies} 33
(129): 24--46.

Xie, Changchun, Zheng Wu, Weizhong Li, et al. 2008. ``Neural Correlates
of Depression in Subjects with Amnestic Mild Cognitive Impairment.''
\emph{Alzheimer's and Dementia} 4 (4), Supplement 1: T259--T260.

Young, Kay, e Jeffrey L. Saver. 2001. ``The Neurology of Narrative.''
\emph{SubStance: A Review of Theory and Literary Criticism} 30 (1/2):
72--84.

Young, Robert Maxwell. 1990. \emph{Mind, Brain, and Adaptation in the
Nineteenth Century: Cerebral Localization and Its Biological Context
from Gall to Ferrier.} New York: Oxford University Press.

Yuste, Rafael. 2015. ```Cuando entendamos el cerebro, la humanidad se
entenderá a sí misma' {[}Entrevista com Núria Jar Benabarre{]}.''
\emph{El País} (25 Maio).
\textless{}\emph{https://bit.ly/32057oE}\textgreater{}

Zaidel, Dahlia W. 2013. ``Art and Brain: The Relationship of Biology and
Evolution to Art.'' In Finger, Zaidel, Boller, e Bogousslavsky 2013.

Zalewski, Daniel. 2009. ``Ian McEwan's Art of Unease.'' \emph{The New
Yorker} (23 Fev.).
\textless{}\emph{https://bit.ly/2AWbCwP}\textgreater{}.

Zawidzki, Tadeusz, e William P. Bechtel. 2005. ``Gall's Legacy
Revisited: Decomposition and Localization in Cognitive Neurosciences.''
\emph{In The Mind as a Scientific Object: Between Brain and Culture},
Org. Christina E. Erneling e David Martel Johnson. Oxford: Oxford
University Press.

Zeki, Semir. 1998. ``Art and the Brain.'' \emph{Daedalus} 127 (2):
71--103.

\_\_\_\_\_. 2000. ``L'artiste à sa manière est un neurologue.'' \emph{La
Recherche hors"-série} 4 (Novembro): 98--100.

\_\_\_\_\_. 2001. ``Artistic Creativity and the Brain.'' \emph{Science}
263 (6 Julho): 51--52.

\_\_\_\_\_. 2002. ``Neural Concept Formation and Art. Dante,
Michelangelo, Wagner.'' \emph{Journal of Consciousness Studies}
9:53--76.

\_\_\_\_\_. n.d. ``Statement on Neuroaesthetics.''
\textless{}\emph{https://bit.ly/35k6xfR}\textgreater{}

Zeki, Semir, e Tomohiro Ishizu. 2013. ``The `Visual Shock' of Francis
Bacon: An Essay in Neuroesthetics.'' \emph{Frontiers in Human
Neuroscience} 7:850. doi:10.3389/ fnhum.2013.00850.

Zhou, Haotian, e John Cacioppo. 2010. ``Culture and the Brain:
Opportunities and Obstacles.'' \emph{Asian Journal of Social Psychology}
13:59--71.

Zhu, Ying, Li Zhang, Jin Fan, e Shihu Han. 2007. ``Neural Basis of
Cultural Influence on Self"-Representation.'' \emph{NeuroImage}
34:1310--1316.

Zhu, Ying, e Shihu Han. 2008. ``Cultural Differences in the Self: From
Philosophy to Psychology and Neuroscience.'' \emph{Social and
Personality Psychology Compass} 2 (5): 1799--1811.

Zunshine, Lisa, Org. 2010. \emph{Introduction to Cognitive Cultural
Studies.} Baltimore, Md.: Johns Hopkins University Press.

Zwijnenberg, Robert. 2011. ``Brains, Art, and the Humanities.'' In
Ortega e Vidal 2011, 293--309.
\end{Parskip}
